\begin{document}
\frontmatter

% página 1
\newpage
\thispagestyle{empty}
{\textcolor{white}{.}}

% página 2
\newpage
\thispagestyle{empty}
{\textcolor{white}{.}}

% página 3
\newpage
\thispagestyle{empty}
{\textcolor{white}{.}}

\vspace{30mm}

\begin{center}
	\LARGE{Senderos sociológicos}
\end{center}

% página 4
\newpage
\thispagestyle{empty}
{\textcolor{white}{.}}

% página 5
\newpage
\thispagestyle{empty}
\begin{center}%,draft
{\sc\large{esteban ezequiel vila}}\\ %compiladoras
\end{center}

\vspace{30mm}

\begin{center}
\LARGE{Senderos sociológicos}\\\vspace{10mm}

\Large{Trayectorias de sociólogos argentinos\\ en el siglo XX}
\end{center}

\vfill

\begin{figure}[b]
\centering
\includegraphics[width=20mm]{./media/logo-imago-ByW.png}
\end{figure}

% página 6
\newpage
\thispagestyle{empty}
%\begin{figure}[t]
%\centering
%\vspace{-10mm}
%\includegraphics[width=20mm]{./media/desconocido.png}\\
%\end{figure}

\noindent Esteban Ezequiel Vila \\
\noindent Senderos sociológicos. Trayectorias de sociólogos argentinos en el siglo XX. 1\sptext{ra} ed. Buenos Aires: 2025.\\
\noindent \ztotpages\ p.; \valorEspecifico. ISBN 978-950-793-484-1 \\
\noindent 1. Sociología. I. Título \\
\noindent CDD 301\\
\noindent Fecha de catalogación: 12/07/2025 \\
\noindent \textcopyright~2025, Esteban Ezequiel Vila \\
\noindent \textcopyright~2025, Ediciones Imago Mundi\\
%\noindent Imagen de tapa: .\\
\noindent Hecho el depósito que marca la ley 11.723\\
\noindent Impreso en Argentina, tirada de esta edición: 150 ejemplares\\

\vfill

\noindent \qrcode[height=2cm]{https://www.edicionesimagomundi.com/}\bigskip

\noindent Ninguna parte de esta publicación, incluido el diseño de cubierta, puede ser reproducida, almacenada o transmitida de manera alguna ni por ningún medio, ya sea eléctrico, químico, mecánico, óptico, de grabación o de fotocopia, sin permiso previo por escrito del editor. Este libro se terminó de imprimir en el mes de octubre de 2025 en San Carlos Impresiones, Virrey Liniers 2203, Ciudad Autónoma de Buenos Aires, República Argentina.


\Author{Sumario}
\tableofcontents

\chapter{Presentación}
\Author{Esteban Ezequiel Vila}

Recuerdo que cuando cursaba Historia Social Latinoamericana en la Facultad de Humanidades de la Universidad Nacional de La Plata, en paralelo a las lecturas obligatorias de la materia, como Víctor Bulmer-Thomas, Tulio Halperín Donghi, Waldo Ansaldi o Fernando Mires, me encontraba leyendo el último volumen de la trilogía \emph{Memorias del Fuego} de Eduardo Galeano. Allí también estaban Augusto César Sandino, José Batlle y Ordóñez, Eloy Alfaro, Getúlio Vargas y otros tantos destacados nombres del siglo~XX latinoamericano, aunque narrados de otra forma, es decir, a partir de anécdotas personales más o menos pintorescas y menudencias de la vida cotidiana que uno no se imagina que les ocurran a personas extraordinarias.

Esta forma de contar la historia, alejada de los grandes procesos sociales, las estructuras y los papeles históricos que cumplir, a veces incluso en un devenir ineluctable que escaparía a las voluntades individuales, supone posicionarse en un nivel de observación que no es el más común en sociología. Este libro, compuesto de trabajos publicados a lo largo de los últimos años, recupera ese espíritu al centrarse en las trayectorias de algunos destacados sociólogos argentinos del siglo~XX. El objetivo es dar cuenta de algunos procesos históricos del devenir de la sociología argentina, desde los inicios de la pasada centuria hasta los comienzos de la actual, a partir de dichas trayectorias. Esto quiere decir que, aunque los capítulos que componen el libro están centrados en derroteros individuales, es posible encontrar en ellos elementos que den cuenta de procesos sociales que los exceden.

Así, la centralidad de la sociología cordobesa durante el período de entreguerras; la supervivencia del \enquote{estilo de trabajo} de la \enquote{sociología de cátedra} más allá de 1957; el nacimiento de la investigación empírica en la Universidad Nacional de Córdoba; el carácter marcadamente antiperonista de la \enquote{sociología científica}; la transición teórica desde esta última tradición a la \enquote{sociología marxista}; y, finalmente, su posterior moderación a partir de la adopción de posiciones socialdemócratas en el período democrático abierto en 1983, son algunas cuestiones que podrán encontrarse en los capítulos que componen este trabajo.

Por cierto, si el lector posee conocimiento del campo, notará que las trayectorias trabajadas no son las de los miembros más famosos de las tradiciones sociológicas de las que forman parte. De modo que, los capítulos dedicados a la \enquote{sociología de cátedra} no abordan la vida de Alfredo Poviña; los dos siguientes, sobre la \enquote{sociología científica}, no se enfocan en Gino Germani; y, cuando en las últimas dos secciones se estudien exponentes de la \enquote{sociología marxista}, tampoco aparecerá Juan Carlos Portantiero. Esto se debe a que, como se ha dicho, la intención ha sido descifrar qué es lo que estas trayectorias tienen para decirnos sobre la vida institucional e intelectual de la sociología argentina en el siglo~XX.

Última indicación sobre lo que el lector no encontrará en este trabajo. Se trata de tres ausencias que, por distintas razones, han escapado a mis posibilidades. La primera, no hay un capítulo dedicado a la \enquote{sociología nacional/peronista}, cuya figura más destacada fue Roberto Carri. En este caso, se debe a que las \enquote{cátedras nacionales} se han convertido en un tema tan rigurosamente trabajado que lo han convertido en un objeto en sí mismo, lo cual me dificultaría poder decir algo novedoso y relevante al respecto. Segunda ausencia, la \enquote{sociología católica} que, si bien es una tradición mucho menos explorada, por lo que brinda mayores posibilidades de investigación, se trata de una agenda para un futuro próximo. Finalmente, la cuestión de género también se encuentra ausente\footnote{Aquí vale mencionar una investigación reciente, \emph{La universidad como espacio biográfico} (2022) de Sandra Carli, donde se abordan trayectorias de sociólogas argentinas.}, aunque esto tiene que ver más con la falta de fuentes disponibles que de interés por las trayectorias de algunas sociólogas destacadas, como Yolanda Bórquez (UNCuyo) o María Angelina Roggero (UNLP).

\froufrou

Como corresponde a este tipo de ocasiones, quisiera agradecer a varias personas que participaron de un modo u otro en lo que termina siendo este trabajo. Ordenados por el azar, agradezco a Guido Giorgi, Severino Fernández, Hernán González Bollo, Miranda Lida, Luis Escobar, Pablo Requena, Ezequiel Grisendi, Alejandro Blanco y, especialmente, a Ada Caracciolo, recientemente fallecida. Con todos ellos hemos tenido intercambios en distintas instancias sobre varias de las trayectorias que aquí se estudian. Además, muy especialmente, a Diego Pereyra, quien generosamente se prestó a escribir el epílogo. Este libro está dedicado a mis padres, Néstor Vila y Rosa Marson, y a mi esposa, Carina Sohn.

\mainmatter

%Capítulo 1
\chapter{La apuesta por la sociología. Una reconstrucción de las trayectorias de Ricardo Levene y Raúl Orgaz en perspectiva comparada}

\footnote{\textsuperscript{*} Publicado en \emph{Astrolabio. Nueva época} n.º 27, págs. 193--218. https://doi.org/10.55441/1668.7515.n27.27581.}

\epigraph{\enquote{Solo se explica comparando}.}{\textcite{1617-DURKHEIM2006}}

\section{Introducción}

Es una de las premisas más importantes de la sociología del conocimiento que la toma de decisiones en la vida cotidiana de los agentes debe explicarse por referencia a los orígenes sociales de las formas de pensamiento que guían sus acciones. Sin desconocer la necesidad de atender a las instancias de creatividad individuales tanto como a las interacciones entre los agentes sociales para explicar el surgimiento y los cambios de ideas \parencite{1527-SALES2007}, es en este sentido que debe entenderse la consabida frase de Karl Mannheim (1986:3), para quien sería un error sostener que un individuo aislado piensa y, por el contrario, \enquote{habría que decir más bien que participa en el pensamiento de otros hombres que han pensado antes que él}. Por otro lado, más allá de la situación establecida y los modos preformados de pensamiento y acción ligados a la dinámica del grupo de pertenencia, debería señalarse que la propia reflexión acerca de las raíces sociales de las maneras de pensar responde a una situación social específica, que es la que se experimenta en las sociedades con movilidad social.

Si bien es cierto que en todas las sociedades existieron grupos sociales cuya tarea específica consistió en proveer a sus contemporáneos de una interpretación legítima del mundo (grupo a quienes Mannheim llamaba \enquote{intelectuales}), también es verdad que en la medida en que esas sociedades permanecieron estáticas los miembros de dichos grupos mantuvieron el monopolio de la interpretación, ya que sus \enquote{conflictos intelectuales} no encarnaban desavenencias de orden social. Es solo en presencia de sociedades escindidas en clases sociales que una \emph{intelligentsia} libre pudo surgir, siendo

\begin{quote}
(\dots) su principal característica (\dots) que se recluta, cada vez más, en capas sociales y en situaciones que varían constantemente, y que su modo de pensar ha dejado de estar regulado por un género de organización cerrada como el de la casta. Debido a la falta de una organización propia (\dots) los intelectuales han permitido que ganaran audiencia esas formas de pensamiento y de experiencia que compiten abiertamente entre sí en el mundo más amplio de las otras capas sociales (\dots) los intelectuales aceptaron en una forma aún más acentuada los diversos modos de pensamiento y de experiencia que existían en la sociedad y los esgrimieron unos contra otros. Lo hicieron tanto más cuanto que tenían que competir para conquistar el favor de un público que, a diferencia del público del clero, había dejado de serles accesible sin algún esfuerzo de su parte (Mannheim, 1986:10-11).
\end{quote}

El presente trabajo ahonda en este asunto. Esto es, en términos generales, de qué manera dos agentes que, por sus orígenes sociales, en sociedades tradicionales no habrían estado destinados a cumplir el rol de intelectuales, fueron reclutados para formar parte del \emph{staff} establecido de productores culturales de su sociedad. Si se acepta la tesis de la sociología del conocimiento que sostiene que en las sociedades modernas la actividad intelectual no es un fenómeno restringido a un estrato social rigurosamente definido \parencite{1513-SHILS1976,1521-BAUMAN1997,1576-BOURDIEU1999}(Mannheim, 1957), se hace ostensible el hecho de que miembros de clases sociales dominadas solo pueden integrar el grupo consagrado a la producción de símbolos en el tipo social moderno.

Al mismo tiempo, se intentará comprender el sentido de las apuestas intelectuales realizadas por estos agentes en relación con una disciplina en particular (la sociología), durante el período en que ostentaron los cargos de profesores titulares de cátedras universitarias dedicadas a esta materia. Por \enquote{apuesta intelectual}, aquí deben entenderse las prácticas que acompañaron el progreso de sus carreras en la docencia universitaria, lo cual implica relacionar sus diferentes inversiones en determinados campos del saber con el capital global poseído. Esto fue lo que definió en cada momento sus posibilidades objetivas de obtener beneficios y, por lo tanto, la \enquote{razonabilidad} de las inversiones y desinversiones realizadas \parencite{1576-BOURDIEU1999}.

Antes de avanzar, vale hacer algunas aclaraciones. En primer término, deben ser tenidos en cuenta algunos determinantes estructurales de la vida intelectual para poder dar cuenta del surgimiento de ciertos tipos de pensamiento y no de otros. En su momento, el propio Mannheim (1971) indicó que, como consecuencia de la aplicación del método de análisis sociológico a la vida intelectual, ciertos fenómenos originalmente diagnosticados como manifestaciones de leyes inmanentes del pensamiento pudieron ser entendidos como el producto de patrones estructurales de la sociedad. De tal manera que la evolución y el cambio del pensamiento podría comprenderse a partir de dos determinaciones estructurales: por un lado, la competencia, la cual no puede reducirse a ser un fenómeno de la economía, ya que también es constitutiva de todo producto cultural y, por otro lado, la existencia de generaciones, lo cual da cuenta de la existencia común de problemas en la medida en que las características cualitativas de un objeto determinado solo resultan accesibles para pensamientos situados dentro de una misma estructura.

Estos elementos solo aparecen en las sociedades modernas, donde la diferenciación y la especialización se producen \emph{vis à vis} la constitución las clases sociales. Es en presencia del tipo social complejo donde se evidencia una discrepancia de cosmovisiones presentes en los agentes, lo cual se enraíza en sus pertenencias a grupos sociales diversos, aunque aquellas no agotan su explicación en los orígenes sociales de los agentes que las producen. Son también las redes intelectuales, las rivalidades, etcétera, es decir, en pocas palabras, los cambios históricos en las bases materiales de la vida intelectual los que también explican el surgimiento de las ideas y sus cambios \parencite{1584-COLLINS2000}. Esto no se debe a que los cambios mencionados produzcan cosmovisiones de forma inmediata, sino a que modifican las condiciones en las cuales se producen las ideas. Por lo tanto, evitando el peligro de caer en el determinismo propio del marxismo vulgar, que explica la superestructura jurídico-política y la conciencia social por la base económica, aquí se sostiene que \enquote{solo explorando la variedad de las formaciones grupales ---generaciones, grupos de status, sectas, grupos ocupacionales, etcétera--- y sus modos característicos de pensamiento puede descubrirse una base existencial correspondiente a la gran variedad de perspectivas y conocimientos que realmente surgen} (Merton, 1977:58).

Sin embargo, como se ha dicho, esto no quiere decir que las ideas sean solamente imputables a la base material en la cual están insertos los agentes sociales. Existe también una gran diversidad de acervo de conocimientos, creencias, tradiciones, etcétera, que exceden a la mera imputación causal correspondiente a la clase social de origen, sin que por esto se deba caer en la indeterminación y la imposibilidad de explicar el surgimiento de las ideas. El trabajo empírico de reconstrucción de trayectorias, como se intentará demostrar, puede dar cuenta de los motivos concretos por los que agentes con propiedades sociales similares tienen discrepancias en sus representaciones y realizan apuestas en campos distintos. Se partirá, no obstante, de una característica en común que presentan ambos agentes a estudiar y que los vincula con un elemento distintivo en lo que hace a la producción de ideas, propia de los miembros de la comunidad más amplia de pertenencia: la capacidad de innovación.

Robert Merton ha señalado bajo qué condiciones la desviación de la conducta (\emph{i. e.} innovación) respecto de los códigos sociales establecidos podría considerarse como algo \enquote{que puede esperarse}. Merton entendía que las estructuras sociales ejercían una presión definida sobre ciertas personas de la sociedad para que sigan una conducta inconformista y no una conformista respecto del \emph{statu quo}. Esto era, en definitiva, lo que se encontraba detrás de la anomia, según era entendida por el autor. Por lo tanto, este concepto refiere en su obra a una de las formas en que los agentes pueden responder a las coacciones de la estructura social. Merton introduce así la variable de la estratificación social en su teoría de la anomia, haciendo depender el grado de coacción ejercido sobre un agente de su posición al interior de un grupo. De esta manera, la situación anómica tiene lugar cuando los agentes adquieren compromisos emocionales fuertes con los fines que la sociedad propone, al tiempo que no se los provee de las herramientas que les posibilitarían alcanzarlos a través de medios considerados socialmente válidos. Esta actitud \enquote{pueden adoptarla individuos de todos los estratos sociales}, aunque se \enquote{muestran uniformemente proporciones más altas en los estratos inferiores}, por lo que es sobre ellos que \enquote{se ejercen las presiones más fuertes hacia la desviación} (Merton, 2002:223).

Trasladado el argumento hacia el mundo académico, es decir, hacia el espacio privilegiado de desenvolvimiento de los intelectuales y de la producción de ideas, puede también verse cómo la distancia entre los \enquote{fines culturales} considerados deseables y los medios legítimos establecidos, que en términos de Pierre Bourdieu serían \enquote{las reglas del campo}, lleva a que quienes ocupan posiciones subordinadas tiendan a la \enquote{innovación} en la producción de ideas si es que aspiran, eventualmente, a llegar a las posiciones dominantes. Sin la intención de forzar las categorías (puesto que las trayectorias de los agentes que se abordarán no se expresan en la dinámica de un campo propiamente dicho, o en todo caso, lo hacen en un campo en proceso de formación), aquí se estudiarán de manera comparada dos productores de bienes culturales intentando dar cuenta de las distintas apuestas que realizaron en relación con el mundo académico y, en particular, la sociología.

En principio, esto implica que los autores a abordar deben ser situados dentro de un grupo más amplio que, en términos muy generales, podría denominarse \enquote{clase media}, es decir, que se trata de agentes que mantenían una \enquote{dependencia material} respecto de las fracciones dominantes de la sociedad, aunque con la particularidad de poseer un abundante capital cultural \parencite{1576-BOURDIEU1999}. Esta situación de subordinación respecto de quienes ostentaban los lugares privilegiados al interior del espacio de socialización por excelencia de las élites económicas, culturales y políticas de Argentina, como lo fueron las facultades de Derecho y Ciencias Sociales de Buenos Aires y Córdoba \parencite{277-AGUERO2017,1542-ORTIZ2012}, implicó la adopción de ciertas estrategias \enquote{innovadoras} por parte de ambos agentes, aunque en disciplinas distintas.

Para decirlo en pocas palabras, y precisando el recorte del objeto a estudiar, los dos pensadores que se analizarán formaron parte de la misma clase social y sus trayectorias fueron extremadamente similares, pero la apuesta por la sociología fue contundente en un caso y no en el otro. En términos bourdesianos, la pregunta que guía estas reflexiones es por la \emph{ilussio} de uno y otro, es decir, por la creencia en el valor de aquello que estaba en juego en relación con la práctica de la sociología y, por lo tanto, si valía o no la pena involucrarse e invertir recursos en esta disciplina. En este sentido, el concepto de \emph{ilussio} es entendido como \enquote{interés} en un sentido fuerte, es decir que refiere al hecho de estar involucrado en un determinado \enquote{juego}. \enquote{Estar interesado quiere decir aceptar que lo que acontece en un juego social determinado tiene un sentido, que sus apuestas son importantes y dignas de ser emprendidas} \parencite[80]{1577-BOURDIEU1995}.

Las trayectorias que aquí se abordarán son las de dos profesores de sociología argentinos de la primera mitad del siglo~XX: Ricardo Levene y Raúl Orgaz. Ambos son representantes de una generación de \enquote{transición} en la historia de la sociología vernácula, ya que son el punto intermedio entre los pioneros de la disciplina que, por otra parte, fueron quienes los antecedieron en las cátedras que ellos luego ocuparían (Ernesto Quesada, en la Facultad de Filosofía y Letras de Buenos Aires, y Enrique Martínez Paz, en la Facultad de Derecho y Ciencias Sociales de Córdoba) y, al mismo tiempo, apadrinaron a quienes darían la disputa por el sentido que la sociología adquiriría en el momento en que se expandió por toda América Latina la llamada \enquote{ola cientificista} hacia mediados del siglo~XX, siendo los nombres propios de Gino Germani y Alfredo Poviña los representantes de tal conflicto a nivel local.

Lo sustancial del problema reside en la pregunta de por qué razón dos miembros de una misma generación, con orígenes y propiedades sociales similares y ocupando posiciones semejantes, pudieron llegar a tener visiones tan distintas sobre la misma disciplina que los acobijó y, en definitiva, que sus apuestas intelectuales hayan mostrado variaciones de magnitudes considerables. Puesto en otras palabras y perfilando el análisis histórico, ¿por qué Ricardo Levene y Raúl Orgaz, ambos intelectuales miembros de una clase media en ascenso, ambos formados en derecho y profesores universitarios de sociología durante largos años, tuvieron producciones tan disímiles en relación con esta ciencia social? La pregunta no solo es relevante por el hecho de comparar niveles de productividad de dos de los sociólogos más importantes del país de la primera mitad del siglo~XX, sino porque dar cuenta de esta situación resultará, a su vez, en una posible respuesta al por qué de la ausencia de trabajos que aborden la historia de la sociología porteña durante el período de entreguerras.

En general, este campo de estudios ha privilegiado el período posterior a 1957 en la UBA \parencite[por ejemplo][entre otros]{1450-VERON1974,1508-SIDICARO1993,1550-BLOIS2018} y, aunque existen reflexiones sobre la época del Centenario, las que están centradas en Buenos Aires finalizan con la Reforma de 1918 (Pereyra, 2000). Por cierto, las investigaciones en torno a la sociología cordobesa \parencite[por ejemplo][]{1447-CARACCIOLO2010,1528-REQUENA2010}(Grisendi, 2011) tampoco han pensado este caso en el contexto más amplio de la sociología argentina. Es decir que todavía no se han estudiado en forma articulada los desarrollos institucionales de las cátedras de Buenos Aires, La Plata, Córdoba y Santa Fe para el período 1920-1940.

Más bien, lo habitual en este campo de estudios es encontrarse con dos sesgos, uno espacial (enfocarse en la sociología porteña) y otro temporal (partir del año establecido por Germani como el inicio de la sociología en Argentina). Aunque este no es el lugar para desarrollar un estado de la cuestión exhaustivo sobre el conjunto de investigaciones de esta rama de estudios, debe decirse que en términos generales han acompañado el \enquote{mito fundacional} de Germani, por lo que el hecho de que no existan indagaciones que aborden en profundidad las décadas de 1920 y 1930 se explica por los sesgos mencionados, especialmente el espacial. Como consecuencia, la hipótesis que se sostiene es que la sociología cordobesa adquirió centralidad entre la Reforma de 1918 y la fundación del Instituto de Sociología de Buenos Aires en 1940, momento en el cual la capital argentina recuperó el lugar más destacado dentro de la sociología local.

Esta afirmación se sustenta en el cotejo de las producciones intelectuales en torno a esta disciplina por parte de los líderes de las sociologías cordobesa y porteña de la época. Estas son el indicador que da cuenta de que las discusiones en torno al objeto de la sociología, su significado y el quehacer de los sociólogos se centraron en la ciudad mediterránea por estos años. Lo que esto implica para el desarrollo del trabajo es, por un lado, la reconstrucción del proceso de institucionalización de la disciplina a nivel superior en Buenos Aires y Córdoba, en el cual se insertaron ambos intelectuales y, por otro lado, la de sus trayectorias sociales y sus apuestas intelectuales, las cuales se ponderarán con base en la cantidad de escritos que uno y otro autor consagraron a la materia. La comparación de las propiedades sociales de Levene y Orgaz permitirá observar una serie de equivalencias que, sin embargo, se mostrarán heterogéneas en relación con una variable: la producción sociológica.

Esta comparación, que deberá ser ampliada y profundizada en futuras investigaciones, posibilita el control de la afirmación realizada en relación con la capacidad de innovación que presentan los agentes que ocupan posiciones subordinadas al interior de un campo. Esta forma de abordaje se enmarca entonces entre los enfoques comparativos que privilegian la confrontación de \enquote{sistemas \emph{más semejantes}}, es decir, cuyo \enquote{tipo ideal} de investigación pasaría por encontrar entidades similares en todas las variables excepto en aquella que interesa explicar \parencite[29-49]{1509-SARTORI1994}.

\section{Las trayectorias de los líderes de la sociología argentina hasta la década de 1940}

La enseñanza de la sociología a nivel superior en Argentina data de finales del siglo~XIX. En particular, la primera cátedra universitaria dedicada a la materia se fundó en 1898 a raíz de una serie de presiones que ejercieron algunos intelectuales porteños de la época, entre los que resaltaron Juan Agustín García y Carlos Octavio Bunge. Como ejemplo de tal afirmación puede citarse una ponencia presentada en el Congreso Científico Latinoamericano, celebrado en Buenos Aires en abril de 1898, en la cual se exponía que el mentado congreso, \enquote{a moción nuestra ---decía Bunge---, ha declarado (\dots) por unanimidad que \enquote{es conveniente incluir en los programas de instrucción universitaria y secundaria el estudio de la sociología}} \parencite[80]{1445-BUNGE1898}.

La presión ejercida surtiría efecto y llevaría a la modificación del plan de estudios vigente en la recientemente creada Facultad de Filosofía y Letras de la Universidad de Buenos Aires (FFyL-UBA), comenzando a funcionar el primer curso de sociología el 1 de junio de 1898. Fueron sus profesores más destacados en la primera mitad del siglo~XX Antonio Dellepiane, Alfredo Colmo, Ernesto Quesada y Ricardo Levene, aunque seguramente los últimos dos tuvieron mayor relevancia dada la cantidad de años que permanecieron como titulares de la asignatura. En el caso de Quesada (titular entre 1905-1921), se trataba de un hombre perteneciente a los círculos de la élite porteña, miembro de una familia dedicada a los negocios bursátiles, aunque él en particular, y por motivos de la época, adquirió la mayor parte de su ingente fortuna al administrar la herencia de Eleonora Pacheco, su primera esposa, hija de Román Pacheco y Reynoso y nieta de Ángel Pacheco, célebre militar de las guerras de independencia \parencite{1535-BUCHBINDER2012}.

Levene, por su parte, había ingresado como suplente a la cátedra en 1911 y titularizó en 1922, luego del retiro de Quesada. A su vez, fundó el Instituto de Sociología en la FFyL-UBA en 1940 y se hizo cargo de la primera cátedra de sociología de la Facultad de Humanidades y Ciencias de la Educación de la Universidad Nacional de La Plata (FaHCE-UNLP) desde su fundación en 1924, permaneciendo en todos estos cargos hasta 1947, con lo cual su participación en esta disciplina abarca un período más que considerable de su trayectoria. Desafortunadamente su biografía es prácticamente desconocida en lo que refiere a su infancia y juventud, por lo que la reconstrucción de su posición social de origen, ligada inevitablemente a la de sus padres, resulta dificultosa.

No obstante, algunos datos revelarían que, al igual que otros profesores de la cátedra como Dellepiane y Colmo, Levene era hijo de inmigrantes de sectores medios que vieron en la educación superior un canal de movilidad social ascendente. Nacido en Buenos Aires en 1885, provenía, según \textcite{1628-GALVEZ2002}, de una familia de origen judío siendo su padre sastre de profesión. Levene fue un estudiante excepcional, ya que finalizó sus estudios en el Colegio Nacional de Buenos Aires en 1900 con tan solo 15 años \parencite{1530-RAJMANOVICH2016} y defendió su tesis de doctorado en Jurisprudencia en la Facultad de Derecho y Ciencias Sociales de la Universidad de Buenos Aires (FDCS-UBA), titulada \emph{Leyes Sociales}, en 1906, cuando apenas había cumplido 21. Durante sus años de estudiante tuvo una incursión en el periodismo entre 1902-1904, escribiendo artículos sobre temas candentes de la época como la inmigración, los aranceles universitarios, las huelgas estudiantiles, etcétera, al tiempo que militó brevemente en política apoyando la candidatura presidencial de Carlos Pellegrini \parencite{1526-RODRIGUEZ2001}.

Luego de su graduación, Levene tuvo un ascenso importantísimo a partir del acceso a diversos cargos de jerarquía a nivel docente e institucional consolidando su posición dominante hacia la década de 1930. En cuanto al primer rubro, además de los mencionados cargos de Sociología en la UBA y la UNLP, ejerció desde 1906 hasta finales de los años veinte como profesor del Colegio Nacional y la Escuela Normal n.° 40 \enquote{Estanislao S. Zeballos}. En 1912 fue designado profesor suplente de historia del derecho en la FDCS-UBA, materia a cargo de Carlos Octavio Bunge, y en 1913 accedió a la titularidad de la cátedra de historia argentina de la Sección de Filosofía, Historia y Letras de la Facultad de Ciencias Jurídicas y Sociales de la Universidad Nacional de La Plata (FCJS-UNLP) \parencite{1627-FINOCCHIO2001}. En la UNLP es donde alcanzó las posiciones más importantes ya que fue decano de la FaHCE-UNLP en dos ocasiones (1920-1923 y 1926-1930), institución en la que fundó la revista \emph{Humanidades} y la Biblioteca de Humanidades en 1921 y el Instituto Bibliográfico en 1926 (Heras, 1959). Ese mismo año creó y dirigió el Archivo Histórico de la Provincia de Buenos Aires y, al finalizar su segundo decanato, fue elegido presidente de la UNLP por dos períodos consecutivos, entre 1930-1935.

Por fuera del ámbito de la enseñanza fue asesor letrado de la Dirección General de Vías de Comunicación y jefe del Departamento de Estadística de la Policía de la Provincia de Buenos Aires en 1915. En 1933 presidió la Comisión Revisora de Textos para la Enseñanza de la Historia y la Geografía y en 1938 promovió la creación de la Comisión Nacional de Museos, Monumentos y Lugares Históricos, de la cual fue presidente hasta 1946. Dos años antes había sido designado vicepresidente de la Comisión del IV Centenario de la Fundación de Buenos Aires junto a Emilio Ravignani, su presidente (Heras, 1959). Y es precisamente con Ravignani con quien compartió una de sus apuestas intelectuales más importantes, es decir, aquella ligada al oficio de historiador, siendo especialmente relevante la dedicación a la historia del derecho desde la cátedra que heredaría luego de la muerte de Bunge en 1918.

Tanto Ravignani como Levene resultaron ser, junto a Diego Molinari, Rómulo Carbia y Luis María Torres, protagonistas principales de la renovación historiográfica de la década de 1920 que Juan Agustín García identificó como la Nueva Escuela Histórica \parencite{282-ZARRILLI1998}. Precisamente, la Junta de Historia y Numismática Americana (a partir de 1938, Academia Nacional de la Historia), de la cual Levene fue presidente entre 1927-1931 y desde 1935 hasta su fallecimiento en 1959, fue un lugar clave de su trayectoria. En este contexto, un hecho que resulta llamativo es que, a pesar de haber adquirido la titularidad de dos cátedras de sociología a comienzos de la década de 1920, su producción de reseñas, artículos, libros, etcétera, en relación con esta disciplina fue escasa.

En concreto, debería resaltarse que la producción sociológica de Levene, por lo menos hasta 1940, cuando funda y dirige el Instituto de Sociología, lo cual le hace tener alguna dedicación un poco más importante hacia la ciencia social, consiste fundamentalmente en siete textos: su tesis doctoral, \emph{Leyes Sociales} (1906); el libro \emph{Leyes Sociológicas} (1907), cuyo contenido es similar aunque no idéntico al de la tesis; el texto \emph{Los orígenes de la democracia argentina} (1910), escrito para acceder al cargo de la cátedra de sociología de la FFyL-UBA; la presentación que en 1928 hiciera de Célestin Bouglé en ocasión de la conferencia que brindara en la FaHCE-UNLP; el prólogo de 1934 que redactara al libro de Alfred Vierkandt, \emph{Filosofía de la Sociedad y de la Historia}, impreso por la UNLP; el prólogo al manual de Justo Prieto, profesor de sociología en la Facultad de Ciencias Económicas de la UBA, titulado \emph{Síntesis Sociológica} (1937); y finalmente, el artículo \enquote{Sarmiento, sociólogo de la realidad americana y argentina}, publicado por \emph{Humanidades} (1938). Más allá de estos escritos, podrían mencionarse algunos de sus discursos \parencite{1621-ESCUDERO2010} y, con posterioridad a la fundación del Instituto de Sociología, aparecen algunos informes y, fundamentalmente, el libro \emph{Historia de las ideas sociales argentinas} (1947).

Ahora bien, si estas intervenciones se compararan con la prolífica obra sociológica de Raúl Orgaz, no puede dejar de llamar la atención las diferencias en las apuestas que uno y otro autor realizaron en relación con la disciplina. En números precisos, a lo largo de su trayectoria académica Orgaz redactó un total de 146 artículos, siendo la mayoría publicados por el diario \emph{La Prensa} (46), la \emph{Revista Derecho, Historia y Letras} (18), la \emph{Revista de Filosofía} de José Ingenieros y Aníbal Ponce (16) y la \emph{Revista de la Universidad de Córdoba} (11). El resto de sus trabajos se reparten entre la \emph{Revista Argentina de Ciencias Políticas} (6), la \emph{Revista Jurídica de Córdoba} (3), el \emph{Boletín de la Facultad de Derecho y Ciencias Sociales} de la UNC (3), los \emph{Anales de la Academia de Derecho y Ciencias Sociales de Córdoba} (3), la \emph{Revue Internationale de Sociologie} (3) y el diario \emph{La Voz del Interior} (3), más alguna publicación esporádica en otros foros como la revista \emph{Humanidades} de La Plata, el \emph{Boletín del Instituto de Sociología} de Buenos Aires o el Instituto de Filosofía de la UNC.\footnote{La reconstrucción completa de su bibliografía se encuentra en \textcite{1544-ORGAZ1960}.}

Es importante resaltar que la labor sociológica e historiográfica de Orgaz se basó principalmente en la publicación de artículos, ya que prácticamente la mitad de sus libros consistieron en compilaciones. Así, de sus 15 obras publicadas, siete pueden considerarse originales\footnote{\emph{Condición jurídica internacional de las sociedades anónimas} (1913); \emph{La sinergia social argentina} (1924); \emph{Introducción a la sociología} (1933); \emph{Echeverría y el saintsimonismo} (1934); \emph{Alberdi y el historicismo} (1937); \emph{Vicente Fidel López y la filosofía de la historia} (1938); \emph{Sarmiento y el naturalismo histórico} (1940).}, seis fueron compilaciones,\footnote{\emph{Estudios de sociología} (1915); \emph{Cuestiones y notas de historia} (1922); \emph{Páginas de crítica y de historia} (1927); \emph{Ideas y doctrinas de nuestro tiempo} (1929); \emph{La ciencia social contemporánea} (1932); \emph{Ensayo sobre las revoluciones} (1945).} y finalmente aparece una \emph{mélange} entre textos publicados e inéditos que sería su \emph{opera magna}, \emph{Sociología} (1942), reeditada en 1946 con ligeras modificaciones. De todos ellos, diez fueron publicados en Córdoba (ocho por Imprenta Argentina, una editorial del reformismo universitario) y cinco en Buenos Aires, estos últimos entre 1924 y 1933.\footnote{Con una reedición de las conferencias del Colegio Libre de Estudios Superiores, \emph{Introducción a la sociología} (1933), en 1937.} Ahora bien, ¿cuál fue el contexto de institucionalización de la sociología en la Facultad de Derecho y Ciencias Sociales de la Universidad Nacional de Córdoba (FDCS-UNC) y de qué manera se vinculó Orgaz con ella?

En la universidad de la ciudad mediterránea, la actualización del sistema de enseñanza fue motivo permanente de conflicto entre sectores clericales y liberales durante el siglo~XIX, siendo que ya desde la propuesta del deán Funes se habían hecho palpables \enquote{las dificultades para romper con el marco escolástico y con la impronta religiosa que signaba a la casa de estudios cordobesa} \parencite[21]{1536-BUCHBINDER2010}. Para fines de la centuria, puede citarse como ejemplo la conocida anécdota de las primeras tesis de la FDCS-UNC, pertenecientes a José del Viso (1883) y Ramón J. Cárcano (1884), las cuales fueron rechazadas porque no se atenían al dogma católico. Esto evidencia que la estrecha vinculación con la Iglesia y el modelo de enseñanza adoptado por la UNC, es decir, aquel surgido en Europa occidental en tiempos medievales, aún mantenía vigencia hacia fines del siglo~XIX y resultó una herencia que condicionó en gran medida los intentos de modernización de la universidad cordobesa.

En esta época, y a pesar de que el Congreso había aprobado en 1878 la ley por la cual se creaba la Facultad de Humanidades de la UNC sobre la base del Colegio Monserrat, en la práctica nunca se implementó. Posteriormente, con la sanción de la \enquote{ley Avellaneda} en 1885, la UNC pretendió adoptar una orientación profesionalista en tanto la ley consideraba que la articulación de saberes debía \enquote{responder al propósito de las profesiones más requeridas por la sociedad} \parencite[26]{1561-CHAVES2013}. Entonces, aunque la Universidad de Córdoba se mostraba con intención profesionalista en la vuelta de siglo tenía como cuenta pendiente \enquote{el desarrollo de los estudios de las humanidades} \parencite[28]{1561-CHAVES2013}. Hacia 1887 la FDCS-UNC

\begin{quote}
\enquote{(\dots) manifestando el propósito de \enquote{uniformar} la enseñanza con la de la Facultad de Derecho de Buenos Aires (\dots) reformó nuevamente el plan de estudios. Las novedades incorporadas ---Introducción al Derecho y Filosofía del Derecho--- completaban por fin la supuesta afinidad con el esquema de materias, títulos profesionales y formas de graduación que en Buenos Aires habían terminado de definirse en la primera mitad de la década de 1870} \parencite[49]{1560-CHAVES2013}.
\end{quote}

A su modo, en Córdoba también se impulsó el desarrollo de una \enquote{cultura científica} \parencite{1506-TERAN2008}, la cual alimentaba el ascenso de los \enquote{hombres de ciencia} frente a los espacios ya consagrados de la medicina y el derecho. En este contexto, si bien la sociología no dejaría de ser en todo el siglo~XX un conocimiento auxiliar en la formación de abogados, en el marco de la modernización de la educación superior el estímulo al desarrollo de la disciplina estuvo estrechamente vinculado a la firme convicción de renovar la institución universitaria (Grisendi, 2010).

Es en este marco de disputas entre católicos y liberales en los claustros universitarios que la fundación de la cátedra, impulsada por los últimos, llevó a que Isidoro Ruiz Moreno y Urquiza, familiar del general Justo José de Urquiza por parte de su madre, asumiera el cargo de profesor titular de sociología entre 1907-1908, la cual abandonó luego de ser electo diputado provincial por el Partido Autonomista Nacional. El breve paso de Ruiz Moreno por la cátedra dio lugar a que su suplente, Enrique Martínez Paz, lo reemplazara en la titularidad durante el período 1909-1918. Martínez Paz, al igual que Ruiz Moreno, tenía un origen patricio, ya que era el único hijo de un acaudalado matrimonio conformado por el dueño de la empresa de tranvías de la ciudad, Pedro Martínez Caballero, y Constancia Paz y Peña, quien pertenecía a la tradicional familia Paz. La condición social que ostentaban ambos agentes los distingue claramente de quien los sucedería en el cargo.

Así, la cátedra de sociología contó entre sus primeros estudiantes a un joven entusiasta de la ciencia social que, como ya se ha visto, llegaría a desarrollar una vasta producción ligada a la materia. Nacido en Santiago del Estero en 1888, Raúl fue el hijo mayor de diez hermanos producto del matrimonio de Eleodoro Orgaz, un agrimensor que trabajó en una oficina precursora de lo que después sería Catastro en Córdoba, y Mercedes Ahumada, hija de una familia tradicional de la Villa de San Pedro en Traslasierra (Ighina, 1996). Desde muy temprana edad, Raúl se vinculó con personajes como Arturo Capdevila y Octavio Pinto, quienes, con su propio hermano Arturo, formaron parte del círculo que rodeó a Martínez Paz.

Luego de concluir sus estudios en el Colegio Monserrat, su vocación sociológica orientó de forma acentuada sus lecturas durante sus años de estudiante universitario. Sus primeras publicaciones muestran un predominio en el interés por discutir las proposiciones de la ciencia social. En general, estas primeras aproximaciones se plasmaron en artículos de la \emph{Revista Derecho, Historia y Letras}, la cual era dirigida por Estanislao Zeballos, profesor de Derecho Internacional Privado en la Universidad de Buenos Aires. No por casualidad, Zeballos sería padrino de su tesis, la cual defendió en 1913. Una vez concluidos sus estudios en Derecho, la Universidad Nacional de Córdoba le concedió una beca para estudiar en Europa (lo cual da cuenta de que no poseía recursos propios para realizar este tipo de viajes, a diferencia de otros sociólogos de la época como, por ejemplo, los porteños Ernesto Quesada o Leopoldo Maupas). Desafortunadamente, como comenta Henoch Aguiar,

\begin{quote}
(\dots) lo sorprendió la guerra del 14, cuyo estallido lo obligó a regresar al país antes de haber cumplido su programa de estudio que se había trazado y, cuando solo había visto superficialmente a Londres, La Haya, Bruselas y comenzado a asistir en \enquote{La Sorbona}, a alguno de los cursos dictados por eximios profesores como Durkheim, Capitant, Planiol (Aguiar, en \cite[10]{1544-ORGAZ1960}).
\end{quote}

Inmediatamente luego de su regreso al país ingresó a la cátedra de sociología de la FDCS-UNC en 1915,\footnote{Ese mismo año Orgaz ingresó al Poder Judicial como Secretario del Juzgado Federal de Córdoba, pero permaneció allí tan solo hasta el año siguiente. Su reinserción tuvo lugar recién en 1943, como vocal del Tribunal Superior de Justicia de la Provincia hasta 1947 (en 1944 presidió el cuerpo) \parencite{1528-REQUENA2010}.} en carácter de suplente, para luego de la Reforma convertirse en titular hasta 1946, lo que no haría sino profundizar su especialización en la ciencia social de la modernidad. Al margen de su participación en dicha cátedra, dictó clases de historia argentina y de castellano en el colegio Monserrat entre 1915 y 1946\footnote{En uno de los prólogos a los libros publicados luego de la muerte de Orgaz, Aguiar comentaba que Capdevila recibió una carta de Orgaz luego de su cesantía en la cual escribía: \enquote{he perdido mi apacible club (llamo así a la tertulia intermitente de los recreos en el Colegio Nacional)} (citado por Aguiar en \cite[20]{1545-ORGAZ1950}).} al tiempo que formó parte de comisiones evaluadoras de idiomas, por ejemplo, de italiano. También, aunque de forma más modesta, tuvo participación en algunos cargos a nivel institucional. En este rubro se destacan su ejercicio como decano de la Facultad de Derecho (1942-1943), vicerrector y rector interino de la UNC (1943, 1945) y presidente del Tribunal Superior de Justicia de Córdoba (1944), aunque como se ve ejerció esos cargos por breve tiempo y hacia el final de su trayectoria.

Entonces, podría afirmarse que Orgaz perteneció, para decirlo con \textcite{1576-BOURDIEU1999}, a la fracción dominada de la clase dominante, en tanto se ganó la vida como asalariado. No formó parte de una familia adinerada, ya que su padre fue un empleado estatal y su madre ama de casa. Sus ingresos provenían de su ocupación como abogado en el estudio jurídico que pertenecía a Henoch Aguiar, donde trabajó durante un cuarto de siglo, y al cual había ingresado luego de que Arturo Capdevila dejara la ciudad en 1922. Combinado con el dictado de clases a nivel secundario y universitario, estas fueron sus actividades principales. Por lo tanto, puede apreciarse que su condición social podría mostrar un \enquote{avance} en relación con la capacidad de \enquote{innovación} al interior de la disciplina, comparado con sus antecesores Ruiz Moreno y Martínez Paz, siendo análoga a la condición que en Buenos Aires mostraba Levene en relación con Quesada.

Pero antes de avanzar, debería mencionarse otra de las apuestas intelectuales de Orgaz que, aunque esté por fuera del objeto de estudio aquí propuesto, fue en el campo de la historia, rubro en el cual también entabló relaciones con Levene. En el contexto de la política expansiva desarrollada por la Junta de Historia y Numismática Americana en la década de 1920, se fundaron filiales en el interior del país, siendo Martínez Paz y Orgaz los iniciadores de la de Córdoba. Posteriormente, Orgaz participó de la fundación del Instituto de Estudios Americanistas en 1936 (además del de Filosofía), durante el rectorado de Sofanor Novillo Corvalán en la UNC, y fue, asimismo, delegado en el Congreso Científico Panamericano de Lima (1925), en el II Congreso Internacional de Historia Americana de Buenos Aires (1937) y en viajes a los Estados Unidos a comienzos de los 40 \parencite{1528-REQUENA2010}.

Finalmente, en cuanto a su ubicación en el espectro ideológico, resulta una tarea más difícil de lo que, \emph{a priori}, podría pensarse. Por fuera de sus simpatías por el reformismo universitario, lo cual lo ligaría al liberalismo reformista de principios de siglo, Aguiar manifestó que Orgaz era partidario del radicalismo (Aguiar, en \cite{1544-ORGAZ1960}). No obstante, este abogado cordobés, cuadro de la Unión Cívica Radical, obvia mencionar que Orgaz ingresó al Partido Socialista (PS), al cual pertenecía hacía varios años su hermano Arturo, luego del golpe de 1930 \parencite{1507-TCACH2012}. De hecho, el PS presentaría la fórmula Nicolás Repetto-Arturo Orgaz a la elección presidencial de 1937. Costaría, por tanto, encontrar en Raúl un adherente al sabattinismo, aunque sus cargos de mayor jerarquía hayan sido durante un gobierno de dicho signo.\footnote{Esto se explica porque los reformistas, aunque no se llevaron nada bien con Sabattini por su declaración como neutral ante la Gran Guerra y su aceptación de una condecoración de Mussolini, tuvieron buena relación con su sucesor Santiago del Castillo, gobernador de Córdoba entre 1940 y 1943 y, sobre todo, con Arturo Illia, su vicegobernador, quien encabezó el cortejo fúnebre de Deodoro Roca en junio de 1942 \parencite{1624-FERRERO1984}.} Por otro lado, si se tiene en cuenta que la mayor parte de sus artículos fueron publicados en un diario liberal-conservador como \emph{La Prensa,}\footnote{Una hipótesis plausible respecto de su inserción en este diario supone que Orgaz usufructuó el contacto con Estanislao Zeballos, que tenía relación con José C. Paz, dueño de la empresa periodística (Grisendi, 2011).} se evidencia una pluralidad que estaría a la par del eclecticismo que presenta su obra sociológica.

Evidentemente no fue un hombre de \enquote{izquierdas} tal como lo caracterizó \textcite{1624-FERRERO1984} aunque, si se tiene en cuenta el clivaje dominante en el ambiente universitario cordobés en el cual estaba inserto, tampoco podría considerárselo un partidario del conservadurismo clerical. Podría entonces concluirse que en Orgaz primó su pertenencia a la corporación judicial cordobesa a través de la FDCS-UNC, con acercamientos eventuales a sectores más progresistas o más conservadores. Más allá de los lazos establecidos por la coyuntura, en sus trabajos se adivina una mirada afín al liberalismo. De lo contrario no se entendería su alejamiento de la UNC en 1946 \enquote{por razones de carácter político} \parencite[25]{1558-CHAMORROGRECA2007}. De todas maneras, en Orgaz primó la dimensión intelectual con una verdadera vocación sociológica formando parte de su labor académica de principio a fin.

\section{La apuesta por la sociología}

Este \emph{racconto} de las trayectorias de Levene y Orgaz resulta axial para la explicación del motivo por el cual sus apuestas en relación con la sociología resultaron tan disímiles. Si se presta atención a los elementos recogidos en el apartado anterior podría decirse que, en términos generales, los dos tuvieron inserciones en las mismas tres esferas de conocimiento: sociología, historia y derecho. La cuestión a explicar es por qué mientras para Levene resultaron ser más importantes la historia y el derecho (y a la postre, la historia del derecho) que la sociología, por el contrario, para Orgaz la sociología fue su principal apuesta, quedando relegada la historia a un segundo plano y la práctica del derecho solo como un medio de subsistencia en el estudio jurídico en el cual trabajó en relación de dependencia. Como ya se ha dicho más arriba, en términos bourdesianos, el elemento a explicar es la \emph{illusio} adquirida por Orgaz en relación a la sociología.

Entonces, si se quisieran sintetizar las relaciones que ambos autores tuvieron solo en relación con la sociología deberían tomarse los datos biográficos esenciales de uno y otro y cómo se vincularon a esta ciencia social para poder establecer puntos de comparación. De allí podría observarse lo siguiente:

Cuadro 1: Síntesis biográfica comparativa Ricardo Levene-Raúl Orgaz

\begin{longtable}[]{@{}
  >{\raggedright\arraybackslash}p{(\columnwidth - 4\tabcolsep) * \real{0.3333}}
  >{\raggedright\arraybackslash}p{(\columnwidth - 4\tabcolsep) * \real{0.3333}}
  >{\raggedright\arraybackslash}p{(\columnwidth - 4\tabcolsep) * \real{0.3333}}@{}}
\toprule\noalign{}
\begin{minipage}[b]{\linewidth}\raggedright
\emph{Datos}
\end{minipage} & \begin{minipage}[b]{\linewidth}\raggedright
\emph{Ricardo Levene}
\end{minipage} & \begin{minipage}[b]{\linewidth}\raggedright
\emph{Raúl Orgaz}
\end{minipage} \\
\midrule\noalign{}
\endhead
\bottomrule\noalign{}
\endlastfoot
\emph{Años de vida} & 1885-1959 & 1888-1948 \\
\emph{Origen social} & Hijo de inmigrantes & Hijo de migrantes internos \\
\emph{Estudios realizados} & Derecho en Buenos Aires & Derecho en Córdoba \\
\emph{Fecha de graduación (doctorado)} & 1906 & 1913 \\
\emph{Ingreso a la cátedra de \enquote{Sociología} (suplente)} & 1911 (FFyL-UBA) & 1915 (FDCS-UNC) \\
\emph{Titularización de la cátedra de \enquote{Sociología}} & 1922 (FFyL-UBA) // 1924 (FaHCE-UNLP) & 1918 (FDCS-UNC) \\
\emph{Producción sociológica} & 4 Libros y 7 Artículos\footnote{Aquí se toma como libro su tesis doctoral y como artículos tanto la presentación de Bouglé como los prólogos a los libros de Vierkandt y Prieto. A su vez, se tiene en cuenta un artículos posterior a 1940: \enquote{José Victorino Lastarría, sociólogo} (1942), publicado por el \emph{Boletín del Instituto de Sociología} de Buenos Aires.} & 15 Libros y 146 Artículos\footnote{En el caso de Orgaz se utiliza un criterio laxo para designar el conjunto de sus obras como \enquote{sociológicas}, ya que resulta complejo circunscribir algunos trabajos exclusivamente al campo de la historia o de la sociología, tal como pueden ser sus textos sobre Echeverría, Alberdi y Sarmiento. A su vez, en términos generales aun sus escritos con mayor cantidad de contenido histórico mantuvieron estrechas relaciones con la sociología, como por ejemplo \emph{La sinergía social argentina} (1924).} \\
\emph{Alejamiento de la cátedra de \enquote{Sociología}} & 1947 & 1946 \\
\end{longtable}

Fuente: elaboración propia.

El cuadro muestra una serie de similitudes que podrían resumirse en lo siguiente: ambos autores pertenecieron a una misma generación; los dos fueron hijos de inmigrantes (internos uno, extranjeros el otro); ambos venían de familias de clase media, las cuales tuvieron la posibilidad de educar a sus hijos en la universidad; los dos se formaron en las facultades de Derecho de sus ciudades respectivas, pero tuvieron una temprana inclinación por la sociología; la fecha de ingreso a las cátedras de \enquote{Sociología} como profesores suplentes, el año de su titularización y la fecha de su alejamiento de las cátedras son prácticamente coincidentes. Por otra parte, se desprende también de sus trayectorias que la consolidación de sus patrimonios económicos tuvo lugar, probablemente, hacia las décadas de 1930 en el caso de Levene (momento en que deja de dictar clases en la escuela secundaria y es elegido presidente de la UNLP) y 1940 en el de Orgaz (cuando llega al Rectorado de la UNC y a presidir el Tribunal de Justicia de Córdoba). Sin embargo, si se tiene en cuenta que Levene fue profesor titular de dos cátedras de sociología, mientras que Orgaz tan solo de una, ¿por qué este último invirtió tanto esfuerzo para actualizarse y producir innovaciones (dentro de los límites de su época), mientras el primero produjo tan poco?

La única respuesta que puede desentrañar el problema es de carácter sociológico y se corresponde con una nueva similitud entre Levene y Orgaz, la cual podría describirse a partir del concepto de \enquote{grupo de referencia} desarrollado por Merton. Este sociólogo norteamericano entendía que la \enquote{socialización anticipadora} resultaba ser una función positiva de integración de los individuos que se orientaban hacia grupos a los cuales no pertenecían. Es decir que solía ocurrir en distintas instituciones que se tomara como sistema de referencia el conjunto de normas y valores propios de un grupo del cual no se formaba parte y, por lo tanto, se obrara de forma positiva hacia ellos. En palabras del autor, \enquote{para el individuo que adopta los valores de un grupo al cual aspira, pero al cual no pertenece, esta orientación [a las normas del grupo que no se pertenece] puede servir a la doble función de ayudar a su elevación dentro de ese grupo y de facilitar su adaptación una vez que ha llegado a formar parte de él} (Merton, 2002:345).

De esta manera, tanto Levene como Orgaz veían a la élite argentina, con la cual compartieron el espacio de socialización de las facultades de Derecho de Buenos Aires y Córdoba \parencite{277-AGUERO2017,1542-ORTIZ2012}, como el grupo social de referencia del cual aspiraban a formar parte, por lo que sus apuestas intelectuales más importantes debieron estar inevitablemente ligadas a las actividades que realizaban en vínculo con él. De tal forma que la única diferencia en relación con la sociología entre Levene y Orgaz consiste en el espacio social en el cual impartieron la materia, ya que el reclutamiento del público que asistía a cada una de estas instituciones resultaba bien distinto. Así, mientras Levene dictaba la materia en las facultades de Filosofía y Letras de Buenos Aires y Humanidades de La Plata, espacios que podrían pensarse mayormente compuestos por miembros de la clase media \parencite{1537-BUCHBINDER1997,1627-FINOCCHIO2001}, Orgaz estaba a cargo de la misma asignatura en la Facultad de Derecho, el cual constituía \enquote{el espacio del poder} de la sociedad cordobesa \parencite[39]{1447-CARACCIOLO2010}.

Esto implica una diferencia cualitativa de magnitudes considerables, la cual solo se manifiesta cuando se comparan las producciones de uno y otro autor en relación con la disciplina. El hecho de que la inserción de Orgaz en la sociología se diera en la FDCS-UNC es la causa de que su principal apuesta intelectual estuviera ligada a dicha disciplina y, como corolario, la abundante producción, actualización e innovación que dejó en relación con esta materia. Esto se debió a que la sociología fue para Orgaz el principal canal de diálogo con los sectores dominantes de la sociedad cordobesa ya que, por lo visto anteriormente, no compartió otros espacios de socialización con la élite de la ciudad por fuera de la cátedra universitaria hasta su ingreso al Tribunal Superior de Justicia de Córdoba en 1943. Por ello, convertirse en la referencia de la sociología para los miembros de la élite de la ciudad mediterránea se transformó en el principal interés intelectual del sociólogo santiagueño.

Si su trayectoria no mostró avances hacia la investigación empírica, la cual comenzó en el ámbito académico argentino recién hacia la década de 1940, luego de la fundación del Instituto de Sociología de la UBA y los inicios de la labor en su interior de Gino Germani (González Bollo, 1999), es producto de las dificultades de encarar un proyecto colectivo exitoso en el ámbito en el que se desenvolvía, es decir, dentro de la rigidez que mostraban las estructuras de la FDCS-UNC. De hecho, hacia comienzos de la década de 1940 Orgaz había comenzado a ver con buenos ojos la articulación de la teoría con la empiria, tal como lo expusiera en el prólogo al libro \emph{Sociología y Filosofía Social} de Renato Treves \parencite{1546-ORGAZ1941}. Orgaz se sitúa entonces en una época de transición hacia la constitución de un nuevo mercado de bienes culturales del cual no pudo obtener frutos, ya que falleció en 1948, cuando la sociología científica estaba aún en vías de formación.

Por el contrario, el vínculo de Levene con la élite porteña que se formaba en la FDCS-UBA tuvo lugar a partir de las cátedras de introducción al derecho y de historia del derecho. A pesar de que sería un capítulo aparte y que, por otro lado, correspondería a los historiadores del derecho indagar qué tanto innovó en los estudios de este campo, lo cierto es que basta ver que el Instituto de Historia del Derecho de la FDCS-UBA lleva el nombre de este historiador porteño para tomar conciencia de que Levene invirtió una mayor cantidad de recursos en este rubro antes que en la sociología. Un indicador interesante que confirma esta aseveración es que, cuando en 1947 se sancionó la ley universitaria 13.031, la cual indicaba en su artículo 59 que ningún profesor titular podría desempeñarse al mismo tiempo en más de una cátedra, Levene optó por renunciar a las materias de sociología y se quedó con historia del derecho en Buenos Aires. Sin desconocer la abundancia de notas manuscritas y mecanografiadas existentes en su archivo personal\footnote{La referencia es a la Biblioteca, Museo y Archivo Ricardo Levene, la cual funciona en la que fuera la casa de Levene en el barrio porteño de Recoleta y que actualmente forma parte de la Biblioteca Nacional de Maestros.} vinculadas a libros de sociólogos de su época, lo cierto es que solo fueron anotaciones preparadas para el dictado de clases, antes que un proyecto original ligado al estudio de la ciencia social.

\section{Conclusiones}

A lo largo del presente trabajo se han movilizado algunos elementos nodales de la sociología del conocimiento, para aplicarlos al análisis comparado de las trayectorias de dos destacados exponentes de la sociología argentina. A partir de su abordaje, se ha podido explicar el hecho de que su condición subordinada al interior de la clase dominante los impulsara a realizar innovaciones con el fin de alcanzar posiciones elevadas en el campo intelectual. Se ha visto que, si bien mostraban similitudes notables en muchas de sus características, la diferencia sustancial entre uno y otro radicó en el espacio social en el cual dictaban sociología, de tal manera que esta disciplina resultaba ser mucho más atractiva y, por lo tanto, pasible de ser objeto de inversiones más abundantes cuando formaba parte de la carrera de abogacía. Esto se debía a que las élites locales se abocaban al estudio de esta profesión, con lo cual resultaba ser lo que los vinculaba con la clase dominante, a diferencia de lo que ocurría cuando la materia formaba parte de otras carreras como historia, filosofía o letras. De aquí que la inversión intelectual de los agentes analizados estuviera estrechamente ligada al área de desempeño en las facultades de Derecho, antes que en las de Filosofía y Letras y, en definitiva, esto es lo que terminó por explicar la apuesta de Orgaz por la sociología y la escasa producción de Levene en este campo de estudios.

Si a la labor que Orgaz desarrolló desde su titularización en 1918 se suman las producciones de Poviña y Francisco W. Torres, adjuntos de la cátedra desde 1930, se evidencia una masa de trabajos sociológicos que no tiene comparación ni con Buenos Aires ni con Santa Fe, es decir, donde se encontraban las otras cátedras de sociología que funcionaban en Argentina por la época. Por lo tanto, el desplazamiento del eje de la sociología argentina y la centralidad cordobesa adquirida durante el período de entreguerras resulta una evidencia que solo puede ser soslayada si se adopta la perspectiva germaniana que entiende que los sociólogos previos a 1957 fueron practicantes de una filosofía social de orientación metodológica romántico-idealista, antes que \enquote{la} sociología, propiamente dicha.

A partir de estos elementos, y como hipótesis para futuras indagaciones, podría pensarse que si el nivel de innovación de los agentes al interior de la sociología depende en gran medida de qué tanto se encuentren, por sus propiedades sociales, en una condición subordinada respecto de quienes ocupan posiciones dominantes en el campo, por lo menos una parte del resultado de la disputa de mediados de siglo~XX por el liderazgo de la disciplina en Argentina puede explicarse por este motivo. En concreto, si se observa quién sucedió a Orgaz en el cargo, es decir, Alfredo Poviña, puede apreciarse que se trataba de un agente que, por sus características fundamentales, resultaba muy parecido a los sociólogos del Centenario (Quesada o Martínez Paz), tanto por su pertenencia a una familia tradicional de Tucumán, como por el hecho de que su menester principal fuera en el Poder Judicial, al frente de un juzgado.

De aquí que pueda deducirse que sus propiedades sociales resultaron ser uno de los motivos que lo alejaron de la posibilidad de innovar en la materia, ya que si se observa su producción podría apreciarse que simplemente se inclinó hacia la adopción del programa de sociología desarrollado originalmente por Orgaz, quien fuera su maestro en las décadas de 1920 y 1930 en la FDCS-UNC. Poviña fue así representante de la tradición sociológica denominada, de manera peyorativa, como \enquote{sociología de cátedra}, la cual puede encontrarse expresada en su manual \emph{Cursos de sociología}, originalmente publicado en 1945 pero repetido en lo sustancial, aunque con actualizaciones, hasta la década de 1980.

Por el contrario, Gino Germani, al margen de que pudo insertarse en el proyecto colectivo llevado a la práctica por Levene, desarrollado en un ambiente con mayor plasticidad que el que Orgaz podía tener en Córdoba, fue un agente con características que, por un lado, lo volvían aún más marginal que Levene y Orgaz en sus primeros años al frente de las cátedras de sociología y, por otro lado, poseía una formación en economía que traía de su Italia natal, por lo que representó un verdadero cambio en lo que hacía a la definición de la sociología, los alcances y los límites de la disciplina, la metodología para abordar la realidad social, etcétera. En definitiva, se trataba de un agente con condiciones que lo habilitaban a generar una modificación sustantiva de las \enquote{reglas del juego} con las que se practicaba la sociología, por lo menos hasta la década de 1950.

De esta manera, en Germani podrá observarse la introducción de un nuevo estilo de hacer sociología, que se vinculaba fuertemente al trabajo empírico y la movilización de un instrumental estadístico que no se encuentra en ninguno de sus predecesores. De aquí la modificación de lo que hasta ese momento significaba la sociología, ser sociólogo y la práctica profesional de la disciplina. Por lo tanto, un cotejo entre Germani y Poviña daría cuenta del enriquecimiento que significó para la sociología porteña los aportes del sociólogo italiano y del empobrecimiento de la sociología cordobesa que expresó la obra del sociólogo tucumano, aunque esta comparación de trayectorias ya forma parte de otro trabajo.

%Capítulo 2
\chapter{Trayectorias olvidadas, tradiciones silenciadas. Guillermo Terrera y la \enquote{sociología de cátedra}}

\footnote{\textsuperscript{*} Publicado en \emph{Revista De La Red Intercátedras De Historia De América Latina Contemporánea}, n.º 18, págs. 50--68. \url{https://revistas.unc.edu.ar/index.php/RIHALC/article/view/41255}.}

\section{Introducción}

La historia de la sociología argentina se ha escrito centralmente desde su capital. La descripción del cambio de condiciones sociales en el posperonismo, en coincidencia con modificaciones cualitativas en la forma de practicar la disciplina, es recurrente entre los estudiosos del campo \parencite[entre otros]{1450-VERON1974,1508-SIDICARO1993,1550-BLOIS2018,1565-BLANCO2006}. A su vez, suele indicarse como observable empírico de este proceso la actuación de Gino Germani al frente de la carrera y el departamento de sociología, fundados en la Universidad de Buenos Aires en 1957.

Un advenedizo en este campo de estudios, que comience indagando en los textos más conocidos que explican esta mutación (por ejemplo, Germani, 1964), podría suponer que la disputa entre \enquote{sociología científica} y \enquote{sociología de cátedra} no hace sino reproducir en el marco de las ciencias sociales el viejo dilema argentino entre Buenos Aires y el Interior que, en definitiva, no es otro que el de \enquote{civilización vs barbarie}. Sin embargo, cuando el análisis sociológico se sitúa en un nivel micro e incluso meso social, el panorama se revela bastante más complejo de lo que, a priori, podría pensarse.

Entiéndase entonces el sentido de este trabajo. Aquí no se trata de hacer la historia de \enquote{los que perdieron}, sino más bien comprender que, por un lado, el grupo que a partir de mediados del siglo pasado estableció desde Buenos Aires las directrices sobre cómo debía concebirse y practicarse la sociología no lideró de forma automática este campo a nivel nacional y, por otro lado, que los viejos modos de ejercer la disciplina no solo siguieron existiendo durante un tiempo más que considerable, sino que además, muchas veces convivieron armoniosamente con las nuevas formas.\footnote{De hecho, el grupo porteño no fue homogéneo en su interior. Si se le presta atención a las producciones de varios aliados de Germani, como Carlos Alberto Erro o Norberto Rodríguez Bustamante, difícilmente podría enmarcárselos dentro de la \enquote{sociología científica}. En la vereda de enfrente, intelectuales que podrían considerarse renovadores de la sociología argentina como Juan Carlos Agulla, María Angelina Roggero, José Enrique Miguens, Adoflo Critto o José Luis de Ímaz compartieron con los \enquote{sociólogos tradicionales} un espacio como la Sociedad Argentina de Sociología, presidida por Alfredo Poviña.}

De allí el sentido de ocuparse de una figura marginal de esta historia como Guillermo Terrera. Representante de la \enquote{sociología de cátedra} y profesor de la asignatura en las universidades de Córdoba y Buenos Aires, entre otras instituciones, su derrotero es un indicador de la supervivencia de esta tradición con posterioridad a 1957. De hecho, esta sociología practicada por abogados, con todas las características que posee y que se detallarán un poco más abajo, seguiría vigente en Argentina por lo menos hasta la década de 1980.

De allí que antes de pensar en \enquote{etapas} al estilo de Germani (1964), o en \enquote{escuelas} como Marsal (1963), sea más interesante abordar esta historia desde la perspectiva de \textcite{1613-DELICH1977}, quien propone estudiar los distintos \enquote{estilos de trabajo} presentes entre los sociólogos locales. Por ello, si bien es cierto que lo que este último autor rotula como \enquote{sociología de frac} tuvo su momento hegemónico a nivel institucional hasta el primer gobierno peronista, una de las características de la sociología argentina posterior es la coexistencia de varios estilos sociológicos que se superponen en el tiempo.\footnote{Además de la ya mencionada, se encuentran la llamada \enquote{sociología \emph{white collar}} o científica, cuyo autor más representativo sería Germani, y la \enquote{sociología descamisada} o nacional/peronista. A su vez, si bien nunca fue dominante en la universidad, también debería mencionarse la \enquote{sociología marxista}. En el caso de Córdoba, dice Delich en 1977, \enquote{el momento de la sociología de frac se prolonga hasta la actualidad}. Y más adelante agrega: \enquote{es cierto que cronológicamente [los tres estilos de trabajo] aparecieron en forma sucesiva, pero la emergencia de una no implica la desaparición de la antigua} \parencite[28]{1613-DELICH1977}.}

Entonces, la reconstrucción de la trayectoria de Terrera supone seleccionar aspectos relevantes de la vida de este agente social, es decir, dar cuenta de una serie de esferas de acción en las cuales intervino, siendo la sociología la de mayor interés. Aunque la propuesta corre el riesgo de caer en \enquote{la ilusión biográfica}, que supone instituir una identidad social duradera por el solo hecho de que el agente se ubica siempre bajo el mismo nombre, aquí se entiende la trayectoria \enquote{como la serie de posiciones sucesivamente ocupadas por un mismo agente (\dots) en un espacio en devenir y sometido a incesantes transformaciones} \parencite[127]{1578-BOURDIEU2011}.

Esto es, si los individuos poseen historias de vida, las mismas son principalmente, y desde el punto de vista del sociólogo, las historias \enquote{de las relaciones de interdependencia que establecieron con otros individuos} (Lahire, 2016:47). Por ello, si bien la diferenciación social propia de las sociedades complejas daría cuenta de la unicidad de una biografía, la \enquote{identidad personal puede desempeñar, y de hecho desempeña, un rol estructurado, rutinario y estandarizado en la organización social, precisamente a causa de su unicidad} (Goffmann, 2010:79). El itinerario de Terrera, que incluyó la estación de la sociología, da cuenta de la persistencia de un grupo social que continuó con una determinada forma de ejercicio de la disciplina. La producción sociológica de este autor mostrará entonces cómo \enquote{en lo más personal se lee lo más impersonal, y en lo más individual, lo más colectivo} (Lahire, 2006:165).

En el fondo, esto no deja de ser un ejercicio de lo que \textcite{1451-WRIGHTMILLS1956} popularizó bajo el nombre de \enquote{imaginación sociológica}, es decir, el entrecruzamiento de biografía, historia y estructura social. De tal forma que aquí el punto de partida resultará del \enquote{reconocimiento cognoscitivo} de Terrera, es decir, del \enquote{acto perceptual de \enquote{ubicar} a un individuo, en tanto poseedor de una identidad social} (Goffman, 2010:91). Terrera fue un profesor universitario de sociología que, aún varios años después de que Germani implantara la \enquote{sociología científica} en Buenos Aires, siguió enseñando la disciplina bajo formas \enquote{tradicionales}, las cuales suelen caer bajo la denominación de \enquote{sociología de cátedra}. Pero, ¿qué debería entenderse por esta etiqueta?

La \enquote{sociología de cátedra} podría definirse a partir de una serie de características, algunas de las cuales efectivamente formaron parte de la práctica de la disciplina durante la primera mitad del siglo~XX, mientras que otras son más bien construcciones atribuidas a ese grupo de \enquote{sociólogos tradicionales}, antes que aserciones pasibles de contrastación empírica. En primer lugar, en aquella época la sociología era una materia enseñada en las universidades como formación complementaria para otras carreras, centralmente abogacía, historia y filosofía, aunque también se dictó para estudiantes de economía e, incluso, de agronomía como en el caso de la Universidad Nacional de La Plata.

En segundo lugar, al respecto de las características de estos profesores, se trató en general de abogados que hacían de la enseñanza un complemento de sus menesteres principales vinculados a la jurisprudencia. En efecto, muchos de los docentes fueron profesionales del derecho que obraron como jueces (Juan Agustín García, Ernesto Quesada, Alfredo Poviña, Alberto Baldrich), fiscales (Carlos O. Bunge) o trabajaron en estudios jurídicos del sector privado (Raúl Orgaz). Sin embargo, aunque en menor proporción, también hubo profesores de sociología formados en filosofía (Jordán Bruno Genta) y psicología (José Oliva).

En tercer lugar, se ha caracterizado a la \enquote{sociología de cátedra} como una forma de enseñanza libresca de la materia. Esto es verdadero y va en consonancia con el carácter enciclopédico de la disciplina a nivel internacional, el cual se mantuvo hasta la adopción de la \enquote{tesis de la convergencia} parsoniana, expuesta en \emph{The Structure of Social Action} (1937), durante la segunda posguerra. En cuarto lugar, esa enseñanza era \enquote{el límite} de los sociólogos de cátedra, en tanto no desarrollaron investigaciones empíricas. Esto es cierto en gran medida, siempre teniendo en cuenta que varios profesores de sociología atribuyeron suma importancia a la investigación científica e, incluso, algunos de ellos llegaron a realizar parcialmente algunas aproximaciones empíricas.\footnote{Estos serían los casos de Quesada, quien comenzó pero no llegó a completar un estudio sobre la clase media en Alemania (Pereyra, 2000), Francisco Ayala, impulsor de las primeras investigaciones en la Universidad Nacional del Litoral \parencite{1619-ESCOBAR2011} y Ricardo Levene, quien dirigió las primeras indagaciones sobre la realidad social argentina en el Instituto de Sociología de Buenos Aires (González Bollo, 1999).}

En quinto y sexto lugar aparecen las características más reiteradas entre los practicantes de la sociología de la época peronista, aunque se trata de elementos que comúnmente se asocian al conjunto de la sociología pregermaniana. Por un lado, está el antipositivismo sociológico, cuya \enquote{especulación desenfrenada} llegaba al punto de constituirse en filosofía social más que sociología (Germani, 1962) y, por otro lado, se encuentra la adhesión a la doctrina de la Iglesia Católica \parencite{1565-BLANCO2006}. Ahora bien, en la mayoría de los profesores de sociología de la primera mitad del siglo~XX esto no fue así: Antonio Dellepiane, Juan Agustín García, Ernesto Quesada, Leopoldo Maupas, Alfredo Colmo, Ricardo Levene, Enrique Martínez Paz, Raúl Orgaz, José Oliva y José María Rosa no fueron antipositivistas y el catolicismo tampoco resultó central en sus reflexiones sociológicas.

Algo parcialmente distinto ocurre con Poviña, quien impulsó la crítica al positivismo, pero negaba la posibilidad de una \enquote{sociología católica}.\footnote{Para \textcite{1532-POVINA1955} esto se debe a que el cristianismo excede a la ciencia social. Esta última solo tiene por finalidad explicar la conducta humana, a diferencia de la primera que prescribe un \enquote{deber ser}.} Sin embargo, pueden mencionarse varios autores con las dos últimas características descriptas: Gustavo Martínez Zuviría, Rodolfo Tecera del Franco, Fernando Cuevillas, Alberto Baldrich, Jordán Bruno Genta, Francisco W. Torres, Alberto Díaz Bialet, Francisco Soler Miralles y, el objeto del presente trabajo, Guillermo Terrera. Todos ellos, con excepción de Martínez Zuviría y Genta, fueron profesores de sociología durante el peronismo y estuvieron ligados a la Iglesia Católica,\footnote{De todas maneras, también entre los sociólogos católicos hubo debates internos en torno al positivismo. Por ejemplo, Octavio Derisi, fundador de la Universidad Católica Argentina, fue sumamente crítico de esta corriente, mientras que José Enrique Míguens y Antonio Donini, miembros de la siguiente generación de intelectuales del catolicismo, tuvieron una mirada mucho más comprensiva respecto a esta tradición de pensamiento \parencite{281-ZANCA2006}.} por lo que más de uno se ganó el título de \enquote{flor de ceibo}.

Entonces, este trabajo tendrá en cuenta la trayectoria social de uno de los representantes de la \enquote{sociología de cátedra}, con el objetivo de corroborar la permanencia de las características señaladas en la enseñanza de la materia. Como ya se ha dicho, esta tradición ha sido muchas veces descuidada por los estudiosos del campo, quienes se han interesado más por la sociología científica y las sociologías denominadas \enquote{comprometidas}, ya sea en su versión nacional/peronista o marxista, durante las décadas de 1960 y 1970. Sin embargo, una indagación con mayor profundidad mostraría que la \enquote{sociología de cátedra},

\begin{quote}
(\dots) aunque sus esclerosadas prácticas academicistas limitadas a lo pedagógico desentonaran con las renovadas técnicas de investigación social y perfiles intelectuales comprometidos [en las décadas de 1960 y 1970], no dejaba de ser la tradición sociológica más antigua del país y de controlar la mayor parte del campo sociológico nacional \parencite[2-3]{1615-DIAZ2013}.
\end{quote}

De esta forma, teniendo en cuenta el sentido de la selección de la trayectoria y obra de Guillermo Terrera, así como las particularidades de la \enquote{sociología de cátedra}, se avanzará en un primer momento sobre el devenir de la vida del autor, dando cuenta de los espacios sociales por los cuales circuló, los cargos que ocupó y sus principales vínculos con la sociología. En una segunda instancia, se abordarán las concepciones sociológicas de Terrera, sus modificaciones en el tiempo y la perspectiva que finalmente dominó en su obra. Por último, las conclusiones retomarán los elementos más importantes que hayan surgido a lo largo del trabajo, con el objetivo de demostrar la supervivencia de la \enquote{sociología de cátedra} en las décadas posteriores al advenimiento de la \enquote{sociología científica}.

\section{Guillermo Terrera I: sociología, antropología y otras yerbas}

Guillermo Alfredo Terrera\footnote{Casi todos los datos que se citarán a continuación, y que constituyen la mayor parte de la trayectoria de Terrera, fueron extractados de su \emph{Currículum Vitae} (1974).} nació en la ciudad de Córdoba el 26 de septiembre de 1922 y fue hijo de Nicolás Terrera y Margarita Patrucco. Estudió en la Facultad de Derecho y Ciencias Sociales de la Universidad Nacional de Córdoba (FDCS-UNC), donde se recibió como escribano y abogado en 1946. Tres años después se graduaría como doctor en derecho y ciencias sociales con una tesis titulada \emph{Sociología de la Educación}, publicada al año siguiente por la propia universidad, y en 1951 obtendría su título de profesor universitario diplomado. Al respecto de esto último, puede observarse que la docencia fue su actividad principal, aunque también la combinó con cargos administrativos.

Sus puestos más importantes en la Universidad Nacional de Córdoba tuvieron lugar a partir de 1946, cuando la universidad fue intervenida por el peronismo, movimiento político en el cual militó desde sus orígenes. Entre la victoria electoral de 1946 y el derrocamiento de 1955, Terrera ejerció varios cargos al frente del aula: fue profesor de prehistoria, historia antigua y sociología en la FDCS-UNC (donde además fue consejero titular entre 1953-1955); dictó instrucción cívica e historia argentina en el Colegio Nacional Deán Funes; estuvo a cargo de las materias plan quinquenal y sociología y política en la Escuela Sindical de la Confederación General del Trabajo; y enseñó sociología en los Cursos para Oficiales Superiores de la Policía de la Provincia de Córdoba. A su vez, fue investigador del Instituto de Arqueología, Lingüística y Folklore.

En estos años, Terrera trabajó para el Banco de la Nación Argentina (1942, 1947), el Consejo Nacional de Estadística y Censos (1945), el Departamento Provincial del Trabajo (1942), fue vocal del Tribunal de Cuentas de la Municipalidad de Córdoba (1947), asesor letrado y presidente de la Comisión Administradora del Trasporte Automotor (1949), asesor letrado del Ministerio de Obras Públicas de la Provincia de Córdoba (1940) y de la Policía de la Provincia de Córdoba (1950-1955), miembro de la Comisión de Presos y Liberados (1951) y de la Comisión Redactora del Código de Faltas de la Provincia de Córdoba y presidente del Consejo Supremo de Justicia Policial de la Provincia de Córdoba (1954-1955).

Luego del golpe de estado de 1955, Terrera permanecería alejado de la universidad pública hasta la década de 1970, cuando la legalidad del peronismo fue restablecida. En esos 18 años de proscripción se desempeñó en instituciones privadas como el Instituto Superior Universitario de Boulogne, donde enseñó antropología cultural, sociología, historia argentina, lógica e historia de la civilización y ejerció el decanato de la Facultad de Ciencias de la Educación. También dictó sociología en la Facultad de Derecho y Ciencias Sociales de la Universidad Bartolomé Mitre de Olivos (1966-1968) y las materias historia argentina e historia contemporánea en el Colegio Nacional Suizo-Argentino, donde además fue rector.

Terrera también participó activamente de la Universidad Argentina John Kennedy. Aquí enseñó antropología cultural y tuvo a su cargo los decanatos del Departamento de Antropología y de la Escuela de Ciencias de la Educación y, antes del retorno del peronismo a la vida política nacional, fue profesor de política social en el Instituto Superior Diocesano de San Isidro (1970-1972), ciudad donde residió con su esposa, la escribana Eduviges Villar Azurmendi, y sus tres hijos. En 1974, momento en que se acentuó el clima represivo durante el gobierno peronista, Terrera tuvo la oportunidad de volver a trabajar en la FDCS-UNC, a partir de la creación de las cátedras \enquote{B} y \enquote{C} de sociología por parte del decanato de la facultad \parencite{1558-CHAMORROGRECA2007}.

Una vez instaurado el golpe de estado de 1976, y después de haber permanecido cerrada durante varios cuatrimestres, la carrera de sociología de la Universidad de Buenos Aires se reabriría con un plantel docente totalmente nuevo, del cual Terrera formaría parte. En particular, dictaría la materia política social, que estaba dedicada a estudiar la organización y articulación de la sociedad y el estado en sus distintos niveles. Sin embargo, hacia finales de la década de 1970 Terrera tendría un giro importante en su derrotero intelectual, lo cual lo haría un autor mucho menos prolífico en materia sociológica. Luego de 1978, se dedicaría con entusiasmo creciente al estudio de cuestiones vinculadas a la metafísica, la espiritualidad, la ufología, el ocultismo y el esoterismo, publicando varios libros sobre estos temas hasta su fallecimiento en 1998.\footnote{En la transición hacia estos temas pareciera haber jugado un rol importante el fallecimiento de su esposa, según se da a entender en el documental sobre su vida titulado \enquote{30 años de silencio: El secreto de Guillermo Alfredo Terrera}. El mismo se encuentra disponible en la plataforma YouTube: \url{https://www.youtube.com/watch?v=Cyq8QeW85rY\&t=671s}.}

Recapitulando, puede apreciarse que Terrera fue durante poco más de treinta años un abogado que se dedicó centralmente a la docencia, con una marcada inclinación hacia las ciencias sociales y, en especial, a la sociología. De hecho, fue miembro de la Asociación Latinoamericana de Sociología (ALAS) y participó en encuentros de esta disciplina como los Congresos Nacionales de Sociología de Buenos Aires (1950) y Santa Fe (1971) y los Congresos Internacionales de Sociología de Buenos Aires (1953; 1960). A su vez, siguió cursos de especialización de esta disciplina con Alfredo Poviña (1945), Raúl Orgaz (1946) y Francisco W. Torres (1947) en la Universidad Nacional de Córdoba y con Alberto Baldrich (1950) en la Universidad de Buenos Aires.

Sin embargo, este recorte de su trayectoria impediría ver otras facetas del autor, ya que Terrera también se interesó por temas fronterizos de la sociología como el folklore, la tradición, la lingüística, la política, etcétera, y formó parte de asociaciones con finalidades muy divergentes. Por ejemplo, fue miembro de la Academia Americana de la Historia, de la Sociedad Argentina de Escritores, de la Agrupación Gauchos de la Patria y del \emph{Journal of American Studies}, entre otros. Por fuera del mundo académico, estuvo dedicado a actividades agropecuarias, rubro donde obtuvo las membrecías de la Asociación de Criadores de Aberdeen Angus, la Junta Nacional de Carnes y la Federación Agraria Argentina. A su vez, se desempeñó como Síndico de la Cooperativa de Productores \enquote{Unión Tamberos del Centro} de Córdoba.

Esta pluralidad de pertenencias sociales y actividades también se expresa en su obra. A lo largo de toda su trayectoria, Terrera publicó más de setenta libros sobre los temas más variados: historia, política, antropología, sociología, lingüística, musicología, política universitaria, política agropecuaria y metafísica. Al mismo tiempo, presentó más de veinte anteproyectos de leyes y ofreció más de ciento cincuenta conferencias al público en diversas instituciones nacionales, entre las cuales pueden destacarse varias dedicadas al pensamiento sociológico: \enquote{Sociología de la realidad} (1949), \enquote{La sociología de Alfredo Vierkandt} (1949), \enquote{La sociología de Max Weber} (1949), \enquote{Realidad social argentina} (1950), \enquote{Sociología y política} (1951), \enquote{Los procesos socioeconómicos, desplazamientos humanos internos y externos} (1972), \enquote{Marxismo y peronismo} (1974), entre otras.

Buena parte de sus libros fueron publicados por instituciones locales como las universidades nacionales de Córdoba, Buenos Aires, Litoral, Cuyo, la Sociedad Folklórica Argentina, la Academia Argentina de Letras y las editoriales El Escorial, Peña Lillo, Minerva y La Reforma, aunque la mayoría fueron editadas por Plus Ultra. Sus trabajos también suscitaron interés en muchos países extranjeros, siendo varios de ellos divulgados por el Instituto de Cultura Hispánica de Madrid, la embajada de la República de Irak, el British Broadcasting Corporation (Londres) y las universidades de Hamburgo (Alemania), Teherán (Israel), Central de Caracas (Venezuela), Bagdad (Irak), Trípoli y Bengasi (Libia).\footnote{Aquí es interesante señalar que Terrera tuvo vínculos políticos internacionales por demás relevantes con Medio Oriente, entre los que se destaca La Liga Árabe. Por este motivo es que realizó varios viajes en la década de 1970 a países de esta región y dedicó unos cuantos escritos a cuestiones ligadas a la geopolítica.} Esto implicó también que fuera traducido a varios idiomas: inglés, francés, alemán, italiano y árabe (Terrera, 1993).

Por último, como ya se ha dicho, Terrera fue de simpatía peronista y, de hecho, siempre militó en partidos de tendencia nacionalista. En su juventud formó parte de FORJA (Fuerza de Orientación Radical para la Joven Argentina) en 1940, fue uno de los fundadores de la Unión Federal Demócrata Cristiana (1943) y se afilió al Partido Laborista en 1945.\footnote{Fue el afiliado número 27 en el orden nacional.} Esto le valió su detención en 1955 por parte del Comando Militar de Córdoba, el cual emitió una orden de fusilamiento que finalmente no fue cumplida, por lo que permaneció detenido en la cárcel militar de la provincia durante el año 1956. Una vez liberado, fundó el Movimiento Nacional Revolucionario (MNR) en 1958 y lideró el Grupo Revolucionario \enquote{Patria Vieja} (1957-1963). Dados los intentos de derrocamiento de la dictadura instaurada en 1955, Terrera se vio obligado a exiliarse del país en tres oportunidades distintas: dos veces en Uruguay (1957 y 1960) y una en Bolivia (1963).

En definitiva, puede apreciarse que la trayectoria de Terrera muestra una gama de actividades académicas y extra-académicas que podrían sugerir la realización de varios trabajos en torno a su persona y obra. Sin embargo, como se ha señalado al comienzo, aquí interesa su desempeño como sociólogo y, más específicamente, como representante de una forma particular de practicar esta ciencia social, la cual se corresponde con las características ya descriptas que definen a la \enquote{sociología de cátedra}. Por ello, y dado que Terrera no ha suscitado más que unos pocos comentarios entre los especialistas,\footnote{Entre ellos, por ejemplo, \textcite[22]{1447-CARACCIOLO2010} solo señaló que Terrera era una figura cercana \enquote{al peronismo y la Iglesia}, sin ahondar con mayor atención entre los profesores de sociología cordobeses durante la época peronista.} es importante para los objetivos del presente trabajo saber cuál era el sentido que este autor atribuía a la sociología, cómo la definía, de qué problemas intelectuales debía ocuparse, etcétera. Entonces, en el próximo apartado se seleccionarán algunos de sus trabajos más relevantes en torno a la disciplina, con el objeto de dar cuenta de su categorización como exponente de la tradición de la \enquote{sociología de cátedra}.

\section{Guillermo Terrera II: sociología, ciencia del espíritu}

Como ya se ha dicho en el apartado anterior, Terrera fue autor de más de setenta libros, entre los cuales una porción importante estuvo dedicada a la sociología (véase Anexo). Debido al giro hacia la metafísica que este autor experimentó luego de 1978, aquí se le prestará especial atención a sus obras sociológicas (y sobre temas adyacentes a la sociología) hasta esa fecha. Entre ellas se encuentran algunos de los manuales que publicó durante la década de 1960, como su \emph{Tratado teórico-práctico de sociología} (1969). Este texto fue su obra más importante en el campo sociológico, ya que se convirtió en una lectura recomendada por la UNESCO para la enseñanza de la disciplina en América Latina.

Sin embargo, sus primeros escritos no estuvieron vinculados a la sociología, sino que fueron intervenciones políticas. Por ejemplo, en \emph{Comentario político-social argentino} Terrera (1946:13) sentaba su postura al respecto de los acontecimientos que estaba viviendo el país diciendo \enquote{soy partidario de Perón}. Aquí enumeraba los logros de la gestión del hasta entonces secretario de trabajo y lanzaba acusaciones violentas contra la \enquote{oligarquía plutocrática}, la cual estaba compuesta varias asociaciones empresariales del país: Unión Industrial Argentina, Centro y Bolsa de Comercio, Sociedad Rural Argentina, entre otras. A su vez, señalaba con virulencia a los \enquote{coimeros}, a los \enquote{diputados mudos}, a la \enquote{oligarquía decadente}, a la \enquote{oligarquía universitaria}, a las \enquote{familias patricias}, pero sobre todo, y especialmente, \enquote{a esos señores que hablan rumbosamente de libertad y de normalidad constitucional, cuando uno debería gritarles que renuncien a los dos o tres puestos que ocupan y a las 50!! horas de cátedra que detentan, como hay muchos} (Terrera, 1946:40).

Terrera identificaba con claridad a sus adversarios en la universidad, quienes ostentaban los cargos de mayor jerarquía, lo cual se modificaría a su favor al año siguiente con la sanción de la ley universitaria del peronismo.\footnote{Como se comentó en el capítulo anterior, la ley 13.031 indicaba en su artículo 59 que ningún profesor titular podría desempeñarse al mismo tiempo en más de una cátedra en caso de ser titular, lo cual beneficiaría a quienes se encontraban fuera de la universidad. Por otra parte, las intervenciones a las casas de estudio superiores, las cesantías y renuncias de los opositores también redundarían en ventajas para el acceso a cargos de los partidarios del gobierno peronista.} Por ello llamaba a luchar contra el nazismo y el comunismo, defendiendo la democracia social impulsada por el coronel Juan Domingo Perón. Ni demasiado \enquote{pauperismo} ni excesivo \enquote{súpercapitalismo}, lo que traería felicidad al pueblo era la justicia social iniciada el 4 de junio de 1943. Para llevar esto a cabo, según sostendría en la disertación \emph{Nacionalismo social argentino} (1949), y desde la perspectiva de una sociología de la cultura, era necesario un conductor para las masas nativas.

\begin{quote}
Los conductores son los personajes que manejan la acción de las masas. Ellos representan por sí solos la ecuación colectiva del grupo y se convierten en el símbolo animado o viviente de todos los yo individuales proyectados en el yo de la comunidad. El caudillo sugestiona y subordina a los grupos, pues la propia multitud en función de su incapacidad para dirigirse organizadamente entrega la autoridad y la jerarquía en manos de su conductor (Terrera, 1949:27).
\end{quote}

Terrera indicaba que el liderazgo era una necesidad para los movimientos multitudinarios en la historia argentina. Por lo tanto, resultaba necesario un trabajo en conjunto de filósofos y sociólogos nacionales para

\begin{quote}
(\dots) estructurar y sistematizar científicamente todo el material de lo cultural argentino, pues necesitamos crear una filosofía y una sociología de lo autóctono, pero apoyándonos en los métodos, procedimientos y sistemática científica de los conceptos universales e inmutablemente aceptados, para obtener una corriente doctrinaria con un claro sentido de lo argentino (Terrera, 1949:33-34).
\end{quote}

Esta última indicación es por demás interesante porque Terrera, si bien reclamaba autenticidad en los temas de estudio que corresponderían a una sociología vernácula, no renegaba de los procedimientos científicos aceptados de forma universal. De hecho, en su tesis doctoral, y a diferencia de otros exponentes de la \enquote{sociología de cátedra}, reconocía que a nivel internacional se estaba produciendo el abandono de las viejas prácticas teóricas de la disciplina así como aproximaciones al campo de la experimentación y de las realizaciones prácticas. \enquote{Este nuevo impulso de la ciencia social se ha hecho sentir con notable brío en los pueblos de origen anglo sajón o nórdico [sic], ya de suyo grandemente realizadores y prácticos} (Terrera, 1950:11).

De todas maneras, la tesis estaba enfocada en una sociología especial (de la educación), que Terrera separaba de la sociología general, plasmando por primera vez una delimitación de los alcances de esta ciencia que reiteraría en los manuales de la década de 1960. Para Terrera (1950:14), la sociología \enquote{es la ciencia de las relaciones humanas y sus productos, esto último dicho sin ánimo de definir, pues entiendo que las definiciones nunca abarcan el contenido total de la ciencia que define}.

Esta conceptualización era la misma que Alfredo Poviña (1945), profesor de sociología de Córdoba y Buenos Aires, había tomado de Alfred Vierkandt y que luego Terrera hizo suya, ya que estaría presente a lo largo de sus escritos posteriores. La misma venía de una división de aquello a lo cual los sociólogos podían prestarle atención. Por un lado, estaba la sociedad \enquote{en estado naciente}, es decir, las acciones e inter-acciones entre individuos, siempre de carácter \enquote{espiritual} o \enquote{intermental} y, por otro lado, la sociedad institucionalizada, equivalente a la definición de \enquote{instituciones} desarrollada por Émile Durkheim en \emph{Las reglas del método sociológico}.

Sin embargo, a pesar de esta amplitud en la definición, Terrera tomaba partido por el nominalismo sociológico ya que entendía que la sociedad estaba formada por dos elementos fundamentales: \enquote{la simple \emph{agregación} de individuos (\dots) y los procesos de \emph{interacciones} entre todos y cada uno de ellos} (Terrera, 1966:9, énfasis en original). Al mismo tiempo, aceptaba la distinción formulada por Wilhem Dilthey entre ciencias de la naturaleza y ciencias de la cultura. Estas últimas estudian los productos del espíritu humano, \enquote{que no tienen cantidad, sino cualidad, son ciencias de cualidades y no pueden ser medidas ni experimentadas empíricamente} (Terrera, 1966:17).

Esta afirmación resulta toda una declaración de principios en la medida que la sociología ingresaba dentro de estas últimas, siendo además caracterizada como una \enquote{ciencia general y abstracta} (Terrera, 1966:20), es decir, que la ciencia social fundamental y general sería la sociología, mientras el resto de las ciencias sociales particulares no serían más complementos que aportarían a la comprensión de la totalidad social. Por lo tanto, Terrera seguía la línea iniciada por Raúl Orgaz y Alfredo Poviña en la Universidad Nacional de Córdoba a mediados de la década de 1920. De hecho, la estructura de los manuales de Terrera (1966; 1969a) imita en gran medida lo realizado por Poviña en su trabajo de mediados de los años cuarenta.

En ambos textos aparecen capítulos dedicados a la definición de sociología, las relaciones de la sociología con otras ciencias particulares, la historia de la disciplina desde la Grecia clásica hasta comienzos del siglo~XX, los métodos de la sociología, la conciencia social, la organización y el cambio social, el grupo regulado y su unidad funcional, la historia de las ideas argentinas, etcétera. No obstante, entre las sociologías especiales había una que a Terrera le interesaba especialmente que era la sociología jurídica, lo cual estuvo presente en las reflexiones de \textcite{1533-POVINA1954} aunque no con una importancia mayor que otras ramas de la disciplina.

Ahora bien, en el marco de la \enquote{sociología de cátedra} un aspecto llamativo de los textos de Terrera es en relación con el método. En el capítulo dedicado a este tema realizaba una compilación de las diferentes metodologías utilizadas por los sociólogos en sus investigaciones para luego clasificarlas en dos grandes casilleros: \enquote{los métodos \emph{cuantitativos} que buscan conocer el aspecto externo de los \emph{fenómenos sociales} y (\dots) los métodos \emph{cualitativos}, que se dedican exclusivamente a conocer la \emph{trama psíquica} del \emph{hecho} o \emph{fenómeno social}} (Terrera, 1966:46, énfasis en original). Ambos tipos de método, según entiende el autor, no se excluyen sino que se auxilian y complementan unos a otros, cuestión que en efecto fue llevada a la práctica por el propio Terrera en la investigación que realizó sobre el vocabulario del habla popular argentina. Allí sostenía que el método implicado en su pesquisa

\begin{quote}
(\dots) es el denominado \enquote{histórico-cultural}, que realiza dos etapas diferentes de trabajo. En la primera, observa, constata, recopila y clasifica la mayor cantidad de elementos, sean estos de orden material o espiritual. Esta posición metodológica es generalmente de índole cuantitativa y busca reunir cantidades de elementos a investigar. A partir de esta clasificación, el método se convierte en cualitativo y surgen formas intelectuales de deducción, conclusión, análisis y generalización (\dots). [Estas dos metodologías están ligadas entre sí] pues el planteo cuantitativo en un trabajo como el presente es base imprescindible para la evaluación cualitativa y esta, a su vez, para el anterior (Terrera, 1969b:11).
\end{quote}

No obstante, tan solo con una somera lectura de sus textos se evidencia que los conocimientos de Terrera sobre los avances en materia de cuantificación social, en pleno auge por estos años, eran por demás endebles. Por ejemplo, en el capítulo 11 de su \emph{Tratado teórico-práctico de sociología} (1969a), luego de una escueta definición de las clases sociales y sin citar ninguna fuente ni explicar la forma de construcción de sus datos, Terrera señalaba que en las llamadas \enquote{comunidades abiertas}, de las cuales Argentina era un ejemplo, la distribución poblacional entre las clases alta, media y baja sería 10\% la primera, 60\% la segunda y 30\% la tercera. Por el contrario, en las \enquote{comunidades cerradas}, también llamadas \enquote{grupos humanos de tipo totalitario}, un 15\% de la población pertenecería a la clase alta mientras que el 85\% restante formaba parte de la clase baja o pueblo.

Evidentemente, Terrera resultaba anacrónico para la forma en que se practicaba la sociología de entonces. Esto se reiteraba en el plano político, lo cual se hace palpable en varios trabajos de comienzos de la década de 1970 (Terrera, 1970, 1971, 1972, 1973b), los cuales son ambiciosos proyectos de reforma social que incluían una serie de aspectos muy amplios como la familia, la vivienda, la salud, el turismo y el descanso, la educación, la seguridad social, la industria, el transporte, el sistema bancario e impositivo, la justicia, la producción agropecuaria, la obra pública, la descentralización burocrático-administrativa, entre otros. En síntesis, podría decirse que Terrera rechazaba los procesos revolucionarios que la juventud que adhería a su partido político impulsaba en estos años y, por el contrario, abogaba por la construcción de un nacionalismo corporativista, donde la representación pudiera ser practicada directamente por los miembros de las asociaciones profesionales y prescindir así de los políticos profesionales.

Finalmente, deben citarse dos gruesos volúmenes de mediados de los años setenta: \emph{Antropología social y cultural} (1973a) y \emph{El hombre y la sociedad} (1975), aunque ninguno de los dos está dedicado estrictamente a la sociología sino más bien a contenidos de sus orillas. El primero de ellos, por obvias razones, está vinculado a la antropología aunque podría decirse que en partes iguales a la rama biológica de la disciplina cuanto a la social y cultural. Un aspecto relevante del libro, y que se reitera en el segundo trabajo citado, es la abundancia de referencias a la historia social del hombre. De hecho, \emph{El hombre y la sociedad} contiene varios capítulos donde se estudian la historia argentina y americana, con el objetivo de desentrañar qué es lo que constituye el verdadero \enquote{ser nacional} argentino, tema sobre el cual Terrera publicaría un libro en la década siguiente (Terrera, 1983).

A su vez, en el libro de 1975 vuelve a aparecer el derecho y su vínculo con la sociedad, es decir, la ya mencionada sociología jurídica que había captado su atención desde sus primeras publicaciones y, nuevamente, una abundante cantidad de páginas dedicadas a la reforma social que debía impulsarse en el país. Por último, como también se ha mencionado un poco más arriba, en estos años Terrera estableció vínculos políticos importantes con algunos países de Medio Oriente. Por este motivo es que aparecieron algunos escritos orientados a la geopolítica (por ejemplo, Terrera, 1976, 1979) los cuales, sin embargo, no revisten mayor importancia para los fines del presente trabajo.

En síntesis, podría decirse que Terrera tuvo un primer período de producción intelectual fuertemente ligado a la sociología. Son los años del primer peronismo cuando no solo toma cursos con Orgaz, Poviña y Baldrich, sino que además ingresa a dictar clases en la FDCS-UNC. Es a partir de esta formación sociológica que Terrera adopta su perspectiva al respecto de las características y modalidades de ejercer la disciplina. Con posterioridad, no se observa ninguna innovación de importancia en relación a la materia durante los años en los cuales el peronismo se mantuvo en la ilegalidad y Terrera publicó sus obras más célebres.

En este sentido, su trayectoria ligada al grupo de sociólogos de la Facultad de Derecho de la UNC así como su adhesión al peronismo devinieron factores de primer orden para explicar su desencuentro con los sociólogos científicos asentados en Buenos Aires. Por ello, no solo la teoría de la modernización sino también las metodologías de investigación empírica que Germani impulsó desde la carrera y el instituto de sociología de la UBA quedaron muy alejadas de los manuales que Terrera diera a conocer en la década de 1960. Si esto fue así en estos años, durante el decenio siguiente Terrera se alejaría aún más del centro del debate sociológico a nivel nacional, cuando las \enquote{sociologías comprometidas}, ya sean peronistas o marxistas, se hicieron con los lugares de preeminencia en los espacios de enseñanza de la disciplina en las universidades locales \parencite{1508-SIDICARO1993}.

\section{Conclusiones}

En este capítulo se han reconstruido algunos aspectos centrales del derrotero de un personaje secundario en la historia de la sociología argentina. Esta trayectoria, sin embargo, presenta un elemento de suma relevancia para entender la persistencia en el tiempo de formas de enseñanza de la sociología que, aunque crecientemente en desuso y opacadas por la perspectiva dominante centrada en Buenos Aires, no dejaron de existir. Si bien es cierto que se está hablando de un agente marginal, es interesante destacar que cuando se recuperan sus instancias de socialización ligadas a la sociología, aparece una pluralidad de personalidades a partir de las cuales podrían escribirse trabajos similares al presente. Esto da cuenta de que, en definitiva, Terrera no fue un individuo aislado, sino que formó parte de un entramado de sociólogos marginales cuya suma da por resultado un hecho cualitativamente distinto.

Esta red sociológica estuvo mayormente constituida por abogados con concepciones y formas de practicar la sociología similares a las que se han reconstruido para el caso de Terrera. Muchos de ellos mantuvieron sus cargos en las universidades del Interior del país y sus membrecías en asociaciones sociológicas durante las décadas de 1960 y 1970. Para dar cuenta del tamaño de esta red, que podría caracterizarse como \enquote{secundaria} o \enquote{alternativa} al \emph{mainstream sociológico} porteño, puede citarse el trabajo de Pedro David. Este autor escribió la introducción del volumen colectivo titulado \emph{La sociología de Miguel Herrera Figueroa}, dedicado a otro \enquote{sociólogo tradicional} que fue profesor en Tucumán, donde da cuenta de una buena parte de los miembros de la Sociedad Argentina de Sociología

\begin{quote}
(\dots) que presidiera siempre el ilustre maestro Alfredo Poviña, siendo Vicepresidente Herrera Figueroa y el suscrito Secretario General de la misma. Estaban allí sociólogos de distintas especialidades sociológicas: Juan Carlos Agulla, Ezequiel Anderegg, Francisco J. Andrés Mulet, Armando Asti Vera, Diego Balastro Reguera, Alberto Baldrich, Hugo Berlatzky, Juan J. Berruezo, Ernesto Eduardo Borga, Yolanda Bórquez, Luis Campoy Gainza, Julio César Castiglione, José Carlos Romero, Roberto Covián, Adolfo Critto, Fernando N. Cuevillas, Jaime Culleré, Camilo Dagúm, Juan Dalma, Carlos M. de Elía, Alberto Díaz Bialet, Antonio Donini, Jacobo Erlijman, Sara Faisal, Marta de Fernández, Floreal H. Forni, C. V. Gallino Yanzi, Jorge O. García Rúa, Regina Gibaja, M.E. Giménez de Lascano, Juan Ramón Guevara, José L. de Ímaz, Alejandro Jorge, Luis Lucena, Ítalo Argentino Luder, Fernando Martínez Paz, José Enrique Míguens, Argentino Moyano, Juan Pichon Riviere, Eduardo Raúl Piñero, Raúl Puigbo, Benjamín Rattembach, Horacio G. Rava, Manuel Ríos, María Ángela Roggero, Luis O. Roggi, Edgardo Rossi, Julio Soler Miralles, Francisco Suárez, Susana Tarleta, Luis Tribiño, Rodolfo Tecera del Franco, Abraham Váldez, Juan Villaverde, Milan Viscovich y muchos otros \parencite{1585-DAVID2000}.
\end{quote}

Aunque algunos participantes de esta lista (como Agulla, Critto, de Ímaz o Miguens) podrían reconocerse como autores que \emph{aggionaron} la forma de concebir, enseñar y practicar la sociología en Argentina, casi todos los demás miembros da cuenta de la pervivencia de la \enquote{sociología de cátedra}. Varios de ellos, impedidos de acceder a la centralidad que suponía formar parte de la carrera de sociología de la Universidad de Buenos Aires, como Cuevillas, Tecera del Franco o el propio Terrera, pudieron volver al dictado de clases en la capital argentina luego del golpe de estado de 1976. Más tarde, con el retorno de la democracia en 1983 comenzaría un proceso de \enquote{limpieza} de los docentes \enquote{cómplices} de la dictadura y el retorno de muchos exiliados \parencite{1551-BLOIS2009,1552-BLOIS2019}. Paulatinamente, los \enquote{sociólogos de frac} fueron quedaron en el olvido de la historia de la sociología vernácula, no solo porque empíricamente ellos y sus prácticas dejaron de estar presentes en las carreras de sociología, sino también por la aceptación generalizada de la perspectiva germaniana que indica el inicio de la sociología argentina con el nacimiento de la \enquote{sociología científica}.

Se aventura como hipótesis que en la actualidad siguen subsistiendo prácticas de antaño pero, al mismo tiempo, que se ven recluidas en espacios ajenos a los profesionales de la sociología, lo cual explicaría la poca atención prestada desde este campo de estudios. Si bien hoy en día preguntarse qué tanto de la \enquote{sociología de frac} pervive en la formación de abogados o economistas no se relaciona con la práctica profesional de la sociología, si se tiene en cuenta el volumen del estudiantado y el reclutamiento social de las facultades de Derecho y Ciencias Económicas, estudiar la sociología \enquote{para abogados} o \enquote{para economistas} puede llegar a ser relevante para entender dos cuestiones: por un lado, lo que podría denominarse la \enquote{vulgata sociológica} o la forma en la cual la sociología es conocida por un público amplio que excede a sus profesionales y, por otro lado, qué contenidos sociológicos se encuentran presentes en la formación intelectual de las élites del país. Una respuesta a ambas preguntas, por cierto, excedería por mucho estas reflexiones finales.

Anexo

Obras de sociología de Guillermo Alfredo Terrera hasta 1978.

\emph{Comentario político-social argentino}, Córdoba, 1945.

\emph{Socioantropología rioplatense}, Córdoba, 1949.

\emph{Sociología de la educación}, Córdoba, 1950.

\emph{Nacionalismo social argentino}, Córdoba, 1950.

\emph{Nuestra verdadera revolución}, Córdoba, 1950.

\emph{Realidad social argentina}, Córdoba, 1951.

\emph{Sociología general}, Córdoba, 1951.

\emph{La estirpe argentina}, Córdoba, 1951.

\emph{La honradez}, Córdoba, 1951.

\emph{Sociedad y conciencia social}, Córdoba, 1953.

\emph{La policía rural}, Córdoba, 1953.

\emph{Sociología del derecho}, Córdoba, 1953.

\emph{Sociología criminal}, Córdoba, 1954.

\emph{Apuntes de sociología}, Córdoba, 1954.

\emph{El exterminio de los gauchos}, Córdoba, 1954.

\emph{Genocidio y segregación aborigen}, Córdoba, 1960.

\emph{Restauración nacional y popular}, Córdoba, 1960.

\emph{Proclamas revolucionarias}, Córdoba, 1960.

\emph{Programa de gobierno}, Buenos Aires, 1964.

\emph{Ideario y programa de la Revolución Argentina}, Córdoba, 1963.

\emph{Manual de sociología}, Buenos Aires, 1966.

\emph{Sociología y vocabulario del habla argentina}, Buenos Aires, 1969.

\emph{Tratado teórico-práctico de sociología}, Buenos Aires, 1969.

\emph{La crisis social norteamericana. Ocaso de un liderazgo}, Buenos Aires, 1969.

\emph{Política Social. Con la estructura de la democracia funcional argentina}, Buenos Aires, 1970.

\emph{Proyecto completo con la nueva organización político-social de la República Argentina}, Buenos Aires, octubre de 1972 y noviembre de 1972.

\emph{El proceso de cambio en el grupo humano argentino}, Buenos Aires, 1973.

\emph{El hombre y la sociedad}, Buenos Aires, 1975.

\emph{La sociedad organizada}, Buenos Aires, 1976.


%Capítulo 3
\chapter{Juan Carlos Agulla: el nacimiento de la sociología empírica en la Universidad Nacional de Córdoba (1959-1966)}

\footnote{Publicado en \emph{Estudios Sociales} n.º 63 (2), e0036. \url{https://doi.org/10.14409/es.2022.2.e0036}. Juan Carlos Agulla: el nacimiento de la sociología empírica en la Universidad Nacional de Córdoba (1959-1966)}

\section{Introducción}

Un recorrido por la historia de la sociología en Córdoba incluiría algunas estaciones insoslayables, entre las que se cuentan destacados intelectuales como Enrique Martínez Paz, Raúl Orgaz, Alfredo Poviña o Francisco Delich, por solo nombrar algunos de los más relevantes. Entre ellos, Juan Carlos Agulla (1928-2003) ha sido probablemente de los menos visitados entre los estudiosos de la historia de la sociología local. Con excepción de unos pocos trabajos \parencite{1447-CARACCIOLO2010,1558-CHAMORROGRECA2007}(González, 2017; Grisendi, 2012, 2013) valiosos, pero con aproximaciones parciales, no cuenta con demasiadas referencias en el campo. No obstante, el volumen y la envergadura de sus escritos sociológicos lo convierten en un autor indispensable para entender el proceso de institucionalización de la sociología empírica en la universidad cordobesa.

En la actualidad, existen numerosas indagaciones en torno a la historia de la sociología local así como enfoques utilizados para abordarla. El propio \textcite{1631-AGULLA1984} no se privó de postular la existencia de una sucesión de etapas en la constitución de las tradiciones sociológicas en América Latina y Argentina.\footnote{En rigor, la reunión de sociólogos acontecida en 1965 \parencite{1632-AGULLA1966} fue su primera aproximación al tema cuando planteó la existencia de tres etapas del desarrollo en la sociología argentina: la sociología arcaica, la sociología residual y la sociología incipiente.} Originalmente formulada por Gino Germani (1964), esta visión rupturista respecto de un pasado precientífico y tradicional en la forma de concebir y practicar la disciplina, sin embargo, no fue unívoca. De hecho, se planteó como antagónica a la perspectiva de \textcite{1534-POVINA1959}, quien veía una continuidad entre la sociología enseñada en las cátedras de las facultades de derecho y aquella ejercida por los sociólogos profesionales.

Unos años más tarde, \textcite{1450-VERON1974} estudió el desarrollo de la sociología vernácula a partir de la simbiosis producida entre las clases sociales y la penetración del capital imperialista. De esta manera, las orientaciones epistemológicas de los sociólogos locales serían producto de las tensiones generadas por la política exterior de Estados Unidos en América Latina. En otro registro de análisis, Juan Marsal (1963) combinaba el estudio cronológico con el análisis de escuelas (marxista, católica, científica, etcétera) a las cuales adscribirían los sociólogos argentinos.

Aún reconociendo el valor de estos aportes, a los fines del presente trabajo resulta más adecuada la mirada de \textcite{1613-DELICH1977} quien, sin negar \enquote{etapas}, \enquote{escuelas} o \enquote{factores externos} que influyan sobre las matrices epistemológicas, afirmó la existencia de \enquote{estilos de trabajo} sociológicos que se superponen en el tiempo en las distintas universidades del país.\footnote{Así, este autor sostenía que en las universidades del interior del país casi la totalidad de las cátedras están \enquote{a cargo de personas que comparten la sociología de frac (\dots)} \parencite[34]{1613-DELICH1977}, es decir, de los sociólogos \enquote{tradicionales}.} Lo fructífero de esta mirada es que posibilita amalgamarla a las formas en las cuales actualmente los sociólogos escriben su historia. Luego de los aportes seminales de \textcite{1565-BLANCO2006} González Bollo (1999), Pereyra (2005), es posible estudiar en conjunto los procesos de institucionalización de la sociología, las prácticas de los sociólogos y su recepción de ideas a partir de la indagación empírica de fuentes institucionales y de sus publicaciones (artículos, libros, cartas, notas de clase, etcétera).

En este sentido, \textcite[31]{1510-SAPIRO2017} indica una serie de recaudos metodológicos sumamente pertinentes. Por un lado, señala que el mundo intelectual \enquote{no debe ser abordado como un espacio sin anclaje que solo existe en el universo de las ideas, sino como un universo social compuesto por agentes --individuos e instituciones-- que constituyen mediaciones susceptibles de análisis sociohistórico}. Pero, por otro lado, que el adjetivo intelectual remite precisamente a una cultura caracterizada por la palabra escrita. Por lo tanto, un estudio como el que aquí se propone supone tanto la reconstrucción de la sucesión de relaciones de interdependencia y de los espacios sociales de circulación de los agentes como de sus textos. A su vez, también aparecerá como un aspecto relevante la \enquote{circulación de personas} \parencite[35]{1510-SAPIRO2017} como modalidad central de la \enquote{transferencia cultural} que permitió la implantación de una nueva manera de hacer sociología en Córdoba.

Entonces, el objetivo del capítulo será la reconstrucción de la trayectoria social e intelectual de Juan Carlos Agulla, quien se convertirá en uno de los principales promotores de la sociología empírica en la Universidad Nacional de Córdoba durante la década de 1960. Por un lado, en su producción del período 1959-1966 se observará una actualización teórica a la par de lo que se enseñaba en Buenos Aires en la misma época, donde Gino Germani había instaurado la sociología científica. Por otro lado, en estos años desarrollarán las primeras indagaciones empíricas sobre la sociedad cordobesa desde el mundo académico.\footnote{Aquí la referencia central será la publicación conjunta de \textcite{1633-AGULLA1966}. En la misma época, Adolfo Critto realizó un estudio sobre el Barrio Maldonado. En ambos casos, la recolección y procesamiento de datos fue realizada en 1963.} En la constitución de esta forma de concebir y practicar la disciplina jugarán un rol especial la circulación internacional de ideas y personas. En primer lugar, porque la formación sociológica de Agulla tuvo lugar en España, Alemania, Chile y Estados Unidos. En segundo lugar, debido al convenio por el cual varios profesores de sociología de la Universidad de Indiana fueron a Córdoba a dictar clases y a participar de las investigaciones mencionadas.

A partir de estas premisas, el capítulo se divide en tres apartados. En el primero se reconstruye la trayectoria de Agulla, teniendo en cuenta sus orígenes sociales, espacios de socialización y formación sociológica hasta 1966. El segundo y tercer apartado analizan sus textos del período 1959-1966. El primero de ellos se ocupa de los trabajos dedicados a la teoría sociológica, donde se apreciará la importancia de la obra de Max Weber, Karl Mannheim y los funcionalistas. El último apartado retoma sus aproximaciones a la realidad social, las cuales servirán como antecedentes de la primera investigación empírica de largo aliento sobre la estructura del poder en la ciudad de Córdoba en 1966. Finalmente, las conclusiones sintetizan los elementos centrales del escrito.

\section{Contexto y trayectoria social de Juan Carlos Agulla}

Juan Carlos Agulla Granillo nació en la ciudad de Córdoba en 1928. Fue el mayor de seis hermanos de una de las familias de la aristocracia doctoral de la ciudad, ya que era hijo del otrora ministro y diputado Juan Carlos Agulla (p) y de Silvia Granillo Barros. Cursó estudios en el Colegio Monserrat e ingresó en la carrera de abogacía de la Facultad de Derecho y Ciencias Sociales de la Universidad Nacional de Córdoba (FDCS-UNC) en 1946, donde se recibiría tres años después con tan solo 21 de edad. En esa época, Agulla se reconocía como parte de la tradición reformista: \enquote{Yo era, entonces, un gran reformista por muchas razones, entre otras, porque mi familia hizo la Reforma en el 18 (\dots). Si Enrique Vargas y Florencio Valdés que son los que la hicieron, son tíos míos y mi padre estuvo ahí en la pelea} \parencite[276]{1626-FUCITO2004}.

Por este motivo, la política universitaria del primer peronismo (1946-1955), que \enquote{dejaba a un lado los principios reformistas que habían regido el funcionamiento de las casas de estudios desde 1918} \parencite[152]{1536-BUCHBINDER2010}, no le resultó tolerable en absoluto. En ese contexto hostil, se manifiesta una desviación del patrón de trayectoria que suele encontrarse entre los primeros profesores de sociología locales. Teniendo la posibilidad trabajar en los mejores estudios jurídicos de Córdoba y de contraer matrimonio con una miembro de la élite de la ciudad, Agulla decidió marcharse a Europa a estudiar filosofía luego de finalizada la carrera de grado.

Gracias a Pedro Ara, agregado cultural de la Embajada de España en Argentina, el Instituto de Cultura Hispánica le concedió una beca por un año para estudiar en Madrid. Apremiado por un clima incompatible con las libertades públicas ostensible en la Argentina de Perón, Agulla se embarcó hacia la España de Franco en 1950. Allí realizó cursos de doctorado en la Facultad de Derecho de la Universidad Central y en el Instituto de Estudios Políticos, donde se relacionó con destacados sociólogos españoles como Juan Linz, Salustiano del Campo y Juan Francisco Marsal, aunque fueron las clases de filosofía de Xavier Zubirí las que causaron un impacto mayor en el joven estudiante argentino. En este ámbito nació el interés de Agulla por el comportamiento humano, consagrando su primera tesis doctoral al pensamiento sociopolítico del siglo~XVIII, la cual fue dirigida por Enrique Gómez Arboleya, profesor de filosofía del derecho de la Universidad de Granada.

Era 1953, pero todavía no consideraba que fuera el momento de retornar a \enquote{la insoportable Argentina peronista}. Fue entonces cuando inició su viaje a Alemania, donde contaría con la ayuda de Michael Schmaus, profesor de teología de la Universidad de Munich, a quien había conocido en 1952 en los cursos de verano de la Universidad Internacional de Santander. Su beca ya había finalizado y, luego de vivir en condiciones muy modestas en España, este miembro de la élite cordobesa prefirió realizar trabajos manuales en un país donde ni siquiera hablaba el idioma a retornar a su patria.\footnote{En Alemania, \textcite[31]{1634-AGULLA1997} trabajó de \enquote{cargar y descargar camiones, encerar pisos, limpiar alfombras, pintar paredes, lavar copas y platos, limpiar cocinas y, sobre todo, ayudar en obras de construcción como armar andamios de madera o de hierro, poner tejas, acarrear ladrilloso cajones en el mercado, poner pisos, limpiar espacios, transportar baldes, hacer pozos, trasladar maquinarias y accesorios o limpiar máquinas (\dots)}.}

Entre los empleos que tuvo, el más relevante fue el de \enquote{extra} de cine, y no precisamente porque lo consagrara como \enquote{estrella} del rubro, sino porque fue donde conoció a su futura esposa y madre de sus hijos, Alexa Phalsbröker, una huérfana austríaca que trabajaba en la parte de liquidación de sueldos en la productora. Los años 1953 y 1954 fueron muy duros para Agulla, aunque comenzó a cursar en la facultad de filosofía cuando el trabajo se lo permitía. En 1955 realizó un curso con Alfred von Martin, profesor de historia de las ideas y sociología de la cultura, presentando un trabajo sobre Werner Sombart, y con José Ortega y Gasset, quien lo incentivó a leer a Georg Simmel y Max Weber. Al año siguiente, von Martin daría un seminario sobre el proceso de racionalización formal en Weber, que sería un insumo fundamental para su segunda tesis doctoral sobre la sociología del derecho weberiana, redactada en 1958, esta vez bajo dirección de Aloys Dempf.

Un componente central de este trabajo fue la edición de \emph{Economía y sociedad} publicada por el Fondo de Cultura Económica. Agulla escribiría a México solicitando los cuatro tomos de este texto cuya versión castellana estuvo a cargo del sociólogo español José Medina Echavarría, a quien conocería en Chile poco después. Al mismo tiempo, para el seminario de von Martin presentaría una monografía sobre Ralph Dahrendorf, quien articulaba el pensamiento de Karl Marx y los funcionalistas, especialmente Talcott Parsons, autor poco conocido en Alemania en ese momento.

En 1959 decidió volver a Argentina con su familia. Derrocado el peronismo, Agulla le escribió a Germani, entonces director de la carrera de sociología de la Universidad de Buenos Aires, a quien le envió su currículum vitae. A la negativa de este último de aceptarlo como profesor se adjuntó el consejo de \enquote{enseñar Historia de la sociología (según decían sus propias palabras) y en una Universidad del interior y, en especial, la de Rosario} \parencite[77]{1634-AGULLA1997}. Agulla escribió entonces a Poviña, profesor de la FDCS-UNC con quien Germani disputaba el liderazgo de la sociología argentina. La posibilidad que se le ofreció fue dictar una cátedra de sociología de la educación en la Facultad de Filosofía y Humanidades de la UNC.

Fue así que, en 1959 Agulla ganó por concurso la titularidad de la mencionada cátedra y también comenzó a dictar filosofía en el Colegio Monserrat. Sin embargo, el retorno a su ciudad natal duró poco tiempo debido a la obtención de una beca de la UNESCO para estudiar en la Facultad Latinoamericana de Ciencias Sociales (FLACSO) en Santiago de Chile. Esta experiencia fue, según el propio Agulla,

\begin{quote}
(\dots) fundamental en mi formación. Allí adquirí el conocimiento serio y sistemático de la sociología americana y de la investigación empírica (positivista) (\dots) Y por cierto los grandes nombres del funcionalismo pasaron a ser materia de estudio (Merton, Parsons, Homans, Lazarsfeld, Warner, Malinowski, Mead, etcétera), junto al análisis de las grandes investigaciones hechas por la nueva sociología americana tan criticada por Pitirim A. Sorokin y Carl [sic] Wright Mills y la introducción en la estadística \parencite[83]{1634-AGULLA1997}.
\end{quote}

Como ya se ha dicho, aquí conoció a Medina Echavarría, autor de \emph{Sociología: teoría y técnica} (1941), libro que Germani (1964:148) saludaría como el inicio de \enquote{la ola de la sociología científica en América Latina}. Entre Agulla y Medina Echavarría se observarán fuertes puntos en común. En particular, el hecho de que el último \enquote{se esforzó por elaborar un maridaje teórico entre las corrientes europeas, fundamentalmente la sociología weberiana, con los aportes de la sociología norteamericana, especialmente a partir de la lectura de Parsons} (Morales Martín, 2016:620). Las investigaciones empíricas de Medina Echavarría lo llevaron a afirmar la existencia de un rasgo típico de las sociedades latinoamericanas: la convivencia de tradición y modernidad, expresados en los conceptos de hacienda y empresa.\footnote{Como se verá más adelante, Medina Echevarría le permitió a Agulla pensar la presencia conjunta y superpuesta de estructuras \enquote{residuales} (de la sociedad tradicional) y \enquote{emergentes} (de la sociedad en transición) en sus trabajos empíricos.}

En su libro \emph{Consideraciones sociológicas sobre el desarrollo económico} (1964) aparecen claramente estas conceptualizaciones y han sido estudiadas como parte de la recepción de Weber en América Latina \parencite{1541-PEON1998}. Por cierto, en el caso de Germani, la construcción de la sociología científica también implicó una determinada lectura de Weber, es decir, aquella que lo extraía de la vieja disputa entre ciencias de la naturaleza y ciencias del espíritu a partir del análisis de la metodología de los tipos ideales \parencite{1566-BLANCO2007}. Entonces, en estos años la sociología latinoamericana no cambió de naturaleza, ya que nunca dejó de ser una empresa de importación, aunque sí se modificó en términos de qué lectura se hacía de los autores receptados y para qué se los utilizaba.

En el caso de Agulla, sin embargo, esto deberá esperar un poco más. Luego de la experiencia trasandina, retornó a Córdoba y redactó su tercera y última tesis doctoral titulada \emph{El descubrimiento de la realidad social: introducción a Comte} (1962b). Agulla encontró en Córdoba un panorama que auguraba la posibilidad de una renovación sociológica. Si bien es cierto que la llamada \enquote{sociología de cátedra} seguía siendo fuerte en las universidades del interior del país, y que Poviña era su líder indiscutido luego de haber fundado la Asociación Latinoamericana de Sociología en 1950 (ALAS) y la Sociedad Argentina de Sociología (SAS) en 1959, además de que presidiría el Instituto Internacional de Sociología (IIS) en 1963 \parencite{1615-DIAZ2013}, no es menos cierto que Agulla contaba con jóvenes compañeros interesados en actualizar la sociología cordobesa como Eva Chamorro o Adolfo Critto, quien se había doctorado en la Universidad de Columbia bajo supervisión de Robert Merton (Grisendi, 2014).

A su vez, a instancias de la Fundación Fulbright, se firmó un convenio que permitió que fueran al Instituto de Sociología \enquote{Raúl Orgaz}, perteneciente a la FDCS-UNC, tres profesores de la Universidad de Indiana durante tres años seguidos. Fue así que Melvin De Fleur realizó un estudio sobre la delincuencia, Michael Myren otro sobre la policía, y Delbert C. Miller, junto a Agulla y Chamorro, llevarían a cabo una pionera investigación sobre la \enquote{estructura del poder de la ciudad de Córdoba}, lo cual abriría una línea de trabajo que sería continuada hacia 1968 con la publicación del famoso texto sobre el eclipse de la aristocracia cordobesa (González, 2017).

Esto no implica soslayar la disputa real entre sociólogos \enquote{de cátedra} y \enquote{científicos} \parencite{1567-BLANCO2004}. La misma se manifestaba aún en espacios donde se abordaban temáticas afines a la sociología de la modernización como el XX Congreso Mundial de Sociología (1963) celebrado en Córdoba, del cual no participó el grupo de Buenos Aires. Aquí el tema era \enquote{La sociología y las sociedades en desarrollo industrial}, lo cual estaba en línea con el desarrollismo de la CEPAL y de la Alianza para el Progreso. Pero, precisamente por esta razón, además de otros signos de \emph{aggiornamiento} de la sociología cordobesa, como la aparición de los \emph{Cuadernos del Instituto}, es decir, de una publicación especializada en la materia, que indica un mayor grado de institucionalización \parencite{1512-SHILS1971}, es que no debe entenderse esta disputa en términos tan esquemáticos.

En este sentido, la trayectoria de Agulla resulta más que elocuente. En 1964, además de asistir al congreso que se celebró en Alemania por motivo del centenario del nacimiento de Weber \parencite{1635-AGULLA1964}, fue invitado por la Universidad de Muenster/Westfalia a dictar un curso sobre \enquote{Sociología del Desarrollo}. A su vez, luego de ganar la beca Guggenheim, tuvo oportunidad de realizar una estadía de estudios en Estados Unidos en el Institute of Latin American Studies durante 1965, donde fue designado como Visiting Scholar sobre temas de sociología de la educación. En esta estadía en América del Norte logró entablar fructíferas conversaciones con Merton, Parsons y Seymour Lipset, entre otros \parencite{1634-AGULLA1997}.

Por lo tanto, al momento de publicar \emph{De la industria al poder} (1966), Agulla contaba con una sólida y actualizada formación en sociología, producto de sus contactos y estancias de estudios en el exterior. Su trayectoria no sería entonces comparable con la imagen tradicional que se ha construido en torno a los \enquote{sociólogos de cátedra}, es decir, abogados dedicados a otros menesteres que, de manera complementaria, enseñaban sociología de forma enciclopédica, manteniéndose ajenos a la investigación empírica.\footnote{Por cierto, habría que decir que si bien Agulla tenía título de abogado nunca ejerció la profesión.} Esto se evidencia en que la primera opción de Agulla para insertarse en las redes de la sociología local fuera el grupo de Buenos Aires. El hecho de que luego se haya vinculado a la SAS no lo vuelve necesariamente un \enquote{sociólogo tradicional}. Por el contrario, como intentará mostrarse en los próximos apartados, su sociología no solo muestra una apropiación de los autores clásicos diferente de quienes lo precedieron en las cátedras universitarias, sino también una inclinación hacia la investigación empírica en una institución que no estaba orientada en ese sentido antes de su llegada.

\section{El sociólogo habla de la sociología}

Antes de regresar a Argentina Agulla había publicado algunos ensayos de crítica literaria durante su estadía en España y, por cierto, sus primeras tesis doctorales podrían considerarse trabajos enmarcados en la sociología. Sin embargo, aquí se abordará su producción a partir del retorno de 1959. En ese momento publicó su primer artículo en la \emph{Revista de Humanidades} de la UNC dedicado a \enquote{la situación histórica de Comte y Max Weber} \parencite{1636-AGULLA1959}. Este breve texto, aunque no tiene mayor relevancia para su producción posterior, se centra en el clima de ideas dominante de los momentos en que ambos autores desarrollaron sus obras, lo cual revela la temprana inclinación de Agulla por las teorías sociológicas.

Si se tiene en cuenta que una de las máximas de Miguel de Unamuno que Agulla se apropió fue la de \enquote{para novedad, los clásicos}, no es extraño que varios de sus textos más importantes estuvieran apoyados en autores como Weber, Mannheim, Parsons o Merton. Como se verá, entre las obras más visitadas la principal fue la de Weber, aunque varias ideas de Mannheim y de los funcionalistas también estuvieron presentes en la etapa aquí analizada. Por el contrario, el pensamiento francés, más allá de que su última tesis doctoral estuvo dedicada a Comte, no le resultó demasiado atractivo.\footnote{En rigor, esta tesis no se ocupa de Comte sino de los pensadores sociales de los siglos XVII y XVIII (Hobbes, Locke, Rousseau, Montesquieu, etcétera) que crearon las condiciones para la aparición del pensamiento sociológico en el siglo~XIX \parencite{1637-AGULLA1962}.} Es más, puede afirmarse que con excepción del trabajo sobre Ortega y Gasset que se comentará a continuación las referencias a Durkheim, sin ser muy abundantes, tendieron a manifestarse en tono crítico.\footnote{Por ejemplo, Agulla criticó su \enquote{sociologismo de la educación}, que hacía que su aporte quedara \enquote{en sus últimas consecuencias, nuevamente frustrado} \parencite[60]{1638-AGULLA1964}.} Sin embargo, Agulla aceptaba algunas de sus tesis centrales, como aquella que postula la creciente división del trabajo social y la especialización de los individuos como tendencias propias de las sociedades complejas así como el carácter exterior y coactivo de los hechos sociales.

Agulla dio a conocer dos libros en 1962. El primero fue \emph{Contribución de Ortega a la teoría sociológica} \parencite{1639-AGULLA1962}, publicado por la UNC. En este ensayo analiza dos obras del filósofo español: \emph{El hombre y la gente} (1957) y \emph{España invertebrada} (1922). El trabajo se orienta a descifrar qué aporte puede hacer la filosofía orteguiana a la teoría sociológica y, en particular, si sus presupuestos filosóficos pueden servir de sustrato a una teoría de la acción social.

En su reconstrucción del primer texto, Agulla arriba a la definición orteguiana de los \enquote{usos}, que poseen un carácter extra-individual, coactivo e irracional. Esto es, a las dos características que Durkheim atribuye al hecho social, Ortega le agrega el carácter \enquote{irracional}. Los \enquote{usos} son definidos como una realidad \emph{sui generis}, por lo que \enquote{no son, en ningún caso, de los individuos, sino de la sociedad} \parencite[34]{1639-AGULLA1962}. La labor de un sociólogo consiste en indagar las condiciones a partir de las cuales algo adquiere vigencia social, es decir, \enquote{cómo se constituye y mantiene un uso} \parencite[42]{1639-AGULLA1962}. Por lo tanto, la contribución de Ortega a la teoría sociológica es brindar un instrumental teórico que permita conocer la realidad social, describirla y explicarla.

En el segundo texto, Agulla identifica una \enquote{faceta sociológica} en el planteo de Ortega, que es la vigencia de la dicotomía fundamental de toda sociedad: aristocracia -- masas. En este sentido, las sociedades perviven mientras funcione el mecanismo \enquote{ejemplaridad-docilidad}, equivalente al \enquote{liderazgo carismático} weberiano. En caso contrario, es decir, cuando los elementos que componen la sociedad no son funcionales, la sociedad tiende a disgregarse. Por ello, en el proceso de integración de una sociedad Ortega \enquote{sigue la ley de la división del trabajo de Durkheim, ya que en la medida en que la sociedad se organiza y se constituye, surgen, a la par, diferenciaciones en las clases, en los grupos sociales, profesionales, gremios, etcétera} \parencite[61-62]{1639-AGULLA1962}. Una sociedad sana es, pues, una sociedad integrada.

El segundo libro publicado en 1962 fue \emph{Estructura y Función}, donde Agulla analiza los alcances y límites de la teoría funcionalista, teniendo en cuenta las posibilidades de su aplicación práctica para el conocimiento de la realidad social. El libro en cuestión cuenta con cuatro capítulos que podrían resumirse en algunas ideas centrales. En primer lugar, que el intento crear una sistemática de la teoría de Parsons supone aceptar dos cuestiones: que la sociología es un conocimiento científico y que su objeto es un \emph{positum}, es decir, que es una ciencia positiva (o empírica). En segundo lugar, que existe una discrepancia fundamental entre Parsons y Merton en torno a la madurez científica alcanzada por la sociología. Mientras para el primero ya se ha logrado el grado de desarrollo necesario que permite construir una teoría sistemática, para el segundo la inmadurez de la sociología lo lleva a postular las \enquote{teorías de alcance medio} como único camino a la sistematización. Por ello

\begin{quote}
(\dots) mientras para Parsons, todas las investigaciones empíricas deben estar orientadas y ser extraídas de un sistema lógicamente cerrado (\dots), para Merton este sistema es abierto en el sentido que es solo una suma de sistemas de hipótesis que abarcan parcelas de la realidad social que han sido investigadas empíricamente. Lógicamente es más correcta la posición de Parsons, pero también es una forma que no se da en ninguna ciencia empírica (\dots), razón por la cual prácticamente es más correcta la posición de Merton (\dots) \parencite[57]{1640-AGULLA1962}.
\end{quote}

En tercer lugar, Parsons establece que la sociología es una ciencia de síntesis que debe resumir en una teoría general los aportes de las distintas ciencias sociales, a partir de la idea de la acción humana como \enquote{acto único}. Ahora bien, entre los subsistemas que Parsons postula, es decir, el de la personalidad, el cultural y el social, Agulla advierte que el último contiene una determinada estratificación y división del trabajo entre sus miembros, lo cual se conecta con los conceptos de \enquote{roles} y estatus. Estas categorías del estructural-funcionalismo se enlazan con otras de carácter estructural estático (estructura) y dinámico (función) para el estudio del sistema social.\footnote{Una versión más refinada en torno a las relaciones entre roles, estatus, estructuras y funciones sociales se encuentra en \textcite{1635-AGULLA1964}.}

Hete aquí entonces \enquote{el carácter conservador} \parencite[109]{1640-AGULLA1962} que presenta esta tradición, ya que el análisis del cambio social termina por reducirse al ajuste mutuo entre los elementos que componen el todo societario. Este hecho, advertido por Merton, lo hizo introducir categorías como la \enquote{disfunción} o las \enquote{funciones manifiestas y latentes}. Estas últimas entrarán eventualmente en conflicto, situación denominada \enquote{tensión} por parte de los teóricos analizados. Por lo tanto \enquote{el análisis del cambio, para la teoría estructural-funcionalista, implica el examen de las circunstancias o factores que tienden a alterar el equilibrio relativo de una sociedad} \parencite[131]{1640-AGULLA1962}.

En definitiva, la crítica de Agulla se centra en el \enquote{talón de Aquiles} del funcionalismo, es decir, en la ausencia de categorías que den respuesta a los problemas del cambio social que, por cierto, deberían ser probadas empíricamente. La solución a este problema del funcionalismo supondría \enquote{tomar la vía de Parsons y no de Merton} \parencite[158]{1640-AGULLA1962}. Del hallazgo de estas categorías en este campo dependería el futuro de la teoría estructural-funcionalista como un intento de sociología sistemática.

Más allá de esta publicación, Parsons y Merton (además de Edward Shils, Alex Inkeles y otros miembros del funcionalismo) fueron autores que contaron con varias referencias en los textos de Agulla. En el caso de Parsons, fue mencionado en un trabajo de 1960 que, bajo el título \enquote{La educación en la sociedad de masas}, se publicó en 1965. Allí se lo citaba en relación a la capacidad de \enquote{adaptabilidad} del individuo a las \enquote{situaciones de masa}, las cuales cumplirían una \enquote{función socializadora}. En cuanto a Merton, es nombrado en un escrito sobre \enquote{marginalidad y participación}, donde Agulla señaló al \enquote{comportamiento desviado} como una \enquote{forma patológica} de participación en las sociedades modernas \parencite{1641-AGULLA1965}.\footnote{Varias de las referencias que se utilizarán a continuación corresponden a la compilación \emph{Razón y Sociedad} (1965), la cual incluye textos de todo el lustro 1960-1965.}

Esta misma idea se reitera en \enquote{Sistema educativo y clases sociales} en referencia a quienes no creen en los medios institucionalizados \enquote{para alcanzar [las] metas u objetivos culturales que propone el sistema} \parencite[33]{1642-AGULLA1964}, tema sobre el que se volverá en el próximo apartado. Sin embargo, muchas de las alusiones a estos autores fueron implícitas. Agulla se refirió así, por ejemplo, al carácter \enquote{funcional} de las instituciones o de las estructuras sociales, a las \enquote{relaciones funcionales} entre hombre y sociedad, a la necesidad de realizar \enquote{análisis funcionales}, etcétera.

Por ello, conviene observar en qué consistió la apropiación que realizó de Weber. Como ya se ha dicho, la sociología de weberiana impregna buena parte de los textos agullianos del primer lustro de los años sesenta, algunas veces con referencias explícitas, otras implícitas, en muchas oportunidades con un protagonismo estelar y en otras entremezclado con varios autores. Ejemplo de este último caso es \enquote{La sociología alemana contemporánea}, donde aparece mencionado entre muchos otros como Töennies, Simmel, Michels, Mannheim, Sombart, etcétera, a quienes, sin embargo, no se los presenta en términos tan elogiosos como a \enquote{el gran Max Weber} \parencite[90]{1643-AGULLA1963}.

Lo cierto es que la tesis weberiana central que se reitera en los textos de Agulla es aquella que postula la progresiva racionalización de las esferas de acción de la vida de los hombres en las sociedades occidentales. Esta idea ya aparece en el informe técnico que escribió en 1960 sobre el \enquote{Primer Seminario Argentino de Sociología}, organizado por la SAS \parencite{1641-AGULLA1965}. También es mencionada en el informe sobre el congreso en homenaje por los cien años del nacimiento de Weber, donde Agulla daría cuenta de la postura de Ernst Topitsch, profesor de la Universidad de Heidelberg, quien enfatizaba que el presupuesto central de la posición weberiana en torno la cosmovisión del hombre moderno se encuentra \enquote{en la misma sociedad y no [es] otro que el proceso de racionalización de la revolución científico-industrial de la sociedad occidental} (Agulla, 1964a:4).\footnote{Es verdad que este informe reconstruye otras exposiciones como las de Parsons, Aron, Marcuse, Adorno, etcétera, pero Topitsch fue el anfitrión de las jornadas y tuvo a su cargo la disertación introductoria sobre la temática weberiana.}

Dos años antes, Agulla había dicho lo mismo en \enquote{Las humanidades en las sociedades en desarrollo}, donde postulaba que el proceso de racionalización del conocimiento en ciencias sociales, es decir, su adopción del método científico iniciado por Galileo, conducía al \enquote{desencantamiento del mundo}. Aquí se manifiesta una \enquote{correlación del proceso de racionalización del conocimiento y del proceso de racionalización de la realidad social} (Agulla, 1965a:67, énfasis en original). Con un título más que elocuente, \enquote{El proceso de racionalización en la sociedad contemporánea}, de 1964, resaltaba los \enquote{análisis históricos hechos por Werner Sombart y sobre todo por Max Weber sobre la formación del capitalismo moderno (\dots)} (Agulla, 1965a:16-17). Entre las consecuencias de dicho proceso señalaba la \enquote{burocratización}, entendiendo por tal \enquote{la profesionalización y tecnificación de la actividad administrativa y ejecutiva del Estado} (Agulla, 1965a:20).

Sin embargo, este proceso no solo afecta a quienes trabajan en la burocracia estatal. Por el contrario, Agulla refirió al caso de los médicos, donde se manifiesta una evolución de los conocimientos científicos \enquote{constituyendo un proceso \enquote{típico} de racionalización, tal como se presenta en Occidente, [que] lleva la impronta de una racionalización formal de todos los conocimientos como una forma de ir \enquote{desencantando el mundo} (\dots)} (Agulla, 1965a:27). Al igual que el resto de los profesionales, en las sociedades modernas \enquote{el ejercicio de la profesión médica ha de hacerse en forma burocrática} (Agulla, 1965a:34). Es más, esta tendencia de la sociedad \enquote{a organizarse sobre bases racionales} (Agulla, 1965a:42) afecta incluso a instituciones como el matrimonio y el divorcio, temas a los cuales Agulla también prestó atención.

Ahora bien, si la racionalización fue el concepto weberiano más importante para Agulla, no fue el único que sirvió a sus indagaciones. Así, por ejemplo, en el XIII Congreso Nacional de Sociología (1962) celebrado en Sonora, México, Agulla (1965a:85) indicó que, para estudiar las sociedades en desarrollo industrial era muy pertinente revisar \enquote{la vieja metodología weberiana de los \emph{tipos ideales}}. A su vez, sus reflexiones en torno a la teoría del comportamiento humano (Agulla, 1965b; 1966b) están fuertemente apoyadas en \emph{Economía y sociedad}. La definición de sociología como la ciencia que aspira a entender, interpretándola, a la acción social para explicarla causalmente en su desarrollo y efectos, así como la idea del sentido subjetivo mentado por el agente, son el punto de partida para abordar los supuestos antropológicos de esa acción.

Como se ha dicho, además de Weber y los funcionalistas hubo otro autor que le interesó especialmente: Karl Mannheim. De esta manera, las ideas de \enquote{democratización fundamental} (Agulla, 1963a:766; 1965a:18, 88; 1966b:37), es decir, de una mayor participación en la vida política de la sociedad en la modernidad, y de \enquote{planificación democrática} (Agulla, 1965a:19), o sea, la forma en que deben tomarse decisiones desde posiciones clave de los gobiernos, son también entendidas como tendencias necesarias en esta época.

Finalmente, cabe mencionar el libro \emph{Teoría sociológica} (Agulla, 1964d), que recopila varias de las ideas ya tratadas. El primer capítulo se compone de tres partes dedicadas al pensamiento histórico-filosófico de los siglos XVII y XVIII, haciendo hincapié en Hobbes y Rousseau.\footnote{Temas que forman parte de la tesis doctoral sobre Comte (Agulla, 1962b).} El segundo capítulo contiene el artículo sobre Comte y Weber (Agulla, 1959), una parte de la tercera tesis doctoral titulada \enquote{Comte y la Realidad Social} y un texto sobre \enquote{Max Weber y el Orden Jurídico}.\footnote{En síntesis, Agulla (1964d:87) plantea que el moderno Derecho europeo-occidental atraviesa una evolución histórica por la cual \enquote{poco a poco, las formas jurídicas irracionales dejan lugar a las formas jurídicas racionales, debido a la presión de factores extrajurídicos o sociales}. Queda claro que este análisis se subsume bajo la interpretación general que Agulla realiza de Weber, ligada a la racionalización de las sociedades occidentales.} Finalmente, en la última parte se vuelve a publicar el texto sobre Ortega y Gasset (Agulla, 1962a) y una breve reflexión sobre la teoría funcionalista contenida en \emph{Estructura y función} (Agulla, 1962b), los cuales ya han sido comentados.

En síntesis, los textos teóricos de Agulla de comienzos de la década de 1960 combinan en grados distintos autores clásicos como Weber y Mannheim y otros de relevancia contemporánea para la época como Parsons y Merton. Con una primacía de la sociología alemana, se observa un peso significativo de la perspectiva weberiana combinada con la búsqueda de las funciones cumplidas por cada una de las esferas de acción estudiadas. A su vez, las referencias a Mannheim ingresan de forma complementaria a los postulados generales de Weber. Como se verá a continuación, en sus indagaciones empíricas cobrarán más relevancia otros sociólogos como el mencionado Medina Echavarría o Charles Wright Mills.

\section{El sociólogo habla de la sociedad}

Los primeros trabajos empíricos de Agulla estuvieron enfocados en el proceso de transición a la modernidad de la ciudad de Córdoba, siendo buena parte posteriormente utilizada para redactar De la industria al poder (Agulla, Chamorro, Miller, 1966).\footnote{Entre ellos, \enquote{Aspectos sociales del proceso de industrialización en una comunidad urbana} (Agulla, 1963a) es una versión abreviada de la primera parte del libro, por lo que no tiene sentido analizarlo aquí.} Así, temas como la organización familiar, la estratificación social, la industrialización, el sistema educativo y la formación de nuevas clases sociales son aspectos de dicho proceso de modernización que estarán presente en estos trabajos.

El primero de estos artículos, escrito en 1960, indaga en la \enquote{familia argentina} en un momento de transición del país que se debe \enquote{fundamentalmente, al desarrollo económico} (Agulla, 1965a:129). Por esto, es posible que las estructuras económicas, sociales o políticas no coincidan con las reclamadas por el proceso, lo que genera \enquote{estrangulamientos al desarrollo} en tanto tienden a convivir distintos tipos de familia. En la Córdoba \enquote{tradicional}, las \enquote{viejas} clases altas tenían estructuras familiares rígidas y mantenían el control de los poderes político, económico, cultural y social, presentándose \enquote{como \enquote{el} grupo de referencia en todos sus comportamientos} (Agulla, 1965a:137). Por ello, la familia campesina también mantiene aspectos tradicionales, mostrando así el campo un rezago respecto del mundo urbano.

Frente a estos factores residuales que resisten al desarrollo económico, Agulla indica que el tipo familiar moderno que se está conformado en las ciudades argentinas (Buenos Aires, Córdoba y Santa Fe) depende de tres factores: a) el proceso de urbanización; b) el proceso de industrialización; c) la formación de las clases medias, como conjunción de los otros dos factores. En este contexto se crean nuevos marcos de referencia y se rompen los controles externos, producto de la inmigración de personas jóvenes solteras, al tiempo que las mujeres adquieren un mayor status social.

En línea con estas apreciaciones, el año siguiente Agulla presentó un texto sobre la \enquote{Estratificación social en Argentina} en el Primer Congreso Argentino de Sociología organizado por la SAS en Mendoza. El artículo plantea nuevamente el estado de transición de Argentina y sostiene como \enquote{hipótesis general} que en el país existen \enquote{conjunta y superpuestamente, distintos sistemas de estratificación social} (Agulla, 1965a:99), es decir, formas residuales, en transición y emergentes, todas ellas afectadas por la inmigración, la urbanización y el desarrollo económico. Aquí aparece el esquema weberiano que destaca la \enquote{afinidad electiva} entre estructuras económicas y estructuras sociales o económicamente relevantes. Para Agulla,

\begin{quote}
el problema se centra (\dots) en ver el grado y forma del ajuste o adaptación de esas zonas\footnote{Se refiere a las regiones del país que se encuentran más o menos desarrolladas.} y de los sistemas de estratificación al desarrollo alcanzado en otras zonas y en los sistemas de estratificación social; en ver también el grado y la forma en que se resisten o se excluyen recíprocamente; en ver, por último, cuales son los factores de incentivación y los factores de resistencia en las distintas zonas y en las distintas estructuras al desarrollo económico. La riqueza que ofrece el esquema weberiano salta a la vista (Agulla, 1965a:101).
\end{quote}

También en \enquote{Cambio social y estructura de clases en Argentina}, de 1965, se reitera una vez más la idea de una sociedad en transición donde coexisten formas de estratificación residuales y emergentes. En la \enquote{estructura rural} se manifiesta la forma residual, que se caracteriza por la poca movilidad social de tipo vertical, que resulta un \enquote{factor de expulsión} de población del campo a la ciudad. A su vez, el hecho de que el sector oligárquico terrateniente sea un grupo dominante cerrado hace que el campo esté \enquote{marginado de los procesos de modernización (tanto de la cultura material como de la cultura inmaterial) (\dots)} (Agulla, 1965a:118).

Por el contrario, la estratificación que se presenta en la \enquote{estructura urbana argentina} es de aparición reciente como consecuencia del proceso de industrialización y burocratización. Esto hizo que emerja una clase alta estrechamente vinculada a la industria, muy diferente de las \enquote{viejas} clases altas ligadas al control del sector primario. Este nuevo grupo también se compone de altos funcionarios de las empresas y de la burocracia estatal. Por su parte, las clases medias están constituidas por propietarios de industrias y comercios menores, profesionales, técnicos y altos funcionarios del Estado. Finalmente, la clase baja urbana está compuesta de obreros industriales calificados y no calificados, además del \enquote{lumpemproletariado}. En síntesis, concluye Agulla, en Argentina conviven dos sistemas de estratificación, uno rural (o residual) y otro urbano (o emergente).

Por otro lado, es evidente que el comportamiento de las clases sociales debe mostrar diferencias en las distintas esferas de acción en que se desenvuelven. Entre ellas, Agulla se ocupó de estudiar la deserción escolar en el nivel primario en la ciudad de Córdoba, intentando averiguar cuáles son las causas socio-culturales que la determinan. A partir del material empírico recogido en distintas dependencias estatales, la hipótesis que elabora explica la deserción por un conflicto de valores entre el sistema educativo y algunas clases sociales. Es decir, se supone que ante la alta deserción de los hijos de las clases bajas y la disminución de ese número entre los hijos de la clase media, \enquote{los padres de clase media iban a responder afirmativamente a los valores pertenecientes al sistema educativo y que los padres de las clases bajas iban a responder negativamente a esos valores} (Agulla, 1964c:34-35). De aquí la utilización del concepto de \enquote{conducta desviada} propuesto por Merton que se ha comentado en el apartado anterior.

Todas estas preocupaciones que se hacen palpables en las primeras aproximaciones empíricas aparecerán con mayor fuerza en 1966. Como ya se dijo, \emph{De la industria al poder}, escrito en coautoría con Chamorro y Miller, es el texto central de estos años. El mismo está constituido por dos partes. La primera de ellas es teórica y busca dar un marco general para la comprensión del proceso de industrialización ocurrido en Córdoba en el período 1948-1960 y su impacto en la estructura social. La segunda parte estudia empíricamente la estructura de poder de la misma ciudad, es decir, la constitución de los grupos sociales que participan del proceso de toma de decisiones. El segmento teórico constituye una descripción de la estructura social de la ciudad de Córdoba que, aunque sin el mismo vuelo literario, recuerda por pasajes al famoso ensayo \emph{Buenos Aires. Vida cotidiana y alienación} (1964) que en esta misma época convirtió a Juan José Sebreli en un intelectual de relieve en el espacio público argentino. Aquí se enumeran una serie de características de esta ciudad como \enquote{comunidad tradicional} previa a 1948, así como los cambios que trajo aparejado el proceso de industrialización y urbanización iniciado ese año.

De forma sintética, pueden mencionarse las particularidades de las estructuras seleccionadas: a) la estructura ecológico-demográfica se caracteriza por su crecimiento vegetativo, con contribuciones pequeñas de la inmigración, tanto interna como externa; b) la estructura económico-ocupacional es eminentemente agropecuaria, comercial y burocrática; c) la estructura político-jurídica es de carácter \enquote{patriarcalista} (caudillismo) y liberal; d) la estructura sociocultural se caracteriza por sostener un sistema de valores heredado de la Colonia; e) finalmente, la estructura doméstica (familiar) también posee un carácter \enquote{tradicional}, es decir, que el poder y la autoridad son ejercidos por el \emph{pater} familia, mientras que la mujer ocupa un lugar subordinado y limitado a la esfera privada.

La alta inversión en la industria entre 1948 y 1960 generó cambios en el conjunto de estas estructuras, las cuales empezaron a entrar en conflicto entre sí, es decir, a \enquote{no integrarse funcionalmente}. \enquote{Las formas \enquote{emergentes} tienden a estructurar una nueva situación en razón de la racionalización y planificación de la vida económica y, sobre todo, de la vida industrial} (Agulla, Chamorro y Miller, 1966:32). La estructura ocupacional se modifica y, con ella, el prestigio de las profesiones \enquote{residuales} en relación a las \enquote{emergentes}. Esto a su vez genera un cambio en la estructura demográfica, producto de las migraciones hacia la ciudad en busca de los nuevos empleos. La estructura política también sufre modificaciones a partir del proceso de \enquote{democratización fundamental}, que supone sobre todo una mayor participación en la toma de decisiones de dos grupos: \enquote{el sector obrero y el sector femenino} (Agulla, Chamorro y Miller, 1966:40).

En esta \enquote{transición} puede observarse tanto una \enquote{confusión ideológica} en el mundo político como el surgimiento de una nueva \enquote{mentalidad racional} en el mundo cultural. Así, el viejo \enquote{doctor} de la sociedad cordobesa va cediendo su lugar a los técnicos y científicos de la sociedad moderna, en línea con la aparición de \enquote{nuevas clases sociales} (altas, medias y bajas) y de modificaciones en la estructura familiar, que disminuye y adquiere un mayor grado de racionalización y planificación. De esta manera, estructuras tradicionales, en transición y modernas conviven en conjunto y superpuestamente.

En la segunda parte se estudia cómo se forma y elabora la toma de decisiones en la comunidad cordobesa. El diseño de recolección y análisis de datos utilizado es el mismo que se aplicó en ciudades de Estados Unidos (Seattle y Atlanta) e Inglaterra (Bristol), por lo que se trata de un trabajo comparativo donde las diferencias de Córdoba mostrarían, según la hipótesis operativa planteada, que los sectores \enquote{religioso}, \enquote{militar} y \enquote{universitario} dominan \enquote{el vértice de la pirámide estructural del poder de la comunidad y, por lo tanto, habrían de serlos \enquote{focos} o \enquote{fuentes} principales en la formación y elaboración de las decisiones que se tomaban en la comunidad} (Agulla, Chamorro y Miller,1966:66).

Sin embargo, el análisis empírico termina por demostrar que ninguno de estos segmentos domina en la estructura del poder de la ciudad de Córdoba.\footnote{En las conclusiones se admite que la hipótesis operativa \enquote{queda negada en muchos aspectos}, aunque también se señala que \enquote{la influencia de los sectores \enquote{religión} y \enquote{militar} se pone de manifiesto de muchas maneras y a través de diversos canales} (Agulla, Chamorro y Miller, 1966:136).} El estado de \enquote{transición} de la ciudad genera que su estructura de poder sufra un lento proceso de ajuste a los requisitos funcionales de la industrialización, por lo que se pueden observar conjunta y superpuestamente componentes \enquote{residuales} y \enquote{emergentes} de dicha estructura. Al momento en que Agulla escribe, \enquote{no existe en la comunidad una élite de poder} ya que \enquote{la estructura del poder no está controlada por un grupo de individuos que ejercen el poder en forma \enquote{normada} para el logro de algún objetivo, consciente o inconscientemente definido} (Agulla, Chamorro y Miller, 1966:141).

En síntesis, la estructura del poder de la ciudad de Córdoba se presenta como conflictiva y no integrada, \enquote{pero se advierte la tendencia: de la industria al poder} (Agulla, Chamorro y Miller, 1966:93). Para que emerja la élite es necesario que la industrialización supere el estado de transición en que se encuentra. Como ya se ha dicho, estas reflexiones se prolongarán en los años inmediatamente posteriores dando lugar al conocido texto \emph{Eclipse de una aristocracia} (1968). Inserto en una discusión candente donde \emph{La élite del poder} (1956) de Charles Wright Mills, y \emph{Los que mandan} (1964) de José Luis de Ímaz son referencias insoslayables, Agulla profundizará en el análisis del comportamiento de las élites dirigentes, dando así nacimiento a la sociología empírica en la Universidad Nacional de Córdoba.

\section{Conclusiones}

La obra temprana de Juan Carlos Agulla muestra algunas dimensiones que lo alejan del camino tradicional recorrido por los primeros intelectuales cordobeses dedicados a la enseñanza de la sociología. Producto de su formación en el exterior y del contacto con representantes trascendentes para la historia mundial y regional de la disciplina (Parsons, Merton, Medina Echavarría, entre otros), se aprecia una sociología en sintonía con las matrices epistemológicas de comienzos de la década de 1960. Su obra abrevó con intensidad en la tradición alemana, donde Weber y Mannheim aparecen no como los únicos, pero sí como los más relevantes de sus intereses. Como se ha demostrado, los procesos de racionalización, burocratización, democratización y planificación son el telón de fondo de la mayor parte de sus reflexiones teóricas.

Al mismo tiempo, la hegemonía del estructural-funcionalismo norteamericano en el período de posguerra es palpable en la perspectiva de Agulla, lo cual lo convierte en un legítimo hijo de su época. De esta manera, problematizó aspectos centrales de los postulados teóricos de Parsons y Merton en torno a la madurez de la sociología, tomando partido por el primero en cuanto al camino que debía seguir la disciplina si tenía aspiraciones de formular una teoría sistemática. A su vez, adoptó el punto de vista según la cual las distintas esferas de acción cumplen funciones específicas en las sociedades complejas, debiendo conformar una totalidad integrada funcionalmente. Estas observaciones se hacen ostensibles tanto en el plano teórico como en las indagaciones empíricas, en las cuales aparecen bien ponderados los nombres de Medina Echavarría y Wright Mills. En ambos casos se trata de autores que inspiraron a Agulla tanto en la selección de problemas como en el tratamiento que debía dársele a los mismos.

Como se desprende del análisis de los textos, su labor empírica estuvo fuertemente ligada al estudio de los procesos de modernización de su ciudad natal, atendiendo a las distintas esferas de acción (educación, clases sociales, estratificación social, etcétera) donde se demostraba el proceso de cambio acontecido entre la Córdoba \enquote{tradicional} anterior a 1948, y la Córdoba \enquote{en transición} del momento en que Agulla escribía. En este sentido, las indagaciones empíricas de los sociólogos estadounidenses, sumadas a los aportes de Eva Chamorro y Adolfo Critto, muestran por primera vez una verdadera vocación por conocer con fundamento científico cómo funcionaba la sociedad cordobesa.

Este pionero trabajo de investigación se profundizó luego de 1966. Además de continuar sus publicaciones, Agulla inició una labor de formación de jóvenes investigadores que atravesaron la experiencia de la Escuela de Graduados de Sociología, creada bajo su impronta, que funcionará entre 1967 y 1976. En definitiva, a través de los textos de este autor se puede observar el proceso de implantación de una nueva forma de concebir y practicar la disciplina en una institución en la cual hasta ese momento la sociología empírica no había tenido predicamento. La renovación iniciada en esta época, aunque finalmente quedara frustrada por el golpe de Estado de 1976, es uno de los antecedentes más importantes para pensar la institucionalización de la sociología en la Universidad Nacional de Córdoba.

%Capítulo 4
\chapter{Autonomía e integración. Peronismo, clase obrera y democracia en la sociología política de Torcuato S. Di Tella (1957-1970)}

\footnote{\textsuperscript{*} Publicado en \emph{Revista Argentina de Ciencia Política}, n.º 1 (31). \url{https://publicaciones.sociales.uba.ar/index.php/revistaargentinacienciapolitica/article/view/9421}.}

\section{Introducción}

La vida de Torcuato Salvador Di Tella no es ningún misterio, incluso para los no iniciados en la sociología. Conocido para el gran público tanto por la universidad que lleva el nombre de su padre como por su labor como funcionario durante las presidencias de Néstor Kirchner (2003-2007) y Cristina Fernández (2007-2015) no cuenta, sin embargo, con estudios en profundidad de los historiadores de la sociología argentina. De hecho, quitando la biografía escrita por su viuda Tamara Di Tella (2019),\footnote{Compuesta, por cierto, más de anécdotas familiares que de análisis detallados de sus trabajos.} y más allá de haber sido uno de los profesionales más célebres de la disciplina en el país, no hay indagaciones sociológicas en profundidad sobre su trayectoria o su sociología.

En este sentido, las excepciones serían los autores y las autoras que abordaron sus aportes para la comprensión del \enquote{populismo} \parencite{1531-QUATTROCCHIWOISSON1997,1555-CANOVAN1981}(Hennessy, 1970; Laclau, 1977).\footnote{Para un desarrollo de las contribuciones de Di Tella a este campo de estudios, véase \textcite[234-242]{278-AMARAL2018}.} Esto llama la atención si se tiene en cuenta que fue uno de los principales aliados de Gino Germani en la constitución de un nuevo \enquote{estilo de trabajo}, la sociología \emph{white collar} \parencite{1613-DELICH1977}, más conocida como \enquote{sociología científica}, así como un referente central en torno a algunos de los problemas sobre los cuales la sociología argentina hizo sus primeros y más importantes aportes: el sistema político local, la clase obrera y el peronismo \parencite{1511-SARLO2001,1548-NEIBURG1998,1565-BLANCO2006}(Pereyra, 2005).

De tal manera que el objetivo de este capítulo es reconstruir su trayectoria social y orientaciones sociológicas desde que comenzó a trabajar como profesional de la sociología en 1957, hasta finales de la década siguiente. Durante estos años, Di Tella publicó sus primeros ocho libros (de un total de 39), dando cuenta de una progresiva maduración de sus razonamientos en torno a procesos sociales argentinos y latinoamericanos, que se convertirían en preocupaciones constantes de su obra posterior. El punto de corte viene dado por el cese de sus actividades intelectuales plasmadas en publicaciones, prolongada durante casi tres lustros, siendo retomadas recién con la vuelta de la democracia.\footnote{Aquí nos referimos exclusivamente a libros, ya que dio a conocer varios artículos, sobre todo en la revista \emph{Desarrollo Económico}, de la que fue director en estos años.} Así, en 1984 daría a conocer \emph{La rebelión de esclavos de Haití} y, al año siguiente, \emph{Sociología de los procesos políticos}, considerada por el propio autor como su \enquote{contribución teórica principal} \parencite[289]{1553-CAMOU2009}.

La hipótesis que se sostiene es que, durante los primeros años como profesional de la sociología, los textos producido por Di Tella muestran una pobre utilización de la teoría sociológica clásica como consecuencia de una trayectoria e intereses ligados a la política antes que a la instrucción académica. Esto no significa que Di Tella careciera del \enquote{punto de vista} del sociólogo \parencite{1516-ALEXANDER2008,1522-BAUMAN2007}(Lahire, 2016)\footnote{Las referencias clásicas y contemporáneas que abordan la \enquote{perspectiva sociológica} pueden multiplicarse \emph{ad infinitum}. Baste decir que por ella debe entenderse que la conducta humana se explica a partir de su inserción en una red de relaciones sociales.} como puede ocurrir en otros casos que han sido objeto de indagación,\footnote{Por ejemplo, Lahire (2006:329-363) analiza y critica una tesis de doctorado en sociología defendida en La Sorbona, cuya autora realiza una defensa de la astrología pero sin adoptar una mirada sociológica. Por el contrario, lo hace desde el punto de vista de la propia astrología.} sino que debido a poseer una formación de grado ajena a la sociología y desarrollar pocas lecturas sobre temas que estuvieran desligados de sus intereses, se generó una ostensible ausencia de referencias a autores clásicos de la disciplina. Por cierto, esto va más allá de que formalmente tomara cursos con algunos de los sociólogos más destacados a nivel mundial de mediados de siglo~XX.\footnote{Esto quedará más claro a partir de las declaraciones al respecto del propio Di Tella en el apartado siguiente.}

Puede entonces decirse que, más que un experto en sociología, Di Tella fue un \enquote{sociólogo práctico} que encontró en esta disciplina la posibilidad de comprender y explicar el problema central de la política argentina: el peronismo. Por \enquote{sociólogo práctico} debe entenderse que, como consecuencia de los elementos antedichos (poseer una formación de grado que nada tenía que ver con las ciencias sociales --ingeniería industrial-- y la poca atención prestada a cuestiones que no estuvieran relacionadas con su pretensión de explicar --y criticar-- al peronismo), llevaron a que la adopción el mentado \enquote{punto de vista} sociológico tuviera que realizarse necesariamente en el campo concreto de las investigaciones, las cuales estuvieron orientadas por sus intereses políticos.

En estos años, Di Tella se empeña centralmente en la confección de taxonomías del mundo social, especialmente del sindicalismo, con el objetivo de estudiar bajo qué condiciones la clase obrera argentina podría desarrollar una política autónoma del Estado y moderar sus reivindicaciones, para posteriormente integrarse a una democracia constitucional pluralista (Di Tella, 1964; 1969b; Di Tella, Brams, Reynaud, Touraine, 1967). De manera que, durante estos primeros tiempos, en los textos de Di Tella los autores clásicos de la sociología podían ser nombrados al pasar (Max Weber), mencionados a partir de referencias de segunda mano (Karl Mannheim), o, simplemente, prescindir en buena medida de sus aportes (Émile Durkheim). Las excepciones fueron Karl Marx,\footnote{Sobre quien también se observa un uso somero. Como se verá, fue la descripción del \enquote{bonapartismo} en \emph{El XVIII Brumario de Luis Bonaparte} (1852) lo que más interesó a Di Tella, ya que ubicó al peronismo bajo esta categoría.} quien no era de su agrado pero cuya lectura fue estimulada por Seymour Martin Lipset, su director de tesis de maestría, y algunos sociólogos funcionalistas, entre quienes Robert Merton tuvo especial importancia, debido a que fue uno de sus profesores en la Universidad de Columbia.

Entonces, por un lado, Di Tella intentará amoldar a Marx a una versión de izquierda moderada con un fuerte componente liberal, en línea con el Partido Socialista de Argentina, del que fue un afiliado histórico. Por otro lado, sus estudios de maestría en Nueva York bajo dirección de Lipset, con quien mantendría relación hasta el final de su vida, lo dotaron de insumos teórico-metodológicos fundamentales para practicar la sociología durante la década de 1960 aunque, como se ha dicho, el trasfondo teórico de la tradición disciplinaria en sentido amplio no se ve utilizado en los escritos que se analizarán.

De esta forma, posicionado desde el socialismo liberal, o siendo un \enquote{socialdemócrata antes de tiempo} \parencite[283]{1553-CAMOU2009}, como diría de sí mismo muchos años más tarde, Di Tella formaría parte de un grupo de intelectuales que, bajo el liderazgo de Germani, contribuyeron tanto a la comprensión del peronismo como a estudiar las condiciones de su \enquote{superación}. La \enquote{teoría de la modernización} en boga durante el inmediato posperonismo sostenía que las sociedades atraviesan una sucesión de etapas que, en Argentina, culminaría en la \enquote{desperonización} de la clase obrera. Esta perspectiva del proceso social convirtió a la sociología de comienzos de la década de 1960 en un espacio a ocupar si se quería tener una voz autorizada para hablar del peronismo, la clase obrera y el sistema político \parencite{1548-NEIBURG1998}.

Luego de estas observaciones operativas, el capítulo se divide en tres partes. En la primera, se reconstruye la trayectoria de Torcuato Di Tella atendiendo a sus orígenes sociales, espacios de socialización y principales vínculos con la sociología y la política. En la segunda, se abordan sus trabajos sociológicos producidos entre 1957 y 1970,\footnote{Teniendo en cuenta que entre 1957-1958 Di Tella participó en la investigación sobre el sindicalismo chileno impulsada por el Instituto de Sociología de la Universidad de Chile, que se publicaría bajo el título \emph{Sindicato y comunidad. Dos tipos de estructura sindical latinoamericana} recién en 1967.} con el objetivo de observar qué problemas abordó así como sus usos de autores y conceptos relevantes de la tradición sociológica. Como no se pretende ser exhaustivo, se tomará como referencia principal su primer libro publicado en Argentina, \emph{El sistema político argentino y la clase obrera} (1964), atendiendo al núcleo de preocupaciones presentes en este trabajo, pero teniendo en cuenta las profundizaciones de los fenómenos indagados en otros textos del mismo período. Finalmente, las conclusiones retoman y sintetizan las dimensiones de análisis más relevantes del capítulo, intentando brindar una imagen del conjunto del pensamiento sociológico de Di Tella en esta época.

\section{Trayectoria social de Torcuato S. Di Tella}

Torcuato Salvador Di Tella nació en Buenos Aires el 29 de diciembre de 1929 y fue hijo de los inmigrantes italianos Torcuato Di Tella y María Robiola.\footnote{Su padre era ateo y su madre católica. Di Tella cuenta que tuvo una etapa juvenil de \enquote{fervor cristiano} \parencite[264]{1553-CAMOU2009} influida por Miguel de Unamuno, Emmanuel Mounier y Jacques Maritain, aunque confiesa que este \enquote{ataque místico} era una rebelión contra el \enquote{materialismo empresarial} de su padre. Durante la carrera de ingeniería tendría vínculos con el \enquote{Humanismo} y, más tarde, intentaría una sutil ligazón entre cristianismo y marxismo en \emph{¿Socialismo en la Argentina?} (1965a:10), donde plantea que se trata de corrientes con \enquote{numerosos puntos de contacto}.} Como es de público conocimiento, aunque su padre no provenía de una familia adinerada logró convertirse en uno de los empresarios más importantes de Argentina. Fundó la Sociedad Italiana de Amasadoras Mecánicas (SIAM) en 1910 junto a Alfredo y Guido Allegrucci, empresa que se diversificaría en la década de 1920 produciendo surtidores para YPF y, durante el decenio siguiente, desarrollando una línea de electrodomésticos para el hogar, siendo la heladera eléctrica el más célebre. Posteriormente, la empresa incursionaría en otros rubros como el automotor, aunque Di Tella padre no llegaría a verlo ya que fallecería en 1948.\footnote{Sobre la evolución de SIAM desde su fundación hasta su crisis final durante la década de 1990, véase \textcite{1583-COCHRAN2016}.} La importancia de esta empresa fue tal que el propio Torcuato Salvador no se privó de escribir una biografía de su padre, donde cuenta la trayectoria familiar de los Di Tella así como el devenir de la compañía, atendiendo al cambiante contexto político de la Argentina durante la primera mitad del siglo~XX (Di Tella, 1993).\footnote{Para una reconstrucción de la trayectoria de la familia Di Tella que contemple la actividad de Guido y Torcuato como funcionarios de los presidentes Menem y Kirchner, respectivamente, véase \textcite{1448-CASSESE2008}.}

Como se indica en este libro, Di Tella (1993:164) causó un gran disgusto a su padre cuando le dijo que quería dedicarse a la sociología en lugar de a la actividad empresarial. \enquote{Vos no respetás mi obra}, contestó indignado, \enquote{no me has comprendido}. Un derrame cerebral, del cual nunca se recuperaría, terminaría con su vida apenas tres meses después de este altercado. No obstante, Di Tella terminaría la carrera de ingeniero industrial en la Universidad de Buenos Aires en 1952, cumpliendo así con el mandato paterno. Además, reconocería que su formación básica estaba en la tradición familiar de su padre \enquote{y de todos esos italianos que venían y expresaban un socialismo liberal, democrático y claramente antiperonista y anticomunista}\footnote{Se refiere a los inmigrantes antifascistas que formaron parte del círculo de su padre. De hecho, este último no solo sostuvo ideas socialistas sino que estableció \enquote{una relación epistolar con Filippo Turatti, dirigente del ala reformista del socialismo italiano, que organizó en el exilio francés la Concentrazione Antifascista} (Di Tella, 1993:55). Di Tella también colaboró con el envío de fondos a Giustizia e Libertá, La Giovane Italia y Nuovo Avanti.} \parencite[270]{1553-CAMOU2009}.

Fue este influjo paterno el que despertó el interés por la política e hizo que Di Tella nunca ejerciera la profesión de ingeniero, partiendo a los Estados Unidos al año siguiente de haberse recibido para realizar el Master of Arts en Sociología en la Universidad de Columbia. \enquote{En Nueva York --comenta Tamara Di Tella-- estudió Sociología con los grandes. En la Universidad de Columbia tuvo como profesores a Robert Merton y Paul Lazarsfeld (\dots) a Robert Lynd (\dots) y a Daniel Bell (\dots). Su director de tesis fue Seymour Martin Lipset (\dots)} (Di Tella, 2019:30).

Precisamente, fue este último quien, a partir de su investigación sobre los tipógrafos de Nueva York, lo inspiró para su pesquisa sobre el sindicalismo en Chile. Di Tella reconoce que no era un buen estudiante y que, en sus palabras, \enquote{leía mucho de lo que me interesaba, y no me quedaba tiempo para las lecturas obligatorias}. De hecho, cuando Lipset le preguntó cuál era el tema que quería investigar, contestó que

\begin{quote}
quería entender qué era el peronismo. Él me preguntó entonces si yo había leído El Dieciocho Brumario, y le contesté \enquote{¿el 18 de qué?}. Para mí Marx tenía una connotación negativa, porque era materialista y estaba en contra de la religión, y un poco ensuciado por el comunismo, pero además yo no lo había leído \parencite[267]{1553-CAMOU2009}.
\end{quote}

Di Tella dedicó la mayor parte de su tiempo a participar en actividades políticas de la Juventud Socialista, aunque logró finalizar su tesis de maestría sobre educación obrera a fines de 1953. Fue entonces cuando conoció a su primera esposa y madre de sus primeros dos hijos, una mujer india llamada Kamala Apparao. Con ella realizó viajes por Suecia, Yugoslavia e Israel, donde encontró su \enquote{modelo socialista ideal}, el kibutz. Después, cada uno pasó un tiempo en su respectivo país hasta que finalmente Di Tella viajó a India, donde se casaron. De esta experiencia se rescatan los vínculos establecidos con el Partido Socialista local y la fundación del Club del Libro Socialista. Sin embargo, residieron allí por poco tiempo ya que se marcharon hacia Inglaterra en 1955.

En ese año Di Tella comenzó sus estudios de doctorado en la London School of Economics, los que jamás lo finalizó. Según comentó, durante su estancia en Inglaterra ni los profesores eran muy buenos ni causaron algún impacto en él. \enquote{Mi formación intelectual en Inglaterra --decía-- no fue en la universidad sino en el Partido Laborista, en la Sociedad Fabiana, en los sindicatos a los que iba a explicar dónde quedaba la Argentina y qué malo era el peronismo} \parencite[270]{1553-CAMOU2009}. De este modo, desde muy temprano Di Tella estuvo ligado al socialismo liberal, lo cual no solo lo volvía opositor al peronismo, sino que también suponía mantener relaciones tirantes con el \enquote{marxismo duro}, dado su fuerte rechazo a las políticas autoritarias de la Unión Soviética.

De hecho, durante el tiempo que pasó en la Universidad de Chile, donde consiguió su primer trabajo como sociólogo en 1957, se relacionaría con varios políticos socialistas de tendencia moderada como Raúl Ampuero, Federico Klein y Clodomiro Almeyda, a quien había conocido en Londres a través de su amigo chileno Claudio Véliz\footnote{Di Tella colaboró con textos para algunos libros compilados por Véliz: \emph{Obstacles to change in Latin America} (1965) y \emph{Latin America and the Caribbean: A Handbook} (1968).}, quien lo presentó a su vez a Salvador Allende. Fue entre 1957 y 1958 que trabajó en el procesamiento y análisis de datos sobre el sindicalismo chileno. De allí extraería los materiales para la redacción de lo que se publicaría bajo el título de \emph{Sindicato y comunidad. Dos tipos de estructura sindical latinoamericana} (1967). Este trabajo, aunque fue redactado por Di Tella, contó con la participaron de varios investigadores, entre los que se destacan Alain Touraine, Lucien Brams y Jean Reynaud.

Fue también en Chile donde conoció a Germani en 1957, quien insistió en que retornara a Argentina para ser profesor del Departamento de Sociología de la Universidad de Buenos Aires, lo que hizo entre 1959 y 1968. Durante los primeros años de vida de la carrera de sociología de la UBA, fundada en 1957, Di Tella trabajó junto destacados sociólogos y sociólogas como Miguel Murmis, Juan Carlos Marín, Jorge Graciarena, Ruth Sautú, Perla Gibaja y Darío Cantón. En esta época fue profesor asociado \emph{full time}, dictando asignaturas como \enquote{Desarrollo Económico y Social}, \enquote{Sociología Industrial} e \enquote{Introducción a la Sociología}, lo que le permitió escindirse de las responsabilidades familiares ligadas a SIAM.

En estos años el Partido Socialista se dividió, quedando Di Tella vinculado al Partido Socialista Argentino y, posteriormente, al Partido Socialista Argentino de Vanguardia. Colaboró con algunos escritos en las revistas \emph{Sagitario} y \emph{Situación}, aunque no fueron del agrado del partido y terminó por retirarse del mismo. Además, con su hermano Guido, creó la Fundación Torcuato Di Tella en 1958, la cual dio a luz al Instituto Di Tella\footnote{Sobre esta institución, véase \textcite{1629-GARCIA2021} King (2007).} y, unos años más tarde, a la Universidad Torcuato Di Tella. Junto a Aldo Ferrer, fue miembro fundador del Instituto de Desarrollo Económico y Social (IDES), del cual integró la comisión directiva desde 1978 y presidió entre 1984 y 1993. En este instituto dirigiría su prestigiosa revista \emph{Desarrollo Económico} entre 1971 y 1974. Por último, participó en la creación de la Fundación Simón Rodríguez, la cual presidió entre 1971 y 2001 \parencite{1553-CAMOU2009}(Di Tella, 2019).

Di Tella también cumplió funciones institucionales en la carrera de sociología, donde fue secretario general entre 1961 y 1963. Si bien la idea de Germani era que se hiciera cargo de la dirección del Departamento, esto resultaba imposible porque no había sido nombrado profesor a través de la sustanciación de un concurso. De todas maneras, ejerció el cargo informalmente, aunque los inconvenientes que surgieron debido a la mala organización hicieron que terminara renunciando. Sin embargo, siguió dictando clases hasta el año 1968, ganándose así el oprobio de los miembros del plantel docente que renunciaron cuando se produjo el golpe de Estado de 1966.

Tras la intervención a la Universidad de Buenos Aires realizada ese año, conocida como \enquote{La noche de los bastones largos}, se cerró la etapa de renovación iniciada en 1955 y comenzó un éxodo masivo de profesores \parencite{1536-BUCHBINDER2010}. En la Facultad de Filosofía y Letras, Justino O'Farrell y Gonzalo Cárdenas, miembros del plantel docente que había fundado la carrera de sociología en la Universidad Católica Argentina en 1959, se harían cargo de la dirección del Departamento y del Instituto de Sociología, respectivamente. Di Tella, por su parte, alejado por razones ideológicas de la radicalización política que se avecinaba en las aulas de la UBA, y ya habiendo dado a conocer sus libros \emph{El sistema político argentino y la clase obrera} (1964), \emph{¿Socialismo en la Argentina?} (1965), \emph{La teoría del primer impacto del crecimiento económico} (1966), \emph{Sindicato y Comunidad} (1967) y, junto a Germani y Graciarena, \emph{Argentina, sociedad de masas} (1965), inició un período de docencia en el extranjero.

Fue entonces profesor de la Universidad de California (1968), experto de la UNESCO en Río de Janeiro (1968), donde conoció a Fernando Henrique Cardoso, quien lo invitó como profesor del Institute of Latin American Studies de la Universidad de Londres y de Oxford (1969-1970). Más tarde, con intermitencias, ejerció la docencia en la Facultad Latinoamericana de Ciencias Sociales (1978-1985) y, una vez retornada la democracia, en el Ciclo Básico Común (1985-2002). A su vez, debe mencionarse una actividad en paralelo a las ya mencionadas que fue su ingreso a la carrera del Investigador Científico (1960-1967 y 1973-1976). Finalmente, entre los cargos políticos que lo hicieron conocido para el público general, se encuentran los de secretario de Cultura (2003-2004) y Embajador ante la República de Italia (2010-2015) (Di Tella, 2019).

A pesar de su alejamiento de la Universidad de Buenos Aires, Di Tella no solo siguió participando del grupo de sensibilidad liberal que había liderado la renovación de esta institución desde 1955, sino que además continuó con sus preocupaciones en torno al peronismo, la clase obrera y los sistemas políticos latinoamericanos. Producto de estos intereses en 1969 se publicarían dos compilaciones: la primera realizada junto a Halperín Donghi, \emph{Los fragmentos del poder}, que tenía \enquote{como preocupación central la definición de los agentes principales de la configuración de la Argentina moderna}, dando cuenta de \enquote{la construcción de objetos comunes por parte de la historia y las ciencias sociales} \parencite[122]{1511-SARLO2001}; y la segunda, \emph{Estructuras sindicales}, buscaba indagar los diferentes roles que los sindicatos desempeñan en las \enquote{sociedades modernas}. Por último, antes del cese de sus publicaciones de libros durante casi 15 años, Di Tella daría a conocer una serie de ensayos reunidos bajo el título \emph{Hacia una política latinoamericana} (1970).

El conjunto de estos trabajos constituye el \emph{corpus} a ser estudiado en este capítulo. Así, el recorrido a través del itinerario inicial de Torcuato Di Tella volverá ostensible una serie de preocupaciones e inquietudes que se reiterarán una y otra vez: el peronismo, los sistemas políticos comparados latinoamericanos, la autonomía de la clase obrera, los distintos tipos de organización sindical y la evolución del sistema político hacia el pluralismo democrático, siendo esto último una suerte de \enquote{ideal} a ser alcanzado. Según Di Tella, el desarrollo de este tipo de régimen político solo tiene lugar en sociedades con economías altamente desarrolladas, en las cuales la clase obrera está en condiciones de defender sus intereses sin recurrir a la violencia y el autoritarismo típicos de los países atrasados o de industrialización tardía. Como se verá, serán estos los ejes sobre los que girarán las reflexiones del autor.

\section{La sociología política de Torcuato S. Di Tella}

En el primer libro de Di Tella, \emph{El sistema político argentino y la clase obrera} (1964), está contenido un núcleo de preocupaciones que sería retomado por el autor a lo largo de su obra posterior. El problema central que vertebra el trabajo es la incorporación de la clase obrera al sistema político nacional y su grado de organización y autonomía. Di Tella comienza describiendo la vida del trabajador rural ligada al \emph{pater familia}, al patrón y al cura, con quienes se mantiene una relación de obediencia, concentrando así la autoridad. De la misma manera, en las barriadas populares de la ciudad se establece un tipo de vida propia de los inmigrantes, quienes guardan muchos resabios rurales. Entre ellos, el líder o jefe informal se mantiene en la \enquote{barra de muchachos}, siendo quien \enquote{provee de contactos al grupo, debido a su posibilidad y habilidad de moverse en medios sociales distintos} (Di Tella, 1964:18).

Estas características propias de los estratos inferiores de las clases bajas no son, sin embargo, extensibles a todo el mundo obrero. Hay un estrato superior formado por trabajadores de mayor calificación y educación, la \enquote{aristocracia obrera}, que muestra otra mentalidad. Este sector crea instituciones propias y genera dirigentes que actúan autónomamente \enquote{sin control \enquote{desde arriba}} (Di Tella, 1964:23). A su vez, posee mayores posibilidades de movilidad social y, por lo tanto, de desarrollar una mentalidad de \enquote{baja clase media}, lo que puede dificultar \enquote{su comunicación y comprensión con el resto de la clase obrera} (Di Tella, 1964:26). En la Argentina anterior a la Segunda Guerra Mundial fue la \enquote{aristocracia obrera} quien manejaba los sindicatos, mientras la mayoría de los nuevos obreros (migrantes recientes) se mantuvieron al margen de estas organizaciones. La mentalidad de este estrato obrero alto era \enquote{reformista}, es decir, que buscaban cambios graduales en el largo plazo.

Sin embargo, la incorporación a la política de los nuevos obreros tuvo lugar a partir de una \emph{élite de poder} (estratos altos y profesionales --militares, grupos clericales, intelectuales nacionalistas y sectores industriales--) que comprendió su situación vital y modo de pensar y ganó así su adhesión mayoritaria, dando lugar al nacimiento del peronismo. Se aprovechó entonces el fracaso en la constitución de un sistema institucional popular autónomo, adueñándose de las estructuras tradicionales de poder existentes en los barrios bajos. El peronismo otorgó \enquote{dignidad al obrero} (Di Tella, 1964:36), aunque desarrollando un régimen político autoritario favorecido por las formaciones psicológicas y las características de personalidad de los estratos bajos, es decir, un autoritarismo original de la familia rural que fue importado a la ciudad.\footnote{Esta argumentación sigue de cerca el razonamiento desarrollado en el clásico texto de Gino Germani (1956) sobre la integración de las masas y el totalitarismo.}

Se caracteriza entonces al peronismo como una fuerza política \emph{bonapartista}\footnote{Aunque este concepto fue desplazado rápidamente por el de \enquote{populismo} (Di Tella, 1965c), adquiriendo singular fuerza en \emph{Hacia una política latinoamericana} (1970), Di Tella no deja de referenciarse en Marx y el marxismo, mencionando a \enquote{Luis Bonaparte} (1970:98) y al \enquote{cesarismo} (1970:52) como forma de gobierno.}, lo que supone que el sector bajo \enquote{integrado en esa fuerza actúa bastante pasivamente, o al menos muy controlado y dominado, y que los intereses que van a ser defendidos por ella son los de los sectores altos de la sociedad, y no los de la masa obrera y campesina} (Di Tella, 1964:57). Sin embargo, en el peronismo la acción obrera no fue completamente pasiva y controlada por estructuras caudillescas, ya que también existió \enquote{espontaneísmo obrero}, es decir, situaciones donde se dan dirigentes obreros espontáneos, caracterizados por su inestabilidad y alto grado de emotividad. En Argentina, precisamente, el movimiento obrero autónomo (político y sindical) previo al peronismo fracasó por su desconfianza en ese \enquote{espontaneísmo}, en el cual veía inclinaciones autoritarias.

Según Di Tella (1964:64), para que los estratos bajos puedan tener una expresión propia y autónoma en el sistema político, \enquote{es esencial la integración de ese espontaneísmo dentro del conjunto de instituciones de la clase obrera}. Para que esto sea posible deben estudiarse las \enquote{limitaciones estructurales} (concepto mertoniano) de las sociedades. En este sentido, en la conceptualización de Gino Germani y Kalman Silvert, las sociedades con \enquote{participación total}\footnote{El esquema evolutivo de Germani y Silvert (1965) sobre la transición de sistemas políticos en América Latina es el siguiente: 1. revoluciones y guerras por la independencia nacional; 2. anarquía, \enquote{caudillismo} y guerra civil; 3. dictaduras unificadoras; 4. democracia representativa con participación limitada; 5. democracia representativa con participación ampliada; 6 (a) democracia representativa con participación total o 6 (b) participación total a través de revoluciones \enquote{nacional populares}.}, es decir, lo que Mannheim llamaba \enquote{democratización fundamental}, pueden desarrollarse a través de la democracia representativa, como en los países de antigua industrialización, o de \enquote{revoluciones nacionalistas populares}, como algunos casos de América Latina.

Esto es estudiado en profundidad en el capítulo de la compilación \emph{Argentina, sociedad de masas} (1965), enfocado en las \enquote{ideologías monolíticas en sistemas políticos pluripartidistas}\footnote{Versión castellana de la ponencia "Monolithic ideologies in competitive party systems" (1962b), presentada en el Tercer Congreso Mundial de Sociología.} en Latinoamérica. Precisamente, entre los \enquote{nacionalismos populares} son incluidos el peronismo, el castrismo, el PRI mexicano, el MNR boliviano y el APRA peruano. Estos movimientos se forman en los primeros momentos de la industrialización y dependen de la existencia de masas compuestas por inmigrantes recientes del campo a la ciudad movilizadas como resultado de una \enquote{revolución de las aspiraciones}; de élites de estratos medios o altos de la población o el ejército caracterizada por una inconsistencia de estatus; y una ideología o psicología dominante lo suficientemente difundida para generar entusiasmo.

Los \enquote{nacionalismos populares} suelen tener una organización vertical y desarrollar lealtades emocionales entre los seguidores del movimiento hacia su líder. La falta de un canal organizacional de la clase obrera integrada a movimientos de masas hace que sea necesaria la manipulación de las emociones colectivas a través de medios de comunicación, sin lo cual el movimiento se desintegraría por falta de organización. Por lo tanto, sostiene Di Tella (1965b:278),

\begin{quote}
(\dots) la situación peculiar de América latina (\dots) consiste en que el período de democracia parcialmente ampliada ha sido reemplazado por una etapa de participación total antes de que la clase obrera fuera capaz de desarrollar sus propias organizaciones para los fines de la participación (\dots) [por lo tanto,] el único modo de crear un movimiento político fuerte sobre la base de la clase obrera es construir una versión nacionalista popular del mismo. Pero, en tal caso, tendrá características monolíticas y, posiblemente, autoritarias, lo que hace difícil su integración en un sistema pluralista.
\end{quote}

Al concentrarse en estudiar la democracia a partir de sus vinculaciones estructurales con las dimensiones sociales (formación de la clase obrera) y económicas (grado de desarrollo industrial), este análisis político-intelectual de Di Tella lo colocan tanto dentro de una primera oleada de estudios sobre la democracia en América Latina,\footnote{Junto con Germani, Lipset, Barrington Moore, entre otros. Existe luego una segunda oleada de estudios en el pasaje de la década de 1970 a la de 1980, durante las dictaduras militares del Cono Sur y, hacia los años noventa, una tercera generación que amplifica y diversifica las indagaciones en torno a cuestiones relativas a las democracias latinoamericanas. Al respecto, véase \textcite{1554-CAMOU2007}.} así como entre aquellos interesados en la compleja relación que esta entabló con el peronismo en Argentina.\footnote{Tema que, por cierto, aún hoy en día convoca a reflexiones intelectuales. Véase \textcite{1543-NOVARO2014}, especialmente el capítulo de Loris Zanatta sobre \enquote{El peronismo y la vía holística a la democracia}.} El trasfondo normativo de estos escritos se encuentra en el supuesto de que una evolución normal de las \enquote{sociedades modernas} llevaría a una democracia constitucional pluralista, siendo los populismos en general, y el peronismo en particular, una suerte de \enquote{patología} o \enquote{anomalía} que debería reencauzarse hacia el ideal democrático.

Este análisis de lo que, desde el marxismo, pueden denominarse \enquote{variaciones estructurales}, con sus consecuentes influencias en la emergencia de \enquote{factores superestructurales}, complementado con \enquote{la construcción de teorías e hipótesis de alcance medio} (Di Tella, 1966a:16) están presentes en el libro \emph{Teoría del primer impacto del crecimiento económico} (1966).\footnote{Algunos de los resultados expuestos en este libro fueron adelantados en Di Tella (1961).} En línea con lo dicho hasta aquí, Di Tella revisa algunas de las características de la \enquote{sociedad tradicional},\footnote{Di Tella recogió datos para este trabajo durante la investigación realizada por el Instituto de Sociología de Rosario sobre los procesos de cambio social en zonas rurales. El estudio sobre el Valle de Santa María (situado en las provincias de Catamarca, Tucumán y Salta), permitió la publicación del libro de Albert Meister, Susana Petruzzi y Élida Sonzogni, \emph{Tradicionalismo y cambio social} (1963), editado por la Facultad de Filosofía y Letras de la Universidad Nacional del Litoral, la cual también publicó el mencionado libro de 1966.} donde encontraba importantes estratos medios. Con el crecimiento económico, muchos miembros de esta clase media rural se convierten en proletarios y, con la ampliación de la educación, se produce la ya mencionada \enquote{revolución de las aspiraciones}.

En el proceso de transición de la \enquote{sociedad tradicional} a la \enquote{sociedad moderna} puede darse un alto desarrollo industrial, con la consiguiente \enquote{concentración de una clase obrera altamente capacitada y entrenada en experiencias asociacionistas (partidos, sindicatos, entidades culturales varias)} (Di Tella, 1966a:161). En este caso la clase obrera organizada será reformista y autónoma, y estará integrada al sistema social dominante, justamente por las conquistas y ventajas que obtiene del mismo. Sin embargo, cuando esa clase obrera posee un \enquote{bajo nivel de secularización, organización y conciencia}, las expresiones políticas tenderán a adquirir la forma \enquote{de movimientos nacionalistas populares de diversas orientaciones, o de agitación anómica, como huelgas y protestas esporádicas} (Di Tella, 1966:162). Este último fue el caso del peronismo.

La paradoja que observa el autor es que los partidos obreros son necesarios para la democracia pluralista pero, al mismo tiempo, una amenaza por su ideología autoritaria. Para pasar de una forma de participación total a la otra, es decir, del nacionalismo popular a la democracia representativa,\footnote{Di Tella (1964:112) reconoce que en el momento en que escribe se vive un \enquote{importante atraso} en Argentina producto del retroceso hacia una \enquote{democracia con participación limitada}. No obstante, sostiene que puede considerarse un \enquote{camino obligado cuando se termina la vigencia de una experiencia de participación total con nacionalismo popular, y aún no están dadas las condiciones para estructurar esa participación total en una democracia representativa}.} debería constituirse un partido obrero de masas apoyado en los sindicatos, cuya expresión política debería ser gradualista y reformista. A su vez, el pasaje en cuestión estaría sostenido por la capacidad del sistema social y económico de entrar en una etapa de desarrollo sostenido y de avanzada industrialización. Sin embargo, en el caso argentino, Di Tella (1964:68) dice que \enquote{no se puede descartar la permanencia de una estructuración del tipo peronista, o sea la continuación de la coalición nacionalista popular}.

Precisamente, en \emph{¿Socialismo en la Argentina?} (1965a), donde estudia las posibilidades de la transición del país al socialismo,\footnote{El texto es más una proclama política más que un análisis sociológico, más allá de algunas referencias a Marx o a las \enquote{limitaciones estructurales} (Merton) del país para lograr la mentada transición. De hecho, durante todo el escrito apela a los \enquote{valores} del socialismo y finaliza con un programa de acción.} tendrá en cuenta que el peronismo se encontraba en un proceso de evolución que lo estaba convirtiendo en \enquote{un movimiento típicamente obrero}, es decir, que reducía su poder al control sobre las estructuras sindicales. Según Di Tella, una organización sindical que se organice conformando una organización de masas de tipo federativa que represente la realidad múltiple y multifacética del país, tendrá la tendencia a encaminarse por la vía gradualista y difícilmente podrá estructurarse como herramienta de cambio social en forma vertical y elitista como en la mayoría de los países subdesarrollados. Por este motivo, contar con el apoyo de los sindicatos resultaba una condición esencial para un movimiento socialista.

\begin{quote}
La forma concreta en que se estructura en la Argentina el partido popular que englobe a las varias corrientes ideológicas de la izquierda (justicialismo, socialismo, social-cristianismo) (\dots) [tendrá como] su principal componente (\dots) el mismo justicialismo, y su apoyo directo o indirecto la C.G.T. Lo más probable es que a ese justicialismo se sumen otros grupos de caudal mediano aunque no despreciable: fracciones socialistas, católicos progresistas, quizás radicales de izquierda, dentro de un Frente que al comienzo será una mera alianza electoral y que luego podrá irse fusionando cada vez más, construyendo una estructura central creciente. Los grupos intelectuales y profesionales, donde la izquierda predomina, podrán unirse a este conglomerado en alguna etapa de su formación, contribuyendo a darle una tónica socialista (Di Tella, 1965a:74-75).
\end{quote}

Sin embargo, ¿estaba el sistema político argentino en condiciones de darle lugar a la participación a la clase obrera bajo esta modalidad? En principio, Di Tella consideraba que \enquote{un sistema democrático constitucional, dentro del sistema capitalista, beneficia más a la clase obrera que lo que la perjudica} (Di Tella, 1964:108), en tanto las clases altas deben otorgar concesiones a las bajas, resultando en un sistema que, relegando la violencia, conviene a ambos sectores. Por lo tanto, cuanto más fuerza electoral tenga la derecha, los grupos de presión que actúan en ella tendrán más probabilidades de conseguir sus objetivos por medio de las urnas. Esto tiende a generar una mentalidad legalista, haciendo menos atractivo el camino del golpismo. Por su parte, la moderación de la clase obrera peronista y la incorporación de intelectuales y políticos de izquierda a esta coalición también podrían llevarla a una tendencia reformista o gradualista.

Este razonamiento desemboca en una hipótesis que Di Tella (1964:111) sostendrá en su obra posterior, la que supone que

\begin{quote}
\emph{la polarización del espectro político argentino en dos fuerzas, una de derecha y otra de izquierda obrera, lejos de constituir un factor de desequilibro institucional, constituye una de las principales garantías de permanencia y robustecimiento del sistema democrático constitucional} (bastardillas en el original).
\end{quote}

Esta conjetura se profundiza en \emph{Los fragmentos del poder} (1969a), donde Di Tella colaboró con una reflexión sobre el rol de la educación en el cambio social y el progreso económico.\footnote{Se trata de un texto anterior de Di Tella (1966b) con ligeras modificaciones.} Aquí comparaba el mercado de la educación privada con el de la educación estatal y aquel que debería desarrollarse en las organizaciones populares. Di Tella entendía que Argentina poseía \enquote{casi todos los requisitos económicos y sociales para poder darse entre nosotros el tipo de convivencia de los países pluralistas modernos} (Di Tella, 1969a:318). En este sentido, mientras el sistema educacional privado es uno de los seguros de vida de las clases altas del país, y como tal es integrante del sistema pluralista, el sistema estatal no bastaba como contrapeso del privado.

Contra las instituciones educacionales de los ricos, un país pluralista contrapone las instituciones educacionales populares, igualmente privadas pero con basamento distinto. Así, los sindicatos, cooperativas y partidos políticos obreros deberían organizar un sistema educacional de formación de cuadros que se caracterizan por cumplir una importante \enquote{función latente}, según expresión de Merton, la cual consiste en servir \enquote{de respaldo y de centro de operaciones, de basamento y apoyo, a los trabajadores intelectuales de su bando que operan en el sector estatal (\dots)} (Di Tella, 1969a:319). En la medida en que se produzca una transformación de los empresarios industriales en lo que Marx llamaba \enquote{clase para sí} se hacía \enquote{más necesaria la consolidación del otro sistema educacional paralelo, popular (\dots) [ya que] solo en esta forma se puede estructurar el poder de contrapeso que es necesario para asegurar un genuino pluralismo educacional en la Argentina} (Di Tella, 1969a:322-323).

Además, una de las incógnitas que despejaría la experiencia es si este sistema pluralista mantendría el \emph{status quo} o serviría para un \enquote{cambio social racional}, aunque lo importante es la existencia de un apreciable número de probabilidades para su afincamiento a largo plazo en Argentina. En este sentido, hay una serie de \enquote{requisitos funcionales} que deben cumplirse tanto para el arraigo del sistema democrático \enquote{con participación total} como para el sostenimiento de una coalición obrera integrada al mismo. Entre ellas, cuando el nacionalismo popular se convierte en un partido de sindicatos, pasa a compartir el enfoque reformista y gradualista que caracteriza a la clase obrera de un país industrializado y urbanizado y, al mismo tiempo, tiende a desaparecer el grueso de la élite no obrera de este movimiento y las masas movilizadas se convierten en obreros organizados. La ideología permanece, aunque expresada de forma más módica.

Esta tendencia del sindicalismo fue estudiada en \emph{Sindicato y comunidad} (1967), aunque también realizó algunas indicaciones en la introducción de \emph{Estructuras sindicales} (1969b).\footnote{Compilación de Di Tella donde se incluyen textos de Hugo Callelo, Miguel Murmis, Juan Carlos Marín, Julio César Jobet, T. S. Simey, Romain Gaignard, Seymour Martin Lipset, Martin Trow, James Coleman, V. L. Allen, Manuel Fernández y Azis Simão.} En este último libro Di Tella traza un esquema evolutivo de las organizaciones sindicales señalando una situación de \emph{masa aislada}, propia de los trabajadores alejados de los centros urbanos que, sin mucho contenido ideológico, encauzan su conducta a través de la violencia; otra \emph{anarco-portuaria}, es decir, un sindicalismo de élites combativas influidas por activistas ideológicos, con poco componente obrero, que adoptan una programática revolucionaria, aunque son débiles y difícilmente pasan a la acción; un \emph{sindicalismo paraestatal}, que genera movimientos de tipo populista, dirigido por sectores de la burguesía o del ejército que usan a los sindicatos como arma de control social y movilización de masas, orientándolos desde el Estado; finalmente, el \emph{sindicalismo de masas autónomo} es la situación típica de los países desarrollados. Este último tiende a ser burocratizado, con dirigentes pagos y servicios para los afiliados, manteniendo autonomía del Estado, aunque acatando las leyes del juego político.\footnote{En \emph{Sindicato y comunidad} (1967), los dos primeros casos son llamados \emph{espontaneísmo obrero} y \emph{asociacionismo voluntarista}.}

El último tipo es el que le interesa especialmente a Di Tella, ya que puede darse en una versión \emph{reformista pragmática} o \emph{reformista ideologista}, siendo la primera la que busca solo una mejora del nivel de vida de sus afiliados, mientras la segunda plantea, además, la obtención de una sociedad mejor, siendo comúnmente alguna versión de sociedad socialista. Por cierto, aunque ninguno de los tres primeros tipos son los típicos de países con alto grado de industrialización, ni lo fueron en su historia en la misma medida que en América Latina, Di Tella observa que en Argentina el \emph{sindicalismo paraestatal} predominante se encontraba en una evolución hacia un \emph{sindicalismo de masas autónomo}, por lo que cabría pensar que esta sociedad estaba adoptando parámetros de los países industrializados.\footnote{La idea que \enquote{Argentina ya no es el típico país subdesarrollado} (Di Tella, 1962a:45) era sostenida por el autor desde hacía varios años.}

Este esquema evolutivo aparece en \emph{Sindicato y comunidad} (1967), donde se realiza una comparación minuciosa de las estructuras sociales, actitudes y mentalidades de grupos obreros de dos importantes industrias chilenas: por un lado, la planta de acero de Huachipato y, por otro lado, la mina de carbón de Lota, ambas ubicadas en la ciudad de Concepción. Entre las características centrales de estas ciudades, se observaba que Lota era una comunidad más homogénea y culturalmente más cerrada que Huachipato, aproximándose a lo que Durkheim denominara \enquote{solidaridad mecánica}, donde no había condiciones sociales para una identificación de la clase obrera con los valores de la clase media. Así, se contrasta que, mientras en Huachipato un nivel mayor de educación se correspondía con ingresos más elevados, maximizándose las funciones de control social cumplidas por la educación, en Lota se daba la situación inversa: \enquote{a más educación corresponden menos ingresos, lo que minimizará las funciones de control social cumplidas por la educación} (Di Tella, Brams, Reynaud y Touraine, 1967:180).

Entonces, ¿qué tipo de sindicalismo puede encontrarse en una y otra industria? En Huachipato se identificaba un momento intermedio entre el \emph{Asociacionismo voluntario} y el \emph{sindicalismo de masas autónomo}, mientras Lota se encontraba en una etapa de \emph{asociacionismo voluntario} con importantes elementos de \emph{espontaneísmo obrero}, que se manifestaba en huelgas espontáneas, expresiones de violencia no planeada y metas y organización de corta duración. En las conclusiones del libro, Touraine explica que estas dos formas de organización no representan casos particulares sino \enquote{momentos sucesivos} o \enquote{dos etapas} de una misma evolución. De esta manera,

\begin{quote}
Lota está profundamente signada por el capitalismo familiar; Huachipato, creado por iniciativa del Estado, está dirigida por \emph{managers}. Generalmente de la una a la otra progresó eso que Max Weber denominaba burocratización, el establecimiento de una autoridad impersonal, fundada sobre reglas precisas, reemplazando a la autoridad personal de los propietarios y sus representantes (Di Tella, Brams, Reynaud y Touraine, 1967:320).
\end{quote}

En consecuencia, tanto los niveles de \enquote{conciencia proletaria} como las actitudes de los obreros de Lota y Huachipato se expresan de maneras distintas. Sin embargo, el \enquote{pasaje de Lota a Huachipato no es el del conflicto a la negociación sino de la ruptura a la oposición, de un conflicto de principios a un conflicto de objetivos} (Di Tella, Brams, Reynaud y Touraine, 1967:325). En Lota hay reacciones obreras guiadas por los afectos, en Huachipato son una consecuencia de la ideología. En el primer caso se reivindica la lucha, en el segundo el progreso. En el fondo, no se trata de otra cosa que distintas dimensiones de análisis de la transición de las \enquote{sociedades tradicionales} a las \enquote{sociedades modernas}. Este problema clásico de la sociología, actualizado en la denominada \enquote{teoría de la modernización} por muchos sociólogos y sociólogas latinoamericanos, entre los cuales Torcuato Di Tella ocupó un lugar destacado, dio lugar a profundas reflexiones intelectuales sobre el cambio social y político de América Latina durante la década de 1960.

\section{Conclusiones}

Torcuato Salvador Di Tella presenta una trayectoria que no es usual entre los practicantes de la sociología profesional de la segunda mitad del siglo~XX en Argentina. Con orígenes sociales elevados debido a la actividad empresarial de su padre, tuvo la posibilidad de formarse académicamente en los centros universitarios y los círculos intelectuales más importantes del mundo. Sin embargo, como se ha constatado en este capítulo, la educación formal y los vínculos con algunos de los sociólogos más destacados de la época no se tradujeron en una utilización relevante de la tradición sociológica clásica en sus investigaciones.

Esto último se explica por varios motivos. En primer lugar, porque el influjo paterno llevó a que sus estudios a nivel de grado no tuvieran nada que ver con la sociología o las ciencias sociales. En segundo lugar, como el propio Di Tella refiere, durante la instancia más importante de su formación sociológica en Estados Unidos \enquote{leía mucho de lo que me interesaba, y no me quedaba tiempo para las lecturas obligatorias}, pasando la mayor parte de su estancia en ese país en las reuniones de la Juventud Socialista. En tercer lugar, también admite que, cuando cursaba sus estudios de doctorado en Inglaterra, sus aprendizajes no tuvieron lugar \enquote{en la universidad sino en el Partido Laborista, en la Sociedad Fabiana, en los sindicatos}, a donde iba a explicar \enquote{dónde quedaba la Argentina y qué malo era el peronismo}.

En síntesis, si se toman en cuenta esta serie de observaciones, no cabe lugar a dudas que la adopción del \enquote{punto de vista} sociológico por parte Di Tella tuvo que llevarse a cabo \enquote{en la práctica} de la investigación concreta, iniciada en el marco provisto por la Universidad de Chile entre 1957 y 1958. Por lo tanto, la herencia económica paterna no resultó tan importante para su formación intelectual comparada con aquella inmaterial que recibió de su ideología y la de los exiliados italianos antifascistas, quienes le inculcaron el antiautoritarismo desde sus primeros años de vida. Fueron ellos quienes, como se ha dicho, \enquote{expresaban un socialismo liberal, democrático y claramente antiperonista y anticomunista}, lecciones aprendidas desde muy temprano por Torcuato. En este sentido, sus investigaciones sociológicas estuvieron fuertemente orientadas por sus intereses políticos, enfocados en desterrar lo que para él representaba el autoritarismo en el sistema político argentino, es decir, el peronismo.

Di Tella presentará así una versión sutil y refinada, aunque carente de citas eruditas (más allá de sus referencias a Marx y Merton), de la \enquote{teoría de la modernización}, entendida como una progresión de las \enquote{sociedades modernas} hacia la organización democrática en el plano político y autónoma del Estado en el mundo sindical. Sus razonamientos sobre el problema de la integración de las masas resuelto por vía del \enquote{totalitarismo}, en línea con lo planteado por Gino Germani (1956), abordaban la piedra en el zapato de quienes hubieran querido lograr esa representación política de la clase obrera por la vía institucional, expresada en los partidos de izquierda liberal integrados al sistema político argentino previo a 1943. De este modo, en la obra temprana de Di Tella es posible encontrar dos perspectivas sobre la \enquote{superación} del peronismo.

En una visión más rígida, el retroceso del sistema político luego del golpe de Estado de 1955 no era más que tomar impulso desde una sociedad democrática \enquote{con participación limitada} para acceder a otra con \enquote{participación total}, pero no ya en su versión \enquote{nacional-popular} sino \enquote{democrática representativa}. En este caso, el peronismo es entendido como \enquote{patología}, \enquote{anomalía}, \enquote{desviación}, etcétera, de lo que por naturaleza debería ser el sistema político en una sociedad altamente industrializada. En una visión más laxa del proceso argentino, se encuentra una insistencia en la \enquote{reeducación} de la clase obrera que, aunque mantuviera sus simpatías por el nacionalismo popular, debía finalmente moderar sus objetivos, incorporar aliados de izquierda en un sentido amplio y reconvertirse en un partido \enquote{reformista} o \enquote{gradualista} que se integrara a la democracia pluralista. A su vez, en lo que respecta a la expresión sindical de la clase obrera, de lo que se trataba era de lograr una organización autónoma del Estado.

Como contrapartida, y en pos de evitar la tentación del golpe de Estado, el sistema político democrático también debía incorporar un partido fuerte de derecha que lograra una representación de los intereses sectoriales de las clases altas. Dicho en términos bourdieusianos \parencite{1579-BOURDIEU2007}, la apuesta política e intelectual de Di Tella fue por la constitución de un \emph{habitus} \emph{democrático} de los actores encargados de representar los principales intereses socioeconómicos de la Argentina posperonista. De esta manera, en una oscilación entre la intención de integrar al peronismo bajo formas democráticas, y la búsqueda de una autonomía de la clase obrera respecto del Estado que, a su vez, dejara atrás el autoritarismo popular propio de las \enquote{sociedades tradicionales}, se constituyeron las reflexiones prácticas de uno de los sociólogos políticos argentinos más destacados de la segunda mitad del siglo~XX.

%Capítulo 5
\chapter{Una transición teórica hacia el marxismo latinoamericano. Análisis del concepto de \enquote{movilización política} en la obra temprana de Atilio Borón (1967-1975)}

\footnote{\textsuperscript{*} Publicado en \emph{Argumentos. Revista de crítica social} n.º 30, págs. 571-599. \url{https://doi.org/10.62174/arg.2024.9956}.}

\section{Introducción}

La importancia de la fundación de la Facultad Latinoamericana de Ciencias Sociales (FLACSO) en 1957, donde comenzarían a funcionar las Escuelas Latinoamericanas de Sociología (ELAS) y Ciencia Política y Administración (ELACP) en 1958 y 1966, respectivamente, en la constitución de Santiago en un polo de atracción regional para los científicos sociales latinoamericanos, es una historia que ha sido contada en varias oportunidades \parencite{1523-BEIGEL2009,1538-PEREZBRIGNOLI2008,1625-FRANCO2007}(Gómez de Benito y Morales Martín, 2022).\footnote{Por supuesto, en este aspecto deben tenerse en cuenta otras instituciones de suma importancia como, por ejemplo, la Comisión Económica para América Latina y el Caribe (CEPAL) o el Instituto Latinoamericano de Planificación Económica y Social, en la sede de la Organización de Naciones Unidas que se instala en Chile en la segunda posguerra.} En ese momento, los promotores de FLACSO y el Centro Latinoamericano de Pesquisas em Ciências Sociais de Río de Janeiro establecieron una división intelectual del trabajo entre la primera, que debía dedicarse exclusivamente a la enseñanza, y el segundo, que se orientaría hacia la investigación empírica.

En este sentido, el caso de la ELACP resulta representativo, ya que en sus orígenes tuvo por objetivo la formación de expertos, es decir, técnicos \enquote{que trabajan en y para el Estado, (\dots) para las ONG y organismos internacionales} \parencite[15]{1549-NEIBURG2004}, cuya especialidad serían los procesos de integración latinoamericana. Este proyecto, enmarcado en la Alianza para el Progreso, estaba estrechamente vinculado a los intereses de las fuentes internacionales de financiamiento provenientes de la UNESCO, el Banco Interamericano de Desarrollo (BID) y las Fundaciones Ford y Rockefeller, pero también de la Universidad de Chile, que prestaba sus instalaciones, y del propio Estado chileno, fuertemente interesado en comprender, evaluar y orientar los ostensibles cambios de su sociedad \parencite{1514-ABARZUACUTRONI2010,1524-BEIGEL2010,1529-QUESADA2010}.

Pero, ¿en qué consistían estas transformaciones? Desde fines de la década de 1930, durante el gobierno del Frente Popular, Chile había experimentado un proceso de modernización social y económica a tono con la época. El \enquote{pacto desarrollista} de la posguerra impulsó a la Corporación de Fomento de la Producción (CORFO) a crear empresas como la Sociedad Abastecedora Minera, la Compañía de Acero del Pacífico, la Industria Nacional de Neumáticos, el Laboratorio Chile S.A., Manufacturas del Cobre, Electromat, la Empresa Nacional de Petróleo, la Empresa Nacional de Energía y la refinería petrolera el Concón, entre otras.

Aunque la burguesía no estaba de acuerdo con la existencia de estas empresas estatales, compartió con el proletariado la política de industrialización. Sin embargo, los campesinos fueron excluidos del pacto, lo cual comenzaría a modificarse a partir de la reforma electoral de 1958. Desde ese momento, empezaron a tenerse en cuenta sus condiciones laborales y la reforma agraria pasaría a ser parte de las plataformas electorales de los partidos políticos. En 1962, el presidente derechista Jorge Alessandri Rodríguez sancionó una ley de reforma agraria que distribuyó tierras y creó la Corporación de la Reforma Agraria (CORA) y el Instituto de Desarrollo Agropecuario (INDAP). Luego de la victoria de la Democracia Cristiana (DC) en 1964, el proyecto del presidente Eduardo Frei se orientó hacia la incorporación de los campesinos y las mujeres, ampliando el aparato burocrático del Estado, que asumió el control mayoritario de la minería (la llamada chilenización del cobre) e incorporó técnicos en áreas sensibles \parencite{1517-ALWIN1986,1614-DERIZ1979}(Molina Silva, 1972).

La denominada Revolución en Libertad puso así punto final al proyecto de la oligarquía chilena (Lechner, 2004). Como señalan \textcite[613-614]{1518-ANSALDI2012}, al igual que en otros países latinoamericanos, fue relevante \enquote{la erosión de la hacienda, la base material de la dominación oligárquica}. La CORA efectivizaría la reforma agraria de 1967, continuando la iniciada por Alessandri que, a su vez, sería profundizada por el gobierno de la Unidad Popular (UP) entre 1970-1973. Por cierto, la bibliografía sobre este último gobierno, y el fracaso de la vía chilena al socialismo, es profusa.\footnote{Por ejemplo, existen trabajos sumamente interesantes que, con distintos enfoques, recuperan experiencias vívidas de la época como \textcite{1503-TOURAINE1974,1630-GARCES2013}.} Aunque las causas de su derrota son múltiples, existe consenso sobre que la agudización de su crisis tuvo lugar \enquote{cuando el precio internacional del cobre cayó significativamente, provocando la fuga de capitales y el bloqueo comercial y financiero por parte de Estados Unidos} \parencite[408]{1519-ANSALDI2012}.

Fue en este contexto de profundas mutaciones de la sociedad chilena, durante el pasaje del gobierno de Frei al de Allende, que las primeras generaciones de graduados de la ELACP desarrollarían agendas de investigación propias, alejándose del proyecto original limitado a la enseñanza. Además, esta transición coincidió con un momento de cambio en las perspectivas teóricas de las ciencias sociales. Si a comienzos de la década de 1960 era dominante el paradigma estructural-funcionalista, siendo la teoría de la modernización enfocada en la problemática del desarrollo su adaptación latinoamericana, hacia finales del decenio emergería la teoría de la dependencia, revitalizando tradición crítica del marxismo. Esto fue particularmente importante en sociología, donde la incorporación de Marx al canon de autores clásicos de la disciplina daría lugar a nuevas síntesis y reinterpretaciones teóricas \parencite{1516-ALEXANDER2008}.

Como comenta Edelberto Torres Rivas, de la cohorte 1964/1965 de ELAS,

\begin{quote}
(\dots) en FLACSO encontré un clima muy conservador. El director en ese momento era Peter Heinz, un suizo muy orientado por la moda norteamericana, Parsons, Merton y, por otro lado, con la poderosa influencia de Gino Germani desde Argentina. No había ningún curso de marxismo, todo era funcionalismo estructural, con alguna orientación antropológica. En el segundo año fue profesor nuestro Fernando Henrique Cardoso, que impartía un curso de Sociología de América Latina que se llamaba \enquote{Sociología de la modernización}. FLACSO (\dots) se propuso formar sociólogos con una fuerte base técnico-metodológica, con un manejo de base empírica muy fuerte, a los que se calificaban, lejos del marxismo, como \enquote{sociólogos científicos}, porque manejábamos las estadísticas, el análisis multivariado, etcétera \parencite[111]{1520-BATAILLON2006}.
\end{quote}

Sin embargo, el clima de época de finales de los sesenta terminaría por imponerse. De este modo, comenzarían a sentirse los efectos de la Revolución Cubana, el agotamiento del reformismo burgués de Frei, la reforma universitaria (que había comenzado con la huelga de 1966 en la Universidad Católica), el surgimiento de organizaciones revolucionarias como el Movimiento de Izquierda Revolucionaria (MIR) y el Movimiento Acción Popular Unitaria (MAPU), etcétera, conduciendo a una reorientación teórica de las ciencias sociales hacia el marxismo y la teoría de la dependencia \parencite{1452-VASCONI1995}.

Esto se hará palpable entre los estudiantes y graduados de ambas Escuelas de FLACSO, donde varios profesionales de la sociología procedentes de Argentina ganarían visibilidad por sus aportes a la comprensión de la realidad social chilena. Por cierto, si se exceptúa a los pocos argentinos que pudieron realizar estudios de posgrado en los países centrales durante estos años (como Eliseo Verón en Francia o Miguel Murmis en Estados Unidos) resulta evidente que, para la mayoría de los sociólogos provenientes del otro lado de la cordillera, FLACSO se convirtió en la primera instancia de formación de posgrado.

En efecto, si se suman los graduados y graduadas de ELAS y ELACP del período 1957-1973, es decir, entre la apertura de la primera y el golpe de Estado del 11 de septiembre del último año, se contabilizan 61 argentinos y argentinas.\footnote{La distribución por sexo muestra que fueron 42 varones y 19 mujeres. De ELAS se graduaron 29 varones y 16 mujeres y de ELACP 13 varones y 3 mujeres.} Si bien no todos eran Licenciados en Sociología\footnote{Téngase en cuenta que durante estos años se fundan las primeras licenciaturas en sociología de las universidades públicas y privadas de Argentina. La primera se crea en 1957 en la Universidad de Buenos Aires y, más tarde, comenzarían a funcionar las de la Universidad Católica Argentina (1959), la Universidad del Salvador (1963), la Universidad de Belgrano (1964), la Universidad Nacional de Cuyo (1968) y la Universidad Provincial de Mar del Plata (1970) \parencite{1616-DIAZ2016,1622-FICCARDI2013}(Garaventa, Lazarte y Civallero, 2016; Pereyra, 2012). Por la cercanía geográfica con Chile, también varios de los graduados y graduadas de la Licenciatura en Ciencias Políticas y Sociales, creada en 1952 en la Universidad Nacional de Cuyo, realizaron las Maestrías en Sociología y en Ciencia Política y Administración en esta época, como Yolanda Bórquez, Horacio González Gaviola, Ernesto Aldo Isuani, Eduardo Bustelo y Rubén Alberto Cervini.}, hubo varios egresados y egresadas de las carreras de las universidades argentinas con esta titulación como María Eugenia Dubois, Manuel Mora y Araujo, Rubén Kaztman, Norah Schlaen, Ponciano Torales, Carlos Alberto Hasenbalg, Teresa Kaplanski, Patricio Biedma, Juan Perret, José Omar Arguello, Ernesto Pastrana, Bárbara Cajdler y Atilio Borón.

Entre los mencionados, Borón se convertiría en la referencia intelectual más importante de la sociología marxista a nivel latinoamericano, aunque sus primeros trabajos académicos estarían vinculados a la sociología científica. Por este motivo, interesa prestarle atención a sus primeras investigaciones durante sus años de residencia en Chile. Borón realizó aquí la Maestría en Ciencia Política y Administración de la ELACP entre 1967-1968 y se desempeñó como docente e investigador de FLACSO desde 1969 hasta 1972. En esta época publicó de forma fraccionada su tesis sobre \enquote{La movilización política en Chile (1920-1970)} donde expresa su simpatía por el dependentismo y el marxismo, aunque este no fuera su marco teórico.

Esto se debe a que Borón se formó como sociólogo en Buenos Aires a comienzos de la década de 1960 cuando, como se ha dicho, el funcionalismo parsoniano y la sociología de la modernización eran la cosmovisión dominante de la disciplina. Por ello, se comprenderá fácilmente por qué la \enquote{movilización política} fue definida en sus primeros escritos desde las conceptualizaciones de Karl Mannheim y Gino Germani. En particular, fueron los conceptos de democratización fundamental, masas en estado de disponibilidad y actitudes tradicionales, acuñados por estos autores los que le permitieron estudiar el modo en que se incorporaron las clases populares a la vida política chilena.

Si se tiene en cuenta la importancia de Mannheim para la tradición de la sociología científica \parencite{278-AMARAL2018,1504-VILA2023,1565-BLANCO2006}, puede decirse que este autor fue el \emph{istmo teórico} que unió a Germani con Borón durante sus años en FLACSO. El punto de quiebre se producirá en Estados Unidos, cuando Borón se desplace hacia el enfoque marxista. En palabras del autor, fue en Harvard donde \enquote{mis intereses académicos y mi identidad política marxista terminaron de definirse} \parencite[69]{1568-BORON2020}.

Por cierto, este viraje no es algo que haya pasado desapercibido para los especialistas en el pensamiento de izquierda argentino. Según Néstor Kohan (2015:57), Borón invierte la ecuación de muchos marxistas de la década de 1960 que comienzan \enquote{en el comunismo y el marxismo y terminan en la socialdemocracia cuestionando la Revolución Cubana}. Por el contrario, Borón

\begin{quote}
(\dots) comienza en tiempos de estudiante en el catolicismo renovador de los años sesenta, pasa en los setenta al socialismo (\dots) y de allí en más, en forma progresiva e ininterrumpida, va asumiendo la identidad marxista y comunista, defendiendo públicamente (\dots) a la Revolución Cubana con Fidel Castro, al proceso bolivariano con Hugo Chávez y a diversas organizaciones insurgentes (Kohan, 2015:58).\footnote{Esta afirmación también está presente en la biografía dialogada de Borón con Alexia Massholder, donde esta última postula la idea de un itinerario a contramano de la mayoría de los intelectuales \enquote{que se inician en la rebeldía y radicalidad y terminan sus días renegando de sus \enquote{locuras juveniles} desde posiciones socialdemócratas, cuando no francamente de derecha. Quienes lean estas páginas encontrarán la trayectoria inversa} \parencite[8]{1444-BORON2023}.}
\end{quote}

Como estas investigaciones de Borón no han recibido la atención debida y, de hecho, el propio autor nunca explicó en qué consistió el viraje hacia el marxismo en términos teóricos, el objetivo de este artículo será verificar el desplazamiento señalado a través del estudio de sus trabajos sobre la \enquote{movilización política}, escritos al calor de las transformaciones de la sociedad chilena indicadas. Para llevarlo a cabo, en primer lugar, se reconstruirá la trayectoria del autor, prestando especial atención a sus espacios de socialización intelectual y política en Buenos Aires y Santiago. En segundo lugar, se abordarán las conceptualizaciones de Borón en torno a la \enquote{movilización política} en sus textos publicados entre 1970 y 1975, con el objetivo de mostrar el mentado pasaje del uso de categorías de la sociología científica a aquellas de la sociología marxista. Finalmente, las conclusiones retoman y sintetizan los aspectos más relevantes de los parágrafos anteriores.

La trayectoria social de Atilio Borón\footnote{La mayor parte de los datos recogidos en esta sección fueron tomados de la biografía indicada \parencite{1444-BORON2023} y del texto autobiográfico \enquote{Mi camino hacia Marx. Breve ensayo de autobiografía político-intelectual} \parencite{1568-BORON2020}.}

Atilio Alberto Borón nació en Buenos Aires en 1943 en una familia de inmigrantes italianos católicos. Su madre era ama de casa y su padre dueño de una joyería en el barrio porteño de Recoleta. Politizado desde muy temprano debido a las discusiones familiares en torno al peronismo,\footnote{Puede decirse que mantenían un apoyo moderado al gobierno peronista por los derechos otorgados a los trabajadores, aunque esto entró en un \emph{impasse} cuando se inició el conflicto con la Iglesia.} Borón tuvo sus primeros acercamientos a la política durante su adolescencia, primero al radicalismo de Ricardo Balbín y luego a la Democracia Cristiana (de la mano de Guido Di Tella, miembro de la línea interna de izquierda llamada Comunidad), aunque salió rápidamente desilusionado de ambas experiencias.

Decidido a estudiar sociología, aunque imposibilitado por su título secundario de Perito Mercantil para ingresar a la Universidad de Buenos Aires (UBA) sin rendir previamente doce complejas materias, terminó incorporándose a la recientemente creada carrera de sociología de la Universidad Católica Argentina (UCA) en 1959, donde se exigía realizar un curso de ingreso que constaba de cuatro asignaturas. Aquí se nutriría del ambiente de la renovación católica ligada al Concilio Vaticano Segundo, obteniendo fuertes estímulos intelectuales de profesores como José Enrique Miguens, Antonio Donini, Eduardo Zalduendo o Floreal Forni, aunque también formaban parte del plantel docente reaccionarios miembros de la Iglesia como el rector de la universidad Octavio Nicolás Derisi, y Luis María Etcheverry Boneo, a cargo de Introducción a la Filosofía \parencite{281-ZANCA2006}.

Sin embargo, Borón también tuvo vínculos importantes con el grupo de profesores de sociología de la UBA, a cuyos cursos asistía con regularidad. Entre ellos se destacaron Germani, quien posteriormente sería su director de tesis doctoral en Harvard, y Torcuato Di Tella, con quien luego trabajó en el Centro de Sociología Comparada.\footnote{Borón cuenta que su primer acercamiento a la obra de Gramsci fue gracias a Di Tella, quien le presentó a José Nun en 1962. Este último le comentó que estaba leyendo al pensador italiano, a lo que Borón responde \enquote{¿quién es Gramsci?}. \enquote{[Nun] me mira sorprendido, pero con buena onda me explica que: \enquote{fue el fundador del Partido Comunista italiano, un intelectual muy importante que renovó la teoría política marxista y sería bueno que lo leyeras}} \parencite[114]{1444-BORON2023}. Borón pidió a sus familiares en Italia el libro \emph{Notas sobre Maquiavelo} publicado por Einaudi en 1949, por lo que pudo leerlo antes que Héctor Agosti lo tradujera al castellano.} Borón debe a ambos el \enquote{descubrimiento} de América Latina, lo que constituiría una preocupación permanente a lo largo de su obra. A su vez, el otro \enquote{descubrimiento} de esta misma época, es decir, la obra de Marx, vino de la mano del libro de un jesuita francés, Jean-Ivez Calvez, originalmente publicado como \emph{La pensée de Karl Marx} (1956), a partir del cual pudo suturar el hiato existente entre el pensamiento revolucionario y su formación católica.

Borón se recibió en 1964 y ejercería la docencia en la UCA hasta el golpe de Estado de 1966. A partir del ascenso a la presidencia del general Juan Carlos Onganía, la carrera de Sociología de esta universidad sería cerrada y Borón quedaría sin empleo. No obstante, su contacto con Di Tella le permitió acceder a una entrevista para la obtención de una beca otorgada por la Organización de Estados Americanos, a través de un miembro de su Comité, Norberto Rodríguez Bustamante, para estudiar la Maestría en Ciencias Políticas y Administración en FLACSO entre 1967 y 1968, es decir, que formaría parte de la segunda cohorte. Allí contaría con un elenco de profesores notables como Fernando Henrique Cardoso, Robert Dahl, Francisco Weffort, Karl Deutsch, Gino Germani, Emilio de Ípola, Adam Przeworski, Enzo Faletto, Edelberto Torres Rivas, Marta Harnecker, Osvaldo Sunkel, Celso Furtado, entre los más destacados \parencite{1444-BORON2023}.

Esta segunda cohorte se diferencia claramente de la primera, cuyo cuerpo docente estuvo compuesto mayoritariamente por abogados provenientes del BID especializados en problemas de integración latinoamericana y desarrollo. Por el contrario,

\begin{quote}
[en la segunda cohorte] se sumaron graduados jóvenes de la escuela que fueron incorporados como \enquote{encargados de curso-investigadores}, que contribuyeron así al establecimiento de un cuerpo docente estable y se iniciaron entonces los primeros cambios institucionales que permitirían a la ELACP edificar un programa de investigación y docencia autónomo \parencite[82]{1515-ABARZUACUTRONI2014}.
\end{quote}

Entonces, si los primeros programas de la ELACP estuvieron supeditados a los desembolsos del BID, a medida que la escuela se consolidaba se \enquote{estableció un programa propio de docencia e investigación más allá de los intereses del banco} \parencite[83]{1515-ABARZUACUTRONI2014}. En este contexto, Borón dirigiría un proyecto sobre su tema de tesis, \enquote{La movilización política en Chile}\footnote{El cual era cercano al de Joan Reimer, titulado \enquote{Movilización social de sectores marginales urbanos}, del que participaron como ayudantes los argentinos Rubén Cervini y Aldo Isuani. Por su parte, otro argentino, Ernesto Pastrana, formaría parte de la investigación sobre \enquote{La movilización reivindicativa urbana de los sectores populares en Chile, 1962-1972} junto a Joaquín Duque, por lo que se observa que existía cierta comunidad de intereses intelectuales entre los estudiantes, docentes e investigadores de FLACSO de esos años.}, entre 1970 y 1971 (Notas de Investigaciones, 1970:166; Informaciones, 1971a:134). A su vez, tenía previsto iniciar otro al año siguiente titulado \enquote{Análisis comparativo de la emergencia del populismo en Argentina, Brasil y Chile} (Informaciones, 1971b:593) mientras dictaría el seminario \enquote{Movilización y participación política. Estado, clases y participación electoral} junto a Adam Przeworsky. Sin embargo, como ya se le habían otorgado dos prórrogas a su solicitud para ingresar al Doctorado en Ciencias Políticas de Harvard, no tuvo más remedio que radicarse en Boston en 1972.

Allí pudo asistir a las clases de profesores de la talla de Talcott Parsons, John Womack, Barrington Moore Jr., Carl Friedrich, Louis Hartz, Joseph Nye, Seymour Lipset, David Landes, Stephen Krasner, John Rawls, Robert Nozick, entre otros. Si bien Borón planificaba quedarse solo dos años para realizar los cursos obligatorios y, posteriormente, retornar a Chile para escribir su tesis, el golpe de Estado de 1973 frustró sus intenciones. Fue así que no solo debió cambiar su tema de investigación (sus estudios sobre Chile concluyen con un artículo de 1975), redactando una tesis doctoral sobre \enquote{La formación y crisis del Estado oligárquico argentino, 1880-1930}, sino que además debió permanecer en Estados Unidos hasta 1976.

Durante ese año, y pese a tener un jugoso contrato firmado con la Universidad de Yale, la reapertura de FLACSO en México (y el llamado de Arturo O'Connell, su secretario general), fue la motivación para mudarse a este país, donde no solo asumiría responsabilidades como docente de sociología latinoamericana, sino que también profundizaría su especialización en la filosofía política marxista de la mano de Adolfo Sánchez Vázquez, con quien entabló una duradera amistad. Sin embargo, una serie de desavenencias con René Zavaleta Mercado, intelectual boliviano a cargo de la sede mexicana de FLACSO, hicieron que fuera despedido y comenzara a trabajar en la Universidad Nacional Autónoma de México y el Centro de Investigación y Docencia Económicas.

Con el advenimiento de la democracia en Argentina, Borón retornaría definitivamente a su patria en 1984, aunque la reinserción en el circuito académico local le sería dificultosa ya que no solo tuvo escasas posibilidades laborales en la sede local de FLACSO sino que, además, su ingreso a la carrera de investigador científico del Consejo Nacional de Investigaciones Científicas y Técnicas (CONICET) se retrasó cinco años. No obstante, con la apertura de los concursos docentes de la recientemente creada Licenciatura en Ciencias Políticas de la UBA se haría con la titularidad de las cátedras de Teoría Política y Social I y Teoría Política y Social II.

Al mismo tiempo, junto a varios ex alumnos argentinos que retornaban al país, comenzaría con el emprendimiento intelectual que fue EURAL, Centro de Investigaciones Europeo-Latinoamericanas. Con posterioridad, tendría su primera experiencia como militante orgánico de un partido político, Democracia Avanzada, además de las gestiones como vicerrector de la UBA (1989-1994) durante el rectorado de Oscar Shuberoff y, más tarde, como secretario ejecutivo del Consejo Latinoamericano de Ciencias Sociales (CLACSO) (1997-2006). Una vez finalizado su trabajo en esta última institución, e incorporado al Centro Cultural de la Cooperación (CCC), lanzó el Programa Latinoamericano de Educación a Distancia (PLED), con el objetivo de \enquote{librar la batalla de ideas} en sintonía con los desarrollos informáticos de la época. Por último, ya más cercano a nuestros días, se haría cargo del Ciclo de Complementación Curricular del Departamento de Humanidades y Artes la Universidad Nacional de Avellaneda.

Finalmente, con el advenimiento de los denominados gobiernos progresistas en América Latina entabló vínculos con los presidentes que llevaron adelante las políticas reformistas más radicales en la región como Hugo Chávez, Evo Morales y Rafael Correa, además de Fidel Castro, líder de la Revolución Cubana. Esta época también coincidirá con el momento más prolífico de su producción intelectual en torno a varias problemáticas latinoamericanas. Así, entre los libros más importantes de estos años pueden mencionarse: \emph{Estado, capitalismo y democracia en América Latina} (1997), \emph{Tras el búho de Minerva. Mercado contra democracia en el capitalismo de fin de siglo} (2000), \emph{Imperio e imperialismo. Una lectura crítica de Michael Hardt y Antonio Negri} (2002), \emph{Socialismo del siglo~XXI. ¿Hay vida después del neoliberalismo?} (2008), \emph{Aristóteles en Macondo. Notas sobre el fetichismo democrático en América Latina} (2009), \emph{Crisis civilizatoria y agonía del capitalismo: diálogos con Fidel Castro} (2009), \emph{El lado oscuro del imperio. La violación de los derechos humanos por los Estados Unidos} (en coautoría con Andrea Vlahusic, 2009) y \emph{América Latina en la geopolítica del imperialismo} (2012).

Estos trabajos vuelven de forma reiterada sobre una agenda de investigación que comienza en la década de 1970 y que fue progresivamente refinada con cada paso que Borón dio en el camino hacia el socialismo, interrogándose una y otra vez por el capitalismo, el Estado, la democracia y la injerencia del imperialismo norteamericano en los países de América Latina. En el próximo apartado se abordarán algunos de sus primeros textos, en los que se apreciará su conceptualización de la \enquote{movilización política} emparentada con la sociología científica, para después producir un desplazamiento de la conceptualización mannheimiana a la gramsciana hacia 1975.

\section{De Mannheim a Gramsci: la \enquote{movilización política} en Chile}

Como se señaló más arriba, la tesis de maestría de Borón fue publicada de forma fraccionada durante la primera mitad de la década de 1970 en prestigiosos medios académicos como la \emph{Revista Paraguaya de Sociología, la Revista Latinoamericana de Ciencia Política} (Chile), \emph{Desarrollo Económico} (Argentina), \emph{Foro Internacional} (México) y los \emph{Estudios de la ELACP}. El tema de estos trabajos es la \enquote{movilización política}, concepto polisémico que en las formulaciones boronianas de inicios de la década de 1970 aparecerá ligado a la teoría de Mannheim y Germani, mientras que para 1975 se hará ostensible un desplazamiento hacia la conceptualización de Gramsci. Estos escritos buscan comprender de qué manera una serie de elementos como el desarrollo económico, la modificación de la estructura social, la movilización electoral o el nivel de organización de la clase obrera inciden en la conformación de partidos \enquote{de protesta} así como en la creciente preferencia del electorado por ellos.

El su primer artículo, publicado en 1970 bajo el título \enquote{Clases populares y políticas de cambio en América Latina}, Borón parte de la premisa de la situación de dependencia de las naciones latinoamericanas. Su objetivo es realizar una caracterización de los aspectos políticos de esa situación y delinear el tipo de régimen político necesario para superarla. \textcite[101]{1569-BORON2020} critica a los científicos sociales que caracterizan a los sistemas políticos de la región como retrasados en relación a aquellos de los países centrales, siendo su perspectiva etnocéntrica, determinista, dogmática y unilineal \enquote{inaceptable para cualquier estudioso crítico y riguroso de los problemas del desarrollo latinoamericano}.

En este sentido, hay elementos que distinguen a América Latina de Europa y que darían cuenta de la falacia de la vía necesaria al desarrollo, demostrando que las categorías europeas no sirven para comprender las realidades latinoamericanos. Borón retoma la taxonomía de los sistemas políticos de Jorge Graciarena, quien distingue entre los que poseen una orientación hacia el desarrollo y una orientación hacia el compromiso. Los primeros concentran todos sus recursos para la promoción del desarrollo social, económico y político de la nación, alterando radicalmente el régimen político existente en pos de dicha empresa. Los segundos apuntan a la manutención y estabilidad del régimen político, constituyendo a las autoridades en el punto de convergencia de grupos antagónicos en una situación de empate social. Este último es dominante en América Latina, aunque existen diferencias sustantivas entre los países que lo comparten.

Ahora bien, en el momento que \textcite[115]{1569-BORON2020} está escribiendo, los países latinoamericanos muestran un aumento acelerado de la \enquote{movilización política}, es decir, un proceso \enquote{análogo a la democratización fundamental de Mannheim}, que se manifiesta en \enquote{el incremento de la participación política, registrada al menos en su dimensión estatal}. Además, se trata de un proceso expansivo, es decir, que progresivamente abarca cada vez a más sectores y grupos sociales que antes se hallaban excluidos de la política, afectando a \enquote{una variada constelación de actitudes y comportamientos, creencias y normas} \parencite[116]{1569-BORON2020}.

Sin embargo, debe distinguirse entre la integración formal y la movilización política objetiva. Si la primera se limita a la sanción de una ley de sufragio universal, la segunda ocurre cuando los sectores en disponibilidad, según expresión de Germani, asumen un comportamiento activo, siendo normalmente su voto canalizado por los partidos \enquote{de protesta}. Según \textcite[118]{1569-BORON2020}, la \enquote{movilización política} tiene un \enquote{carácter irreversible} en tanto las políticas desmovilizadoras de los gobiernos represivos latinoamericanos terminan siendo de corta duración. Por lo tanto,

\begin{quote}
Si aceptamos que la política del compromiso no garantiza la creación de las condiciones sociales y políticas más propicias para superar la situación de dependencia, y si también se acepta que los intentos desmovilizadores son factores de congelamiento del \emph{statu quo}, está claro que se requieren nuevas alternativas políticas que corrijan las limitaciones del modelo de compromiso. Podríamos ampliar el marco de nuestra pregunta, e inquirir si lo que se cuestiona no solo es el modelo político, sino también la viabilidad del desarrollo económico latinoamericano en el modelo económico capitalista \parencite[119]{1569-BORON2020}.
\end{quote}

Esta primera reflexión sobre la \enquote{movilización política} en América Latina es profundizada en relación a Chile. En \enquote{La evolución del régimen electoral y sus efectos en la representación de los intereses populares: el caso de Chile} \parencite{1570-BORON1971} se reconstruyen los rasgos generales del régimen electoral chileno desde 1833 hasta 1970, teniendo en cuenta el carácter directo o indirecto de las elecciones, la naturaleza pública o secreta del sufragio y las restricciones impuestas para ser calificado como elector hábil. Del análisis surgen cinco etapas claramente delimitadas.

La primera, entre 1833-1874, se caracteriza por la unificación nacional bajo la hegemonía oligárquica. En esta época existía un régimen censitario con requisitos para sufragar que marginaban a las mayorías. La segunda, entre 1874-1920, es el momento de crisis de la dominación oligárquica y el ascenso de la burguesía urbana, que coincide con la época del parlamentarismo, entre la guerra civil de 1891 y el golpe de Estado de 1924. En esta fase se suprimen los requisitos de capital y renta para votar, ya que la reforma electoral presupone que quienes saben leer y escribir cumplen estas exigencias. Además, se garantiza el carácter secreto del sufragio, triplicando el número de votantes en pocos años.

La tercera etapa transcurre entre 1920-1949, con la consolidación de las clases medias, culminando con el triunfo del Frente Popular (integrado por los partidos Radical, Socialista y Comunista). Esta época presenta una inestabilidad política muy grande por las sucesivas intervenciones militares entre 1924-1931, aunque se produce una efectiva ampliación de los intereses incorporados al sistema político. Sin embargo, esto no se tradujo en una extensión del sufragio hasta 1949, cuando se incorporan las mujeres, abriendo una cuarta etapa que concluye en 1962 y que se caracteriza por la ampliación de las bases electorales. Finalmente, la quinta etapa muestra una aceleración del ritmo de movilización electoral, ya que la reforma de 1962 penaliza severamente el incumplimiento de la inscripción en los registros electorales. Esto se complementa con la reforma constitucional de 1970, que concede derechos políticos a los analfabetos y mayores de 18 años.

Este esquema es utilizado por \textcite{1571-BORON1970} en \enquote{Movilización política y crisis política en Chile, 1920-1970}, donde estudia de qué manera estas ampliaciones de estratos sociales que gozan de derechos políticos generan cambios en los partidos políticos y las coaliciones gobernantes. De nuevo, parte de una definición mannheimiana de \enquote{movilización política} como democratización fundamental, generada \enquote{a consecuencia de la activación que la moderna sociedad industrial ejercía sobre capas y sectores sociales que anteriormente se hallaban integradas pasivamente en la vida política} \parencite[2]{1571-BORON1970}.

Pero, en la historia del régimen electoral chileno se encuentra un hiato entre el otorgamiento de los derechos políticos y el efectivo ejercicio de los mismos, que es explicado por lo que Germani llamó el carácter tradicional de los nuevos contingentes incorporados. Entonces, Borón buscará constatar si la \enquote{movilización política} se traduce en la creación de partidos \enquote{de protesta}, es decir, que expresen intereses distintos de los partidos tradicionales, comenzando por la elección de Arturo Alessandri Palma en 1920, quien expresa la llegada al poder de los sectores medios.

Borón observa que el Partido Radical incorporaba en parte los intereses de la oligarquía, eliminando así su agresividad anti statu quo y la posibilidad de articularse con las clases medias menos favorecidas. Entonces, aunque buena parte de las clases medias estaban en disponibilidad, la nueva situación no se tradujo a nivel político. De modo que fue el \enquote{proletariado intelectual} \parencite[15]{1571-BORON1970}, es decir, la clase media ilustrada que creció al calor de la expansión del sistema educativo, la que ocupó un rol político activo. Fue solo luego del golpe de Estado en 1924, y el contragolpe de las Fuerzas Armadas que hizo retornar a Alessandri al poder, que se llevaron a cabo medidas reformistas que supusieron una ampliación súbita y voluminosa del cuerpo electoral, el cual continuaría aumentando durante el período 1925-1949 de forma lenta pero constante.

Después de la crisis de 1930 se inició una etapa de movilización electoral que llevaría a la victoria del Frente Popular. Sin embargo, la izquierda no tendría una \enquote{movilización política} exitosa en tanto buscó reforzar su penetración en su cuerpo electoral preexistente, o sea, en las clases populares urbanas. Al mismo tiempo, el gobierno de clase media que había tenido como aliado al Partido Comunista tomó la decisión desmovilizadora de prohibirlo en 1949, removiendo a todos sus electores de los registros. Finalmente, si las mujeres se incorporan a la vida cívica a partir de 1952,

\begin{quote}
(\dots) es revelador que en Santiago, la más importante ciudad de Chile y principal centro de modernidad del país, se manifieste recién en 1964 una decidida aprobación de la participación de las mujeres en la vida política. Es fácil imaginar la situación en los distritos rurales en lo relativo a este punto \parencite[27]{1571-BORON1970}.
\end{quote}

Sería recién en 1970 cuando las mujeres llegarían a ser el 50\% de las inscriptas para sufragar. Mientras tanto, el voto izquierdista comenzaría a crecer desde 1958 y el derechista a disminuir. Por lo tanto, concluye Borón, la extensión de los derechos políticos no garantiza la \enquote{movilización política}. En Chile, esto último fue producto de la presión de sectores obreros movilizados en coalición con fracciones de clase media, especialmente aquella con acceso a la educación superior. Sin embargo, esta alianza que desplazó a la oligarquía en 1920 era débil y las clases medias no tuvieron fuerza ni para constituir un movimiento anti \emph{statu quo} ni tampoco una clientela de extracción popular. Por su parte, los sectores populares comenzarían a movilizarse luego de 1930. En ese momento el sistema político no podía absorber sus demandas, debiendo ensayar formas de desmovilizar a los nuevos contingentes. Esto cambia hacia 1960, cuando se incrementa la participación política y se fortalecen las estructuras sindicales, liquidando la alternativa electoral conservadora.

De estas observaciones se derivan dos interrogantes. Por un lado, uno sobre la relación que puede establecerse entre la base económica y la superestructura política, que \textcite{1572-BORON1970} trabaja en \enquote{Desarrollo económico y comportamiento político} y, por otro lado, otro que explique de manera comparativa por qué el proceso de movilización electoral chileno fue tan pausado, lo cual se estudia en \enquote{El estudio de la movilización política en América Latina: la movilización electoral en la Argentina y Chile} \parencite{1573-BORON1972}.

En el primer caso, Borón comienza pasando revista sobre varios trabajos que \enquote{miden la democracia} en diferentes países. Por ejemplo, Seymour Lipset indaga en varios índices de desarrollo económico correspondientes a dos conjuntos de naciones (más democráticas o menos democráticas), llegando a la conclusión que los países más industrializados, urbanizados y con mayor nivel de educación son más democráticos. También Phillips Cutright ve a la democracia como una variable cuantitativa que se ubica en los niveles superiores dentro de una escala de desarrollo político, y Arthur K. Smith desarrolla un índice propio de grado de democracia.

Para \textcite[247]{1572-BORON1970}, \enquote{existe un grado de evidencia más que plausible que permite sostener que un mínimo de desarrollo económico y social es condición necesaria --aunque no suficiente- para el mantenimiento de una democracia política}. Sin embargo, en estos índices de democracia \enquote{intervienen supuestos y definiciones emanadas de la ideología liberal, cuyas consecuencias tienden a empobrecer el análisis} \parencite[250]{1572-BORON1970}. Esto ocurre porque se dirigen hacia los aspectos exteriores y formales de la democracia y no hacia las características profundas del régimen político: cuántos eligen, de qué información disponen, qué opciones reales tienen a su alcance, cuántos intereses sectoriales representan, la legitimidad del régimen político, etcétera.

Además, las perspectivas liberales no tienen en cuenta que la relación entre desarrollo económico y democracia no es simple. Para \textcite[252]{1572-BORON1970} \enquote{es necesario retener el carácter conflictivo, violento e inestable que caracterizó la marcha hacia la democracia en los países de mayor desarrollo económico, frecuentemente descuidado en las formulaciones de los científicos sociales}. Los modelos mencionados muestran distorsiones producidas por los estereotipos y valores culturales del contexto norteamericano de mediados del siglo~XX en el que fueron desarrollados. Por el contrario, la tradición marxista

\begin{quote}
(\dots) ofrece una interpretación que es de suma utilidad para la comprensión del proceso global de cambio de las sociedades, en la cual el conflicto es incorporado como una categoría central y privilegiada. El modelo marxista es, tal vez, la más importante teorización en torno a los efectos desestabilizadores emanados del desarrollo económico capitalista: las tesis acerca de la agudización de las contradicciones de clase y la pauperización progresiva son algunas de las líneas de elaboración más prometedoras en esta dirección \parencite[254]{1572-BORON1970}.
\end{quote}

En este sentido, Borón pondera positivamente los estudios de Jorge Graciarena, Enzo Faletto, Fernando Henrique Cardoso, Glaucio Soares y Torcuato Di Tella. En el caso de los tres primeros porque no califican al conflicto como desviación, y en los últimos dos porque demuestran que la relación entre desarrollo económico y radicalismo político es positiva. Sin embargo, en Chile no parece haber evidencia uniforme que pruebe o refute la validez de esta última hipótesis, siendo el resultado diferente según el tipo de indicador utilizado y la unidad de análisis. De este modo, \enquote{el análisis de los coeficientes de correlación entre la votación radical de izquierda y los diversos indicadores de desarrollo económico, parecería indicar que (\dots) la asociación es débil en 1958, y más frágil aún en 1964} \parencite[283]{1572-BORON1970}. No obstante, desde este último año parece haber una creciente autonomía de la conducta izquierdista con respecto al desarrollo económico, que se explicaría por la lenta penetración de comunistas y socialistas en el campesinado chileno.

En definitiva, algunas dimensiones del desarrollo económico pueden no estar asociadas al radicalismo político de izquierda e, incluso, pueden existir variables que influyan negativamente en este sentido. Ahora bien, ¿qué ocurre si estamos en presencia de procesos económicos similares pero que generan \enquote{movilizaciones políticas} diferentes? Esto se aborda en el segundo artículo mencionado, que estudia comparativamente los casos de Argentina y Chile. Si ambos países tuvieron condiciones económicas relativamente similares en sus fases de desarrollo \enquote{hacia afuera}, ¿por qué Argentina tuvo una súbita apertura política mientras en Chile fue mucho más pausada?

Borón parte aquí de los contenidos políticos específicos de los nuevos contingentes incorporados para definir la \enquote{movilización política}. Entre ellos encuentra el cambio en las actitudes tradicionales (desinterés, apatía, falta de información) y su adhesión a partidos \enquote{de protesta} que expresen sus intereses. Así, la definirá como un tipo de comportamiento colectivo que supone que grandes sectores de la población, regularmente pertenecientes a las clases populares, se introducen en un plazo relativamente breve a la política, desempeñándose de una cierta manera. Pero, además, se tienen en cuenta las variaciones experimentadas por el cuerpo electoral, el poderío cambiante de los partidos políticos y las políticas adoptadas por el gobierno, para verificar si efectivamente se ampliaron los nuevos intereses y, por lo tanto, si participaron en la toma de decisiones.

Definido el concepto, Borón avanza sobre la comparación demostrando que, si a comienzos de siglo~XX en Argentina se generó una ruptura de las reglas del juego de la democracia burguesa, en Chile se preservaron debido a los mecanismos de negociación existentes y el mayor grado de institucionalización del sistema político. De igual manera, si en el primer caso la \enquote{movilización política} se desenvolvió de forma acelerada, en el segundo lo hizo lentamente, dando tiempo a la constitución de un sistema partidario sólidamente institucionalizado.

Un tercer punto de contraste refiere a la presencia de partidos obreros. Así, mientras Chile contó con los partidos Comunista y Socialista muy tempranamente, viabilizando la apertura del sistema político, en Argentina la movilidad social ascendente impidió que estos partidos se transformaran en partidos de masas. Además, como el sujeto social que debían representar era en su mayoría inmigrante, se hallaba excluido del juego político. Por último, respecto a la velocidad de la \enquote{movilización política}, en Argentina fue un proceso sumamente acelerado en comparación no solo con Chile, sino también en relación a los países europeos, siendo \enquote{solo en Suecia [donde] las tasas de movilización electoral acusaron un ritmo más acelerado que en la Argentina} \parencite[232]{1573-BORON1972}.

Esto fue producto de la acción conjunta de dos movimientos sociales: por un lado, el sindicalismo, que activó y organizó la protesta obrera contra la explotación industrial, la carestía de la vida, los problemas de vivienda urbana, etcétera y, por otro lado, la Unión Cívica Radical, que se mantuvo fuera de las reglas de juego oligárquico, provocando tres revoluciones (1890, 1893 y 1905) con un fuerte apoyo popular. Estos movimientos, aunque con diferente composición social, características de liderazgo, naturaleza de demandas y contenido ideológico, precipitaron la crisis del Estado oligárquico a través de la conquista del voto masculino. De la misma manera, si el radicalismo significó un cambio drástico para los conservadores, el peronismo liquidó sus posibilidades electorales, debiendo optar por los golpes militares para la representación de sus intereses.

Por el contrario, en Chile los viejos partidos no desaparecieron, adquiriendo una mayor adaptabilidad y pudiendo asimilar la ampliación del electorado. Así, las fuerzas conservadoras sobrevivieron al crecimiento de la masa electoral, debiendo ya en el siglo~XIX garantizar la representación de algunos intereses de las clases medias y sectores de la burguesía, lo que explica la supervivencia de la derecha chilena en el siglo~XX. La diferencia sustancial se encuentra en la actuación de los radicales. Mientras en Argentina no tuvieron una actitud negociadora con el régimen, en Chile fueron más pragmáticos, interviniendo en los comicios organizados por la oligarquía a sabiendas del fraude. En resumen, Borón señala que

\begin{quote}
\ldots quizás la estabilidad del sistema político chileno no se deba tanto a las innegables pautas \enquote{conciliadoras} existentes en su seno cuanto a la escasa agresividad política y económica de las demandas formuladas en nombre de los nuevos estratos políticamente relevantes. En otros términos, la calidad e intensidad de las demandas fueron de tal tipo que a lo largo de muchos años de historia política chilena, sus grupos dominantes pudieron satisfacerlas parcialmente, por cuanto ellas no alteraban las bases esenciales de su dominación de clase. Asegurando la gratificación parcial de esas demandas, regulaban la presión democratizadora dentro del sistema político y perpetuaban las condiciones de su dominación \parencite[239]{1573-BORON1972}.
\end{quote}

Finalmente, aunque en ambos países hubo intentos desmovilizadores (sobre todo golpes de Estado), en Chile no se observa un quiebre del sistema partidario: subsisten a lo largo del siglo~XX los tres partidos tradicionales del siglo~XIX, a los que se agregarían dos de inspiración marxista y uno demócrata cristiano. Por el contrario, en Argentina las distintas fases de movilización fueron seguidas por profundas crisis del sistema partidario. De modo que, así como se comprueba una rápida caída del partido conservador a principios del siglo~XX, puede apreciarse una declinación del radicalismo luego de la integración de las clases populares a mediados de siglo~XX.

Ahora bien, esta construcción teórica de la \enquote{movilización política}, definida desde las sociologías de Mannheim y Germani, y más allá de los comentarios favorables a la teoría de la dependencia y el enfoque marxista, terminaría por desecharse en \enquote{Notas sobre las raíces histórico-estructurales de la movilización política en Chile} \parencite{1574-BORON1975}. Aquí el objetivo es criticar las interpretaciones \enquote{accidentalistas} del triunfo de la UP en 1970. Para Borón, el análisis marxista permite comprender este acontecimiento como parte del proceso histórico chileno. De lo contrario, \enquote{marginada de la totalidad histórico-estructural, la coyuntura se independiza de sus condicionamientos y sus determinaciones se diluyen haciéndose necesario recurrir a los eventos circunstanciales que la caracterizaron a fin de poder explicar su existencia misma} \parencite[68]{1574-BORON1975}.

La coyuntura de 1970 debe entones entenderse a partir de:

\begin{enumerate}
\item las contradicciones y conflictos generados por la industrialización, teniendo en cuenta su carácter dependiente y monopólico;
\item los cambios de la estructura de clases (constitución del proletariado urbano y rural, diferenciación de la burguesía, la expansión de las clases medias, cambios demográficos de la sociedad);
\item la ampliación de las bases sociales del Estado, la diversificación de alianzas de clase en su seno, sus ideologías y el carácter de la \enquote{movilización política} de las clases populares.
\end{enumerate}

Por esto, el concepto de \enquote{movilización política}

\begin{quote}
(\dots) se inserta y adquiere significado en el interior de la teoría marxista del Estado, especialmente tal como fue desarrollada en las obras de Antonio Gramsci. En esencia, aquel concepto representa la aparición de un nuevo sujeto histórico que irrumpe en la escena política y produce una ruptura crítica en la capacidad hegemónica de la clase dirigente. Se trata por lo tanto de una irrupción de las masas en el Estado burgués, de una insurgencia reveladora de una insostenible \enquote{presión desde abajo} que ya no se puede desbaratar con los métodos tradicionales de control político: \enquote{dirección intelectual y moral}, cooptación, exclusión o represión \parencite[71-72]{1574-BORON1975}.
\end{quote}

Por un lado, esto supone la aparición de un nuevo sujeto histórico que altera la correlación de fuerzas a partir de su integración al Estado, siendo protagonista activo de la lucha por sus intereses, y planteando una redefinición del carácter de clase del Estado, precipitando así una crisis de hegemonía. Por otro lado, esta nueva forma de entender la \enquote{movilización política} se relaciona íntimamente con la emergencia de un partido revolucionario y el desarrollo de organizaciones representativas de los intereses de las nuevas clases integradas.

En este sentido, la victoria de la UP, a diferencia de lo planteado por otras interpretaciones del período, no resulta accidental sino que se corresponde con la coyuntura política de la sociedad chilena de 1970. Esta última, tal y como es analizada por \textcite[75]{1574-BORON1975}, presentaba las características que Gramsci asignaba a las situaciones de crisis orgánicas y que, en última instancia, se resumen \enquote{en una ruptura en la relación entre representantes y representados, entre el Estado y la sociedad civil}. Esto significa que las clases subalternas \enquote{habían experimentado un proceso de movilización política a partir del cual estaban negando su subordinación a las clases dirigentes}. De esta manera, para Gramsci, la \enquote{movilización política} se explicaría por la irrupción de las masas; el desarrollo de organizaciones de clase (partidos y sindicatos) y por la crisis de hegemonía de las clases dominantes. Por lo tanto, la \enquote{movilización política} no debe reducirse a la sola extensión del sufragio, debiendo tener en cuenta estos tres elementos, modificándose así el prisma a partir del que Borón explicará el proceso chileno.

En síntesis, resulta meridianamente claro que a partir de 1975 Borón ya no se limitará a manifestarse favorablemente sobre el enfoque marxista y pasará a utilizarlo de forma productiva. De esta manera, la conceptualización de \enquote{movilización política}, originalmente planteada como equivalente del concepto mannheimiano de democratización fundamental, se desplazará hacia la teoría política gramsciana. En este sentido, la pregunta por el grado de organización de los nuevos contingentes incorporados a la política, es decir, si adhieren o no a partidos \enquote{de protesta}, ahora pasa a ser la pregunta por la creación de organizaciones de clase (sean partidos o sindicatos). Por último, los conceptos de la sociología de Germani, como masas en estado de disponibilidad o actitudes tradicionales, pierden peso en estas reflexiones. En los años posteriores esto ya no se modificará, dando lugar a una vasta obra enrolada en las filas del marxismo.

\section{Conclusiones}

La modernización social, económica y política experimentada por la sociedad chilena durante los años sesenta fue acompañada por un rol activo del Estado y de diversos organismos internacionales que financiaron el desarrollo de las ciencias sociales, convirtiendo a Santiago en un polo de atracción regional para los profesionales de estas disciplinas. Entre ellos se encuentran varios argentinos, no solo exiliados de la dictadura de Onganía, sino también jóvenes profesionales que, a través de la cooperación internacional, pudieron acceder a becas que les permitieron realizar sus primeros estudios de posgrado en FLACSO. En el caso de Atilio Borón, pudo aprovechar estas fuentes de financiamiento a partir de los contactos establecidos durante sus estudios de grado en Buenos Aires, por lo que estuvo en condiciones de instalarse en Santiago para cursar la recientemente creada Maestría en Ciencia Política y Administración de la ELACP.

Esta escuela, aunque originalmente tuvo como profesores a varios abogados que mantuvieron una orientación hacia la formación de técnicos o especialistas (en línea con el proyecto del BID), rápidamente incorporó a las primeras generaciones de graduados como docentes e investigadores. Entre ellos, se encontraba Borón quien, debido a su formación en la tradición de la sociología científica, tanto en la carrera de sociología de la UCA como en la UBA (donde asistía con regularidad a los cursos de Germani), comenzaría abordando la problemática de la \enquote{movilización política} en Chile desde de la teoría de la modernización en boga durante los primeros años de la década de 1960.

Esta primera etapa del pensamiento boroniano se liga al contexto histórico en se formó, es decir, cuando la adaptación latinoamericana que la sociología de la modernización hizo del estructural-funcionalismo norteamericano convirtió a la problemática del desarrollo en un tópico común para los científicos sociales de la región. De esta manera, no resulta extraño que el autor parta del concepto mannheimiano de democratización fundamental y que manifieste una clara influencia de Germani en relación a problemas como las masas en estado de disponibilidad o las actitudes tradicionales de los nuevos contingentes incorporados a la vida política.

Sin embargo, como se ha visto, Borón siempre se manifestó crítico del enfoque liberal y favorable a la teoría marxista, al análisis de las clases sociales, a la teoría de la dependencia e, incluso, al materialismo histórico.\footnote{Sobre esto último decía, por ejemplo, que \enquote{la base económica condiciona los movimientos de la superestructura política, y que lo que debe probarse entonces es el grado de ese condicionamiento y las mediaciones a través de las cuales ejerce su influencia} \parencite[228]{1573-BORON1972}.} No obstante, la aplicación efectiva de este paradigma deberá esperar hasta 1975 cuando, ya instalado en Estados Unidos, Borón diera a conocer su último artículo sobre la \enquote{movilización política} en Chile, esta vez definida no solo a partir de la irrupción de las masas y la creación de partidos \enquote{de protesta}, sino más específicamente \enquote{de clase} y como generadora de una crisis de hegemonía de las clases dominantes en el Estado burgués. Esta transformación, que continuará inmediatamente con la publicación de su célebre texto sobre el fascismo como categoría histórica \parencite{1575-BORON1977}, lo acompañará durante el resto de su trayectoria, hasta la actualidad.

%Capítulo 6
\chapter{Orden democrático, acción y decisión. Aportes de Emilio De Ípola a la teoría sociológica}

\footnote{\textsuperscript{*} Ponencia presentada en las I Jornadas de Sociología de la Universidad Nacional de Córdoba.}

\section{Introducción}

Emilio De Ípola fue uno de los profesores más importantes de la carrera de Sociología de la Universidad de Buenos Aires desde el retorno de Argentina al orden democrático-liberal en 1983. Aunque con una trayectoria relativamente conocida, no cuenta con trabajos de los historiadores de la sociología que presten atención a su itinerario intelectual en relación con la disciplina. Esto quizás se deba a que sus intervenciones más relevantes estén vinculadas al análisis del discurso, por obras célebres como \citetitle{1586-DEIPOLA1982} o \citetitle{1593-DEIPOLA2005}. En estos textos, y otros más recientes como \citetitle{1589-DEIPOLA2021}, aunque no faltan citas de Michel Foucault o Erving Goffman,\footnote{E, incluso, una taxonomía de los presos políticos de las cárceles argentinas durante la última dictadura.} las referencias más abundantes corresponden a pensadores como Louis Althusser, Claude Lévi-Strauss, Jacques Rancière o Tzvetan Todorov, además de que se entablan largas discusiones con intelectuales argentinos dedicados a la comprensión del populismo y el peronismo desde el análisis discursivo, como Eliseo Verón o Ernesto Laclau.\footnote{En este sentido, los trabajos más importantes de estos autores serían, respectivamente, \emph{Perón o Muerte. Los fundamentos discursivos del fenómeno peronista} (1986) (en coautoría con Silvia Sigal) y \emph{La razón populista} (2005), aunque De Ípola se refiere casi siempre a \emph{Política e ideología en la teoría marxista} (1978).}

Por cierto, además de los estudios sobre discursos \parencite{1586-DEIPOLA1982,1587-DEIPOLA1987,1588-DEIPOLA1989,1589-DEIPOLA2021,1590-DEIPOLA1982} que comprenden, por ejemplo, la crítica al populismo y a la teoría laclausiana \parencite{1539-PORTANTIERO1988,1586-DEIPOLA1982,1587-DEIPOLA1987,1588-DEIPOLA1989,1591-DEIPOLA2008} así como las observaciones en torno a los \enquote{rumores carcelarios}, a partir de su propia experiencia en prisión durante la última dictadura cívico-militar \parencite{1586-DEIPOLA1982,1592-DEIPOLA1997,1593-DEIPOLA2005,1589-DEIPOLA2021}, son temas recurrentes en sus trabajos: el itinerario intelectual de Althusser\footnote{A quien califica elogiosamente como \enquote{el primero y único marxista de alto nivel que Francia dio al mundo en toda su historia} \parencite[125]{1607-DEIPOLA2011}.} ---y las exégesis de discípulos como Jacques Derrida--- \parencite{1594-DEIPOLA1974,1595-DEIPOLA2007,1607-DEIPOLA2011}; los análisis en torno al socialismo y el peronismo \parencite{1586-DEIPOLA1982,1596-DEIPOLA1987,1588-DEIPOLA1989,1597-DEIPOLA1990}; las reflexiones en relación a aspectos epistemológicos de las ciencias sociales \parencite{1556-CASTELLS1975,1557-CASTELLS1973,1598-DEIPOLA1969}; y los estudios de algunas figuras relevantes del medio intelectual argentino, especialmente Tulio Halperin Donghi, Jorge Luis Borges y José Aricó (1989; 1997; 2001).\footnote{Debe tenerse en cuenta que buena parte de estos libros son compilaciones de artículos previamente dados a conocer en congresos o revistas científicas, por lo que el autor republica en varias oportunidades los mismos trabajos. De allí la reiteración de algunos de ellos en las citas de este texto.}

Ahora bien, ¿dónde puede encontrarse la teoría sociológica en una obra tan variopinta? En principio, debe decirse que De Ípola forma parte de una generación de intelectuales de izquierda que, durante las décadas de 1960 y 1970, fueron influidos por el estructuralismo althusseriano. De modo que, en esta época, pueden apreciarse sendos estudios dedicados al pensamiento de Karl Marx, como el seminario de \enquote{Sociología Sistemática} que dictó en 1972 en la sede de Santiago de Chile de la Facultad Latinoamericana de Ciencias Sociales (FLACSO), más tarde publicado como \emph{Discusiones sobre el materialismo histórico} \parencite{1599-DEIPOLA1974}, o bien la parte teórica de los gruesos volúmenes en coautoría con la demógrafa Susana Torrado, su primera esposa, titulados \emph{Teoría y método para el estudio de la estructura de clases sociales (con un análisis concreto: Chile, 1970)} \parencite{1505-TORRADO1976}.\footnote{Por cierto, Marx tampoco dejó de formar parte de sus trabajos, aunque la experiencia del exilio reorientó su perspectiva sobre el autor. Véase los capítulos que le fueron dedicados en \textcite{1592-DEIPOLA1997,1600-DEIPOLA2001}.}

Posteriormente, y en consonancia con la llamada \enquote{crisis del marxismo} \parencite{279-ANDERSON2011}, De Ípola sería uno de los tantos que abandonarían sus viejas convicciones y se convertirían a la socialdemocracia durante los años ochenta. Esto significa que, con el retorno de la democracia, su itinerario seguiría en buena medida el de otros intelectuales argentinos, como Aricó, Juan Carlos Portantiero, Oscar Terán, Carlos Altamirano, Beatriz Sarlo o Hugo Vezetti, quienes moderarían sus posturas previas \parencite{1446-BURGOS2004}. En efecto, algunos de los nombrados pasarían a formar parte de los \enquote{asesores informales} con los que contó el presidente Raúl Alfonsín, conformando el llamado Grupo Esmeralda. En particular, fueron De Ípola y Portantiero quienes escribieron el famoso \enquote{Discurso de Parque Norte}, pronunciado por el entonces primer mandatario \parencite{1525-RINESI2004,1601-DEIPOLA1994,1602-DEIPOLA2004}.

Fue durante estos años que De Ípola estableció una relación duradera con la sociología en un sentido más estricto, ya que comenzó a trabajar como profesor del área de Teoría Sociológica de la carrera de sociología de la Universidad de Buenos Aires en 1984. Es a partir de esta época que pueden encontrarse trabajos orientados hacia las problemáticas centrales de la teoría sociológica. En otras palabras, como dijera \textcite[18]{1516-ALEXANDER2008}, se trata del nivel de los \enquote{presupuestos} que, dentro los \enquote{elementos apriorísticos} (parte no empírica de la ciencia), representa el más general y con el que debe lidiar \enquote{cada sociólogo en su enfrentamiento con la realidad}. Por un lado, el orden social, aunque en De Ípola no será de cualquier tipo sino, específicamente, democrático \parencite{1540-PORTANTIERO1987,1592-DEIPOLA1997}, incorporando en sus análisis la teoría de Émile Durkheim. Por otro lado, la acción social, a partir de la puesta en evidencia de algunas insuficiencias en las conceptualizaciones clásicas de Max Weber y Talcott Parsons.\footnote{Alexander sostiene que la acción social puede ser instrumental (racional) o normativa (no racional) y el orden social \enquote{externo} o un producto de la negociación e interacción de los individuos. En \textcite{1584-COLLINS2000}, el cruce de estas variables, teniendo en cuenta además si el orden es conflictivo o armonioso, es lo que da lugar a las cuatro tradiciones sociológicas posclásicas: la durkheimiana o sistemática, del orden conflictivo, de la elección racional y las distintas variantes fenomenológicas.} Aquí De Ípola recuperará autores como Paul Ricoeur, Harold Garfinkel, Jacques Derrida, Niklas Luhmann y Francisco Naishtat, enlazando este concepto con los de sujeto y decisión \parencite{1600-DEIPOLA2001,1603-DEIPOLA2004}.

Este capítulo se enfoca entonces en indagar especialmente en estos últimos trabajos, con el objetivo de dar cuenta de las contribuciones de Emilio De Ípola a estos problemas teóricos. Esto significa prescindir en buena medida de su obra, aunque no desconocer su trayectoria, ya que es a partir de ella que puede explicarse tanto el por qué de la selección de sus problemáticas como el modo de abordarlos, es decir, la elección de autores, de perspectivas, la formulación de hipótesis, etcétera. Con estos propósitos, el capítulo se divide en tres partes: en la primera, se reconstruye el devenir de la vida del autor, rastreando sus orígenes sociales y espacios de socialización política e intelectual, cómo se vinculó con la sociología, qué autores llamaron su atención, etcétera; en la segunda parte, se abordan los principales trabajos de teoría sociológica de De Ípola, dando cuenta de sus conceptualizaciones en torno al orden social y la acción social; por último, las conclusiones retoman y sintetizan los aspectos más relevantes del escrito.

\section{La trayectoria social de Emilio De Ípola}

Emilio Rafael De Ípola nació el 1 de febrero de 1939 en la ciudad de Buenos Aires en una familia \enquote{imperfectamente católica}\footnote{Así lo dice el propio autor para referirse a que su padre, aunque era ateo, respetaba a la Iglesia como garante de la moral pública y privada, y su madre, a la inversa, era una ferviente creyente pero detestaba a los curas, los sacramentos y la asistencia dominical a misa. Emilio, por su parte, se fue alejando paulatinamente de la institución hasta que llegó a la conclusión de que había perdido la fe, poco antes de terminar la escuela.} del barrio de Belgrano. Cursó estudios secundarios entre 1952 y 1956 en el Colegio Nacional n.° 8 Julio A. Roca, cuando sus \enquote{lecturas desordenadas} desencadenaron el surgimiento de \enquote{ciertas inquietudes culturales o, más específicamente, filosóficas} \parencite{1604-DEIPOLA2009}. Esto lo llevo a inscribirse, al año siguiente, en la Licenciatura en Filosofía de la Facultad de Filosofía y Letras de la Universidad de Buenos Aires, de la cual egresaría en 1964. De Ípola tuvo una importante actividad política durante estos años en la Federación Juvenil Comunista, llegando a ser consejero estudiantil, presidente del Centro de Estudiantes de la facultad y presidente la Federación Universitaria de Buenos Aires, aunque se desafiliaría de \enquote{la Fede} antes de concluir sus estudios.

Gracias a una beca de posgrado pudo radicarse en Francia en 1965, donde realizaría el Doctorado de Estado en la Faculté des Lettres et Sciences Humaines de Nanterre, defendiendo su tesis en 1969, la cual estuvo bajo dirección de Henri Lefebvre.\footnote{En un principio la dirección de tesis estuvo a cargo de Lucien Goldmann, pero al cambiar el estatus de la misma (de tesis de tercer ciclo a tesis de Estado), se vio imposibilitado de dirigirla.} Comenta que se vio desilusionado por varios de los cursos que se dictaban en París, como el de Claude Lévi-Strauss y el de Lucien Goldmann, aunque hubo otros que atrajeron poderosamente su atención como, por ejemplo, el de Roland Barthes. Sin embargo, fue Louis Althusser quien lo influyó al punto de afirmar que \enquote{fue mi Gramsci, [ya que] me sirvió --como a mis amigos gramscianos-- para salir del marxismo dogmático} \parencite[22]{1605-DEIPOLA1991}. De hecho, De Ípola formaría parte de un grupo de estudiantes latinoamericanos y franceses que se reunían periódicamente para leer su obra \parencite{1601-DEIPOLA1994,1595-DEIPOLA2007}.

En cuanto al ámbito docente, De Ípola ya había trabajado como ayudante en la cátedra de Ética de la Facultad de Filosofía y Letras de Buenos Aires entre 1962 y 1964, es decir, antes de radicarse en París. Hacia 1967, frente a la imposibilidad de retornar a Argentina debido al golpe de Estado de 1966, y gracias a la recomendación de Goldmann, conseguiría un puesto como profesor del Departamento de Sociología de la Universidad de Montreal, Canadá. No obstante, por intermedio de su amigo Manuel Castells, se le ofrecería ser coordinador docente de la Escuela Latinoamericana de Sociología de FLACSO, en la sede de Santiago de Chile, por lo que abandonaría Québec en 1972.\footnote{\textcite{1606-DEIPOLA2010} había dado algunas clases en el Instituto Di Tella en 1971, pero sus obligaciones en Chile le impidieron continuarlas y fue reemplazado por Félix Schuster.} La estadía en Santiago duró poco tiempo, debido al golpe de Estado del 11 de septiembre de 1973, aunque De Ípola seguiría siendo profesor-investigador de FLACSO hasta 1984 (con intermitencias), en las sedes de Buenos Aires y México.

Como comentó en varias oportunidades \parencite{1586-DEIPOLA1982,1601-DEIPOLA1994,1593-DEIPOLA2005,1589-DEIPOLA2021}, estando en la capital argentina, en abril de 1976, sufrió el secuestro y las torturas por parte de un grupo de tareas de la última dictadura cívico-militar (1976-1983), de modo que debió permanecer durante casi dos años como preso político hasta que le fue otorgada la posibilidad de exiliarse en Francia, aunque al poco tiempo decidió radicarse en México, donde se encontraban la mayor parte de los exiliados argentinos. Allí tomaría contacto con el grupo de la revista \emph{Controversia} donde, en línea con la revisión de las viejas tesis marxistas, daría a conocer sus primeros trabajos sobre el peronismo y el populismo \highlight{(Portantiero y De Ípola, 1979) NO LA ENCUENTRO}, iniciando una línea de investigación sobre la política que continuaría en Buenos Aires \parencite{1588-DEIPOLA1989}. Con el retorno de la democracia a Argentina, como se ha dicho, De Ípola y otros destacados intelectuales de izquierda convertidos a la socialdemocracia, pasarían a formar parte del Grupo Esmeralda, el cual fue impulsado por el empresario Meyer Goodbar, dueño de la firma \emph{Executives}, para asesorar al presidente Alfonsín.

El retorno a Argentina también significó, en el plano académico, que De Ípola se integrara a instituciones como el Centro Latinoamericano para el Análisis de la Democracia (CLADE), el Comité Editorial de la revista \emph{La ciudad Futura} del Club de Cultura Socialista y el Consejo de Dirección de la revista \emph{Sociedad} de la Facultad de Ciencias Sociales de la Universidad de Buenos Aires. En esta facultad sería profesor desde 1984 del área de Teoría Sociológica en las asignaturas \enquote{Introducción a la Sociología}, \enquote{Sociología Sistemática} y \enquote{La naturaleza del lazo social: de Durkheim a Touraine}. Además, en 1988 fue Directeur d'Etudes Associé de la École des Hautes Études en Sciences Sociales de París, donde dictó varios seminarios sobre la situación sociopolítica argentina y sobre problemas de cultura e ideología.

De Ípola también estuvo a cargo de varios cursos de posgrado, no solo en instituciones de Argentina (maestrías en Ciencias Sociales de FLACSO; en Investigación en Ciencias Sociales de la Universidad de Buenos Aires; y en Sociosemiótica del Centro de Estudios Avanzados de la Universidad Nacional de Córdoba; doctorado en Ciencias sociales FLACSO / Universidad Nacional de Rosario; también en el Instituto de Desarrollo Económico y Social; en el Centro de Estudios del Estado y la Sociedad; y en la Facultad de Ciencias Sociales de la Universidad Nacional de Santiago del Estero), sino también de Brasil (doctorado de FLACSO / Universidad Nacional de Brasilia), Bolivia (maestría en Ciencias Sociales de FLACSO), México (maestrías en Sociología de la Universidad Iberoamericana; en Ciencias Políticas de la Universidad Nacional Autónoma; y de la Universidad Autónoma Metropolitana de México; y el doctorado en Lingüística de El Colegio de México), y Venezuela (maestría en Sociología de la Universidad del Zulía, Maracaibo).

Además, De Ípola llegó a ser investigador principal del Consejo Nacional de Investigaciones Científicas y Técnicas (CONICET) de Argentina, miembro de la Comisión Asesora de Economía, Sociología, Ciencia Política y Derecho de la misma institución (1985-1986 y 2000-2001) y del Comité de Evaluadores de la Comisión Nacional de Evaluación y Acreditación Universitaria (CONEAU). A su vez, en términos de gestión, cabe mencionar que formó parte del Consejo Directivo de la Facultad de Ciencias Sociales de la Universidad de Buenos Aires entre 1994 y 1998, de la Comisión de Doctorado de la Facultad de Ciencias Sociales y del Comité Asesor y del cuerpo profesoral del doctorado en Análisis del Discurso del Centro de Estudios Avanzados de la Universidad Nacional de Córdoba y ejerció la dirección del Centro de Estudios Avanzados de la Universidad de Buenos Aires entre 1990 y 1993. Por último, recibió varios reconocimientos y distinciones, como las becas Thalmann y Guggenheim en 1996 y 2004, respectivamente, el Premio Konex en Sociología en 2006, y el Premio Houssay a la trayectoria 2009, otorgado por el Ministerio de Ciencia y Tecnología de la Nación Argentina \parencite{1608-DEIPOLA2010}.

En síntesis, puede apreciarse una trayectoria sumamente exitosa, tanto en términos de docencia como de producción intelectual. Con profundos lazos con el pensamiento francés, particularmente el estructuralismo de Lévi-Strauss y Althusser, se observa en De Ípola un marcado interés por las cuestiones vinculadas al análisis del discurso. En este derrotero, si bien la sociología ingresa más o menos tempranamente, a partir del cargo docente en el Departamento de Sociología de la Universidad de Montreal y, poco después, en FLACSO, lo cierto es que el clima de la década de 1960 ligado a la radicalización política impactó claramente en las primeras intervenciones del autor, las cuales fueron realizadas desde y sobre el marxismo. Teniendo en cuenta el objetivo del capítulo, se abordarán entonces los textos de teoría sociológica producidos por De Ípola desde su retorno del exilio mexicano.

\section{Emilio De Ípola y la teoría sociológica I: el orden social democrático}

En el contexto de la \enquote{crisis del marxismo} de la segunda mitad de la década de 1970, los intelectuales argentinos exiliados en México comenzaron a pensar los procesos de transición a la democracia. Esto implicó que incorporaran a sus reflexiones buena parte de la cosmovisión de la tradición política liberal, adoptando así una posición en relación a la democracia que posteriormente sería criticada por \enquote{procedimental}.\footnote{De Ípola siempre fue consciente de las críticas que le realizaban, aunque se defendía diciendo que si, efectivamente, poseía \enquote{una idea procedimentalista de democracia --que nos endilgaron a Juan Carlos Portantiero y a mí (\dots.) eso no agotaba para nosotros el contenido de la democracia} \parencite[101]{1525-RINESI2004}. Lo importante era que las reglas que debían cumplirse estaban fundadas en ciertos valores y la promoción de esos valores era lo esencial en el sistema democrático.} Esto quiere decir que la democracia presupone, en principio, una serie de \enquote{reglas de juego} que deberían ser respetadas. En este sentido, se trata de un \emph{pacto social} para la construcción de un determinado \emph{orden social} (democrático), cuya violación supondría indefectiblemente el retorno del autoritarismo. Va de suyo entonces que los filósofos contractualistas comenzaron a formar parte insoslayable del itinerario de lecturas de los sociólogos argentinos miembros de la \enquote{transitología} (Lesgart, 2003) que se consagraron en el espacio público durante la época posdictatorial.

Esto puede apreciarse con toda claridad en la \enquote{Introducción} de Portantiero y De Ípola a la antología \citetitle{1540-PORTANTIERO1987} (1987). Allí, los autores señalan que los conceptos de Estado y sociedad son recíprocos y complementarios en el pensamiento moderno de Occidente, en tanto resultan \enquote{lo opuesto al estado de naturaleza} \parencite[7]{1540-PORTANTIERO1987}. En este sentido, el pacto social fue una argumentación que sirvió tanto al absolutismo de Thomas Hobbes y al liberalismo de John Locke como a la democracia, de acuerdo a la propuesta de Jean-Jacques Rousseau. Así,

\begin{quote}
Rousseau es el primer pensador moderno que plantea a la democracia directa como forma de articulación entre sociedad y Estado, abriendo una tradición que el socialismo recuperará en el siglo~XIX y que se desarrollará en los planteos políticos del \enquote{consejismo} de Lenin y Gramsci a principios de este siglo \parencite[10]{1540-PORTANTIERO1987}.
\end{quote}

Sin embargo, como la versión \enquote{marxista-leninista} del socialismo había sido dejada de lado por tratarse de una cosmovisión autoritaria de la política, se optará por la tradición sociológica que nació, precisamente, a partir de la preocupación por la recomposición del orden social luego del \enquote{desorden} provocado por la Revolución Francesa. Primero expresada en la filosofía positivista de Auguste Comte, y reformulada más tarde bajo el prisma durkheimiano, la indagación sobre la naturaleza y el papel del lazo social (acompañada de toda una batería de conceptos bien conocidos por los sociólogos, como conciencia colectiva, solidaridad, división del trabajo social, etcétera), pero también, en una línea no tan célebre pero no por ello menos importante, sobre el Estado como \enquote{órgano} responsable de la creación de las \enquote{representaciones colectivas}, serán algunas de las problemáticas más importantes de la época.

De este modo, Portantiero y De Ípola recuperan el proyecto de Durkheim que, enraizado en el contexto de la Tercera República Francesa, se proponía la construcción de una república liberal, laica y respetuosa de los derechos individuales. Cabe entonces preguntarse, desde la perspectiva durkheimiana, ¿cuándo el Estado y la sociedad son democráticos? Aquí la respuesta no da lugar a dobles interpretaciones. Si, como se dijo, Durkheim entiende al Estado como \enquote{el órgano del pensamiento social} encargado de producir las \enquote{representaciones colectivas}, entonces este será más democrático cuanto más estrecha sea su comunicación con las \enquote{conciencias individuales}.

\begin{quote}
Democracia, pues, significa la posibilidad de comunicación entre esas dos esferas del saber y del sentir: el especializado y el difuso. No se trata de que todo el mundo gobierne o que se llegue a una sociedad política sin Estado para hablar de democracia. Se trata de que el poder gubernamental, en lugar de replegarse sobre sí mismo, esté en permanente contacto con las capas profundas de la sociedad, reciba respuestas y reelabore así sus decisiones. Cuanto más sólida y fluida sea la comunicación entre esos dos registros del Estado y la conciencia colectiva y, por lo tanto, cuanto más central sea el papel de la reflexión crítica en la gestión de los asuntos públicos, más democrática será la sociedad \parencite[19]{1540-PORTANTIERO1987}.
\end{quote}

Sin embargo, queda por dilucidar el concepto de sociedad y responder, desde la óptica de Durkheim, bajo qué condiciones es democrática. Al respecto, \textcite[21]{1592-DEIPOLA1997} ensaya una reconstrucción de tres perspectivas posibles sobre el objeto de la sociología, entendiendo que esta disciplina nace \enquote{a la vez como teoría del lazo social y como comentario desolado por su disolución}.\footnote{En esta época, \textcite{1609-DEIPOLA1998} compiló un libro con el mismo espíritu durkheimiano que expresa la frase. En este trabajo se aborda el surgimiento de la \enquote{nueva cuestión social}, como producto de las reformas neoliberales de la década de 1990, siendo las nuevas sociedades caracterizadas como de la \enquote{desafiliación social}, según la conceptualización de Castells.} La primera es la de Louis de Bonald quien, desde el conservadurismo, denuncia su desaparición y reclama un retorno a las viejas sociedades del \emph{Ancien Régime}. La segunda, de Gustav Le Bon, se denomina \enquote{autoritaria y policial} en tanto llama la atención por la peligrosidad de los agregados sociales compuestos de \enquote{masas y cabecillas}. Por último, la apuesta de Durkheim es \enquote{posrevolucionaria y republicana}, ya que aborda \enquote{la problemática relación entre la \enquote{cuestión social} y la \enquote{cuestión democrática}} \parencite[36]{1592-DEIPOLA1997}.

Como se sabe, Durkheim fue un autor preocupado por el incremento de la tasa de suicidios, por el auge de los conflictos laborales, el antisemitismo y la intolerancia religiosa, es decir, \enquote{problemas que hacen a la construcción de un orden en una sociedad --Durkheim no lo ignora-- irreversiblemente moderna} \parencite[36]{1592-DEIPOLA1997}. Frente a esta creciente conflictualidad social, los liberales, los marxistas y los conservadores plantean soluciones de diversa índole que son desechadas por Durkheim, quien contrapone mayores niveles de reglamentación de la vida social, elevando así la regulación de la sociedad sobre los conflictos, con el objetivo de evitar la tan temida anomia, patología del mundo moderno.

En este punto, De Ípola reconoce que hay dos posiciones que, sin llegar a enfrentarse, conviven en las conceptualizaciones durkheimianas en relación a lo social. Por un lado, la que se manifiesta en sus dos primeras obras, \emph{La división del trabajo social} (1893) y \emph{Las reglas del método sociológico} (1895), donde el autor retoma de forma intermitente una polémica con Auguste Comte y Herbert Spencer sobre los rasgos que permitirían pensar a la sociedad como un organismo, poniendo énfasis en el carácter exterior y objetivo del hecho social. Por otro lado, frente al Durkheim objetivista, aparece una segunda cosmovisión en \emph{El suicidio} (1897) y \emph{Las formas elementales de la vida religiosa} (1912), donde postula la naturaleza psíquica de lo social.\footnote{Recuérdese la famosa frase de \emph{El suicidio}, según la cual la vida social \enquote{está hecha esencialmente de representaciones} \parencite[429]{1617-DURKHEIM2006}.} Es entonces el movimiento \enquote{entre la estructura y la representación, lo objetivo y lo subjetivo} lo que marca \enquote{silenciosamente su obra}\footnote{Esta antinomia también es señalada por \textcite{1618-DEIPOLA2012} en la \enquote{Introducción} que redactó a \emph{Las reglas del método sociológico}. Por cierto, esta no es una lectura para nada original, ya que se encuentra en otros intérpretes del sociólogo francés, como por ejemplo Duncan Mitchell (1973).} \parencite[44]{1592-DEIPOLA1997}.

Ahora bien, más allá los desplazamientos internos de la obra en relación a \enquote{lo social}, interesa resaltar que, para Durkheim,

\begin{quote}
(\dots) hasta tanto los valores de la ciencia y los de la democracia liberal se enraizaran en configuraciones sociales tan sólidas y cohesionantes como aquellas de antaño fundadas en los pilares de la religión y la familia, y estuvieran imbuidos del respeto moral de que esas instituciones gozaron entonces, Francia y Europa persistirían en su actual situación de crisis, sepultando una a una todas las soluciones políticas que los reformadores propusieran \parencite[46]{1592-DEIPOLA1997}.
\end{quote}

De modo que, para construir un orden social democrático, no solo se debe contar con un Estado con elevados niveles de comunicación con los individuos para la reelaboración de sus decisiones, sino que también es necesaria la difusión y consolidación de valores democráticos entre los últimos. Por cierto, lo que resulta meridianamente claro es que ambas cuestiones son de interés primordial para De Ípola en el contexto de la transición argentina a la democracia. Además, es interesante señalar que en Argentina es la primera vez que se recupera esta versión de Durkheim, cuyas consideraciones son tomadas del curso póstumo \emph{Lecciones de sociología. Física de las costumbres y del derecho.}\footnote{Se trata de una serie de cursos que Durkheim dictó en Burdeos durante la década de 1890, de los cuales Marcel Mauss dio a conocer solo algunos capítulos en 1937 \parencite{1443-BOUGLE1938}. La obra en francés sería publicada en 1950 y su primera traducción al castellano data de 1966 \parencite{1449-CATANO1998}.}

\section{Emilio De Ípola y la teoría sociológica II: acción y decisión}

Una vez planteadas las cuestiones en relación al orden social, no se observa que De Ípola se haya mostrado interesado en la teoría durkheimiana para pensar la acción social.\footnote{Cuestión que, por cierto, fue un tema descuidado por la teoría sociológica en general. Al respecto, véase Lorenc Valcarce (2014).} En este aspecto, los puntos de partida son las obras de Max Weber y Talcott Parsons, siendo \emph{Metáforas de la política} (2001) no solo su aporte más importante a la teoría de la acción sino a la teoría sociológica en su conjunto. El título del trabajo remite a dos metáforas \enquote{fundantes} de la política que, en alguna medida, atraviesan los capítulos que componen el libro. En un sentido débil, la política puede ser entendida como \enquote{subsistema} (en la semántica de Luhmann) o \enquote{superestructura} (en la de Marx) pero, en cualquier de los dos casos, con sus causas y efectos predeterminados. En un sentido fuerte, que es el que más interesa al autor, y el que mejor se explota en relación a la noción de \emph{decisión}, aparece una dimensión de contingencia en la política que posibilita la intervención (individual o colectiva) y, por lo tanto, la modificación del mundo social.

En el primer capítulo se comienza señalando que la acción social fue una preocupación de primer orden para la teoría sociológica pero, paulatinamente, fue quedando al margen de las inquietudes de las ciencias sociales. Recién hacia mediados de la década de 1980 volvió a adquirir un papel significativo, que se manifestó con toda claridad durante el decenio siguiente, en consonancia con la llamada \enquote{crisis de los grandes relatos}. Para llegar a este momento, se empieza por reconocer los méritos y falencias de los aportes seminales de Weber y Parsons en \emph{Economía y sociedad} (1922) y \emph{La estructura de la acción social} (1937), respectivamente. Entre los primeros se encuentra la \enquote{indeterminación situacional}, es decir, el rechazo a los esquemas estímulo-respuesta conductistas o, en otras palabras, la apertura a la relación de incertidumbre entre la acción y su entorno. Entre las segundas, aparece la ligazón de la acción al \enquote{sentido} o \enquote{significado} de la misma, ya que los motivos, intenciones, deseos, disposiciones, fines, etcétera, de los actores, como tantas veces se ha señalado, remiten a \enquote{estados internos} no observables empíricamente. Como se sabe, fue el método de la \emph{Verstehen}, del cual Parsons se desembarazó rápidamente, la solución planteada a esta cuestión.

En el segundo capítulo, De Ípola comenta que entre las teorizaciones posteriores sobre la acción social persistieron una serie de problemas sin resolver, que clasifica en tres registros: el problema (epistemológico) del punto de vista; el problema (metodológico) del acceso a los estados internos; y el problema (ontológico) de la incidencia de los objetos en el curso de la acción. Respecto al primero, a partir de los aportes de Paul Ricoeur y Harold Garfinkel, se recuperan las dimensiones discursivas y representacionales como partes integrantes de la acción. Según el filósofo francés, existe un lenguaje específico de la acción, es decir, una \emph{semántica natural de la acción} que da sentido y, por lo tanto, contribuye a constituir la acción como tal. Este lenguaje se compone de una serie de nociones como motivos, intenciones, deseos, deliberaciones, etcétera, y, a su vez, está organizado en redes semánticamente conectadas.

\begin{quote}
Todo ello habilita a dicho lenguaje para: a) describir, interpretar y explicar acciones individuales y --eventualmente- colectivas; b) permitir al observador de la acción plantear un conjunto de preguntas a los actores: quién ha hecho qué cosa, bajo qué circunstancias actuó, qué resultados esperaba obtener, etcétera. (\dots) Por último (c), (\dots) proporciona instrumentos para construir teorías generales de la acción (\dots) [a] las ciencias sociales (\dots) \parencite[53]{1600-DEIPOLA2001}.
\end{quote}

Por su parte, la etnometología de Garfinkel aborda la relación entre el lenguaje y la acción cuestionando, a través de diversos experimentos sociales, el mundo de sentido común de los actores. Al igual que para Ricoeur, la relación lenguaje-acción resulta aquí constitutiva. De modo que un enunciado no solo vehiculiza cierta información, sino que además genera un contexto a través del cual esta puede adquirir un sentido determinado. \enquote{Para Garfinkel, esta dimensión reflexiva de las prácticas cotidianas configura uno de los aspectos de mayor relevancia del conocimiento de sentido común} \parencite[41]{1600-DEIPOLA2001}. Más allá de las diferencias entre ambos autores, lo importante para De Ípola es subrayar el papel excluyente que desempeñan las creencias de los actores en sus acciones.\footnote{Esto aparece resaltado en varios trabajos del autor, especialmente en sus estudios sobre las creencias de la sociedad argentina en la década de 1980 en relación a la efectividad de la crotoxina, una droga que supuestamente curaba el cáncer \parencite{1592-DEIPOLA1997,1610-DEIPOLA2002}.}

El segundo problema es quizás el más importante \enquote{que la teorización clásica de la acción dejó sin resolver} \parencite[47]{1600-DEIPOLA2001}. Aquí, De Ípola ligará claramente la cuestión de los \enquote{estados internos} con la del \enquote{punto de vista}, argumentando que la semántica natural de la acción es el único lenguaje disponible al que los actores recurren espontáneamente para \enquote{articular coherentemente los criterios y los índices habitualmente utilizados para detectar una intención (o su ausencia) con la situación que dio lugar a su decisión} \parencite[50]{1600-DEIPOLA2001}. En este punto, se reconocen las limitaciones de las propuestas de Ricoeur y Garfinkel, concluyendo que \enquote{los criterios efectivos a que recurre el actor con vistas a su acción (\dots) permanecen en lo esencial inaccesibles} \parencite[52]{1600-DEIPOLA2001}.

Finalmente, para el tercer problema se retoma la teoría de Laurent Thévenot, quien otorga gravitación a los objetos materiales que forman parte de los cursos de acción. Este autor

\begin{quote}
(\dots) insiste en el hecho de que las personas deben apoyarse sobre objetos para hacer valer la pertinencia de su argumentación. Únicamente el apoyo sobre un mundo común y, por tanto, sobre la objetividad de lo que existe entre las personas, permite a estas últimas mostrar que sus pretensiones no son arbitrarias y que ellas están dispuestas a inclinarse ante una realidad válida para todos \parencite[52]{1600-DEIPOLA2001}.
\end{quote}

Por lo tanto, el mundo social no está hecho solo de palabras y creencias, sino también de acciones materiales mediadas por realidades no menos materiales, que pueden servir como escollos o como recursos para la acción en cuestión, pero sobre los cuales ni Ricoeur ni Garfinkel repararon. Este capítulo se cierra con la crítica de Luhmann a la concepción de sujeto presente en la teoría sociológica tradicional, es decir, un actor portador de una serie de atributos como libertad, propósitos, motivos, adaptación al medio social, etcétera. Esta crítica, más allá de las reservas generales que mantiene respecto de la teoría de los sistemas, es compartida por De Ípola. Sin embargo, lo importante de la alusión a Luhmann es que prepara el terreno para el tercer capítulo, donde se encuentra su aporte más relevante a la teoría de la acción\footnote{Los siguientes capítulos no revisten la misma importancia que los dos primeros, por lo que serán dejados de lado. En síntesis, en el capítulo 4 se hace una evaluación de la obra de Luhmann, en el capítulo 5 se analizan algunos textos de Marx (en particular \emph{El 18 Brumario de Luis Bonaparte}) a la luz de las mentadas metáforas \enquote{fundantes} de la política y, finalmente, en el capítulo 6 se abordan las relaciones entre \enquote{acción y representación} en la obra de Tulio Halperin Donghi.}: la noción de \emph{decisión}.

Al respecto, se arguye en línea con el pensamiento filosófico y político post-marxista y post-estructuralista, que \emph{el sujeto se constituye en el acto mismo de la} \emph{decisión}. Para muchos autores, la decisión debería subsumirse al concepto de acción, considerándola una forma particular de esta. Sin embargo, esto ya resulta objeto de polémica porque en otras formulaciones, como la teoría de los sistemas de Luhmann, se le quita al concepto de acción casi toda su relevancia teórica, separándola de la decisión y, por lo tanto, haciendo de esta última \enquote{un tipo particular de comunicación que, dado el funcionamiento atuorreferencial y autopoiético del sistema social, solo puede dar lugar a nuevas comunicaciones y así indefinidamente} \parencite[71]{1600-DEIPOLA2001}.

Para De Ípola, el problema de Luhmann consiste en referirse al concepto de acción en los términos clásicos de Weber y Parsons, ignorando las líneas de investigación sobre la acción social desarrolladas en Francia\footnote{Fundamentalmente aquellas plasmadas en la revista \emph{Raisons Pratiques}.} y, en particular, los aportes de Ricoeur. En este sentido, la decisión puede ser comprendida como un acto como cualquier otro. Sin embargo, y esta es la hipótesis general de De Ípola, \emph{la decisión es un acto inherente a la política} (en el sentido fuerte arriba señalado), rompiendo de este modo con el sentido que posee la \emph{decisión} para la \emph{semántica natural de la acción}.

Esto se explica por las siguientes subhipótesis: la \emph{decisión} específicamente política no es el producto de una deliberación y, si esta última interviene, supone una interrupción que, a diferencia de la \emph{semántica natural de la acción}, es decisiva en lo que concierne al ejercicio de la \emph{decisión}; en línea con Derrida, hay una urgencia inherente a la \emph{decisión} que la aproxima al acto performativo o ilocucionario; la \emph{decisión política} no puede ser asimilada a una entre varias alternativas,\footnote{A diferencia de Luhmann, para quien la \emph{decisión} supone precisamente la elección \enquote{entre un conjunto finito de posibilidades alternativas, que luego se patentiza en la alternativa elegida} \parencite[71]{1600-DEIPOLA2001}.} siendo en ese sentido instituyente; las \emph{decisiones políticas} son por definición paradójicas, es decir, sus condiciones de posibilidad son al mismo tiempo sus condiciones de imposibilidad; el \enquote{momento imposible} (Slavoj Zizek) de la \emph{decisión} pone al actor ante la responsabilidad de \enquote{decidir sin garantías}; por último, las constricciones del lenguaje y del hacer de la vida cotidiana --que la \emph{semántica natural de la acción} registra-- no afecta a las \emph{decisiones políticas}.

Una última observación sobre el/los sujeto/s y la decisión: para De Ípola es posible la constitución de \enquote{sujetos colectivos}.\footnote{Un desarrollo sobre la emergencia de nuevos sujetos colectivos en el contexto de las protestas piqueteras y las asambleas vecinales realizadas como respuesta a la crisis del 2001 se encuentra en \textcite{1611-DEIPOLA2004}.} Esta afirmación, en línea con lo dicho hasta el momento, se sustenta en la observación de Francisco Naishtat, según la cual

\begin{quote}
(\dots) la declaración de decisión colectiva sanciona la emergencia de una figura nueva, a saber, el sujeto del pacto, el cual surge como efecto de la enunciación (\dots): el colectivo no precede al pacto ni el pacto puede preceder al colectivo: hay emergencia recíproca de ambas figuras (citado en \cite[83]{1600-DEIPOLA2001}).\footnote{Esta cuestión de la constitución de sujetos colectivos en Naishtat también es subrayada en el análisis que \highlight{De Ípola (2004d) NO LO ENCUENTRO} hace del conflicto de la Facultad de Ciencias Sociales a comienzos de este siglo. También aparece en el prólogo a \emph{Problemas filosóficos de la acción individual y colectiva: una perspectiva pragmática} \parencite{1612-DEIPOLA2005}, donde se enfatiza en la \emph{acción colectiva pública} de esos sujetos y, por lo tanto, en la necesidad de una ética específica de este tipo de acción.}
\end{quote}

En síntesis, De Ípola introduce la dimensión discursiva en su conceptualización de la acción, a partir de la articulación de un conjunto de autores que aportan conceptos provenientes de la filosofía del lenguaje, la etnometodología y el pragmatismo, enriqueciendo de esta manera la teoría clásica de la acción social. En una línea genealógica remontable hasta el \enquote{segundo Wittgenstein}, es decir, el de \enquote{los juegos del lenguaje}, De Ípola se apoya en la \emph{semántica natural de la acción} y el conocimiento que los actores poseen de su situación, teniendo en cuenta además la materialidad del mundo social y sus objetos, para enlazar la acción con la noción de \emph{decisión}. Esta última, al mismo tiempo, se liga a la política (en sentido fuerte) devolviendo de esta forma al sujeto de la teoría sociológica la potestad de intervenir en la realidad y modificarla.

\section{Conclusiones}

La trayectoria de Emilio De Ípola posee tantas dimensiones de análisis posibles como su propia obra. En este trabajo se ha elegido hacer foco en sus aportes a las problemáticas centrales de la teoría sociológica, es decir, aquello que constituye lo que Alexander llamó el nivel de las \enquote{presuposiciones}: el orden social y la acción social. Esto no quiere decir que no se haya interesado por otros campos de la sociología, sino que sus contribuciones fueron particularmente importantes en relación a estas cuestiones.

Como se ha visto, los autores seleccionados provinieron en su gran mayoría del medio académico francés, lo cual se explica por su formación de posgrado en Nanterre así como por los vínculos intelectuales establecidos en Francia (además de la provincia francoparlante de Québec). Por este motivo, no llama la atención que su itinerario de lecturas haya incluido a De Bonald, Comte, Le Bon, Durkheim, Althusser, Ricoeur, Derrida, Foucault, Touraine o Thévenot aunque, por supuesto, esto no lo privaba de conocer en profundidad las obras de Marx, Weber, Parsons, Goffmann, Garfinkel o Luhmann. No obstante, lo que sí marca es una clara tendencia o, mejor, preferencia, que se manifiesta claramente al momento de abordar problemas teóricos.

En el caso del orden social, fue especialmente el Durkheim de \emph{Lecciones de Sociología}, es decir, un texto poco estudiado hasta la década de 1980 por los sociólogos argentinos, que De Ípola se propuso pensar las condiciones en las que un Estado y una sociedad adquieren un status democrático. De esta manera, no son solo los vasos comunicantes entre la sociedad y el Estado, para la reelaboración de decisiones por parte de este último, sino también, y fundamentalmente, la encarnación por parte de los individuos de determinados valores vinculados con la tolerancia a las diferencias políticas y religiosas, el laicismo, el respeto a las normas, la libertad de expresión, etcétera, las que hacen a una sociedad democrática. Si se quiere, al igual que lo dicho en el capítulo 4 en relación a las posiciones socialdemócratas de Torcuato Di Tella, en términos bourdieusianos lo que De Ípola reclama es la constitución de \emph{habitus democráticos} por parte de los miembros de la sociedad argentina.

En relación a la acción social (y los sujetos que las realizan), son fundamentales los aportes de Ricoeur, Garfinkel, Thévenot y Naishtat (aunque también una lectura crítica de Luhmann) ya que, para De Ípola, enriquecen las teorizaciones clásicas de Weber y Parsons. Por un lado, la \emph{semántica natural de la acción} y, por el otro, los conocimientos que los sujetos poseen de la situación, son los que permiten narrar sus acciones\footnote{Siendo esto es equivalente a lo que Giddens llamó \enquote{conciencia discursiva}.}, volviendo de este modo al discurso parte constitutiva de la acción. Al mismo tiempo, se recuerda que el mundo social no se compone solo de discursos y creencias, sino que los objetos que lo integran poseen una \emph{materialidad} que incide en los cursos de acción y que, a su vez, los sujetos colectivos se conforman a partir de su enunciación, siendo esta la que los constituye como tal.

Finalmente, aparece la noción de \emph{decisión} como aquella forma específica de acción que es inherente a la política en sentido fuerte. Esto es, los actores que hacen política no hacen otra cosa que decidir, y en esa decisión se manifiestan sus capacidades para modificar la situación en la que se encuentran. La política, por lo tanto, no es solamente un derivado de la estructura material de la sociedad, no es una \enquote{superestructura} (en la jerga marxista) o un subsistema (según Luhmann), cuyo origen pueda ser imputado causalmente a otra instancia mundo social como, por ejemplo, la base económica. Para De Ípola, la política es acción, decisión y, en definitiva, la posibilidad de cambiar el mundo.

\backmatter

\chapter[Epílogo]{Epílogo}
\chaptermark{Epílogo}
\Author{Diego Ezequiel Pereyra}

Diego Ezequiel Pereyra

\Footnote{*}{Instituto de Investigaciones Gino Germani, Universidad de Buenos Aires-CONICET/Universidad Nacional de Lanús.}

\section*{Vidas paralelas y perpendiculares en la sociología argentina}

El proceso de institucionalización de la sociología en Argentina se ha consolidado en las últimas tres décadas (Pereyra, 2010, 2017). Si bien se observa un campo institucional e intelectual muy fragmentado, se ha logrado en el largo plazo una estabilidad de la enseñanza de grado y posgrado. Asimismo, se ha consolidado la participación de los sociólogos y sociólogas tanto en el espacio académico como en ámbitos privados pero también en el Estado, especialmente en el diseño, implementación y evaluación de políticas públicas. Sin embargo, la temprana institucionalización de la enseñanza universitaria y la investigación académica de la sociología en el país (con antecedentes que se remontan más de un siglo atrás) contrasta con una profesionalización tardía y difusa \parencite{1512-SHILS1971}.

La historia de la sociología se ha consolidado en el país como un campo maduro de investigación. En este sentido, en los últimos años se ha logrado una mejor reconstrucción intelectual e institucional de la dinámica de la sociología como campo científico. Los trabajos más recientes han desplegado un conjunto de herramientas teóricas y metodológicas innovadoras, en línea con los debates internacionales más actuales \parencite{1559-CHAPOULIE2009,1562-COLLYER2021,1563-FLECK2015}. Esta senda de indagación recomienda centrarse en los siguientes ejes: Contextos, Instituciones, Instrumentos, Ideas y Actores. De esta manera, en un escenario de nuevos interrogantes sobre la historia de la sociología, un conjunto de trabajos han contribuido desde la década del 2000 a dar novedosas respuestas y generar nuevas preguntas sobre la especificidad del campo sociológico en Argentina.

El libro de Esteban Vila forma parte de este programa de renovación teórica y metodológica, que pudo superar las perspectivas previas y ofrecer así una reinterpretación sobre los relatos \enquote{clásicos} del pasado \parencite{1547-NOE2005,1550-BLOIS2018,1565-BLANCO2006}(González Bollo, 1999; Pereyra, 2007, 2010; Pereyra y Lazarte, 2022). Esta nueva generación de historiadores e historiadoras de la sociología argentina impulsaron nuevos espacios de discusión de ideas. La producción y los intercambios en el seno del Grupo de Estudio sobre la Historia y la Enseñanza de la Sociología (GEHES), del Instituto de Investigaciones Gino Germani (IIGG), en Buenos Aires, le permitieron al autor una mejor comprensión de la dinámica de la sociología argentina, desde una perspectiva histórica y regional.

La mayor parte de los capítulos de este libro fue presentada como borradores de trabajo en seminarios internos, jornadas y congresos; y sus ideas aparecieron entremezcladas en las clases de Sociología Argentina y Latinoamericana, que Esteban dicta en la Carrera de Sociología de la Universidad de Buenos Aires (UBA). Por lo cual, los textos atravesaron la mirada crítica de estudiantes de grado y posgrado, que a la larga enriquecieron el resultado. No obstante, no puede dejar de mencionarse que los diálogos que Esteban pudo establecer con diferentes equipos de trabajo liderados por Alejandro Blanco, en Quilmes; Juan Jesús Morales, en Santiago de Chile, y la siempre atenta mirada cordobesa de Ezequiel Grisendi, Severino Fernández y Pablo Requena posibilitaron fortalecer el conjunto de herramientas heurísticas desplegadas a la hora de reconstruir esas historias.

El trabajo reconstruye las trayectorias intelectuales y académicas de Ricardo Levene, Raúl Orgaz, Guillermo Terrera, Juan Carlos Agulla, Torcuato Di Tella, Atilio Borón y Emilio de Ípola. Ellos reúnen condiciones diversas de prestigio, reconocimiento y acumulación de capitales simbólicos dentro del campo nativo de la sociología. Los dos primeros forman parte de la generación pre-germaniana. El resto construyó su carrera dentro de la disciplina cuando la sociología ya había avanzado en su proceso de institucionalización y profesionalización. Sin embargo, Terrera remite a estilos de obrar y pensar que aluden al período previo. Di Tella, Borón y de Ipola pueden ser considerados como sociólogos modernos. Mientras que Agulla resulta un caso con más problemas para ubicarlo en alguna de estas definiciones. Puede ser quizás clasificado como un intelectual en transición hacia la sociología científica.

Ahora bien, Levene y Orgaz fueron importantes e influyentes figuras en su tiempo, pero luego no recibieron la atención necesaria tanto de los historiadores e historiadoras de la sociología así como de las nuevas generaciones de sociólogos y sociólogas. Por eso, pueden clasificarse como \enquote{recordados-olvidados} (\emph{remembered- forgotten}) \parencite{280-BARGHEER2024}(Mc Gail, 2021). El caso contrario es Terrera, que sería un \enquote{olvidado-olvidado} (\emph{forgotten-forgotten}), ya que no recibió gran reconocimiento durante su vida pero tampoco sus ideas fueron recuperadas por alguna tradición sociológica posterior. De nuevo, el caso de Agulla es más difuso porque, aunque podría situarse en esa misma categoría, sería necesario ampliar la discusión. Los casos de Borón, Di Tella y de Ípola son más cercanos en el tiempo. Podrían ser \enquote{recordados-recordados}, aunque habrá que ver si derivan en algún proceso de olvido. Para que sus ideas sigan desarrollándose deberán hallar una oferta estructurada de lecturas y una demanda de lectores que aspiren a encontrar en ellas nuevas interpretaciones del mundo social. Deberán encontrar fervientes seguidores que canonicen sus ideas y discutan, difundan y continúen su obra.

El autor aborda estas trayectorias a través de innovaciones teóricas (con su correspondiente cambio de enfoque), que se ven a su vez acompañadas por una renovación metodológica. A través de estos cambios, se hace ostensible cómo la estrategia tradicional de la entrevista y el testimonio de los protagonistas va dejando paso a reflexiones más fundamentadas en documentos de vida, análisis de registros institucionales y de redes. Con mayor frecuencia aparecen estudios sobre contextos de producción y circulación de las ideas, que incluyen la reconstrucción de trayectorias, análisis de revistas, materiales curriculares y de enseñanza de la sociología. De a poco, el uso de archivos personales e institucionales, más allá de sus déficits de acceso y disponibilidad, se hace presente en los trabajos de investigación sobre la historia de la sociología en Argentina.

De esta manera, en las últimas dos décadas, en el campo de la historia de la sociología argentina se han logrado avances muy relevantes. Uno de los puntos que debe remarcarse es la desmitificación de los relatos fundacionales. Sin embargo, los cambios más significativos en la reciente interpretación histórica tienen que ver con tres niveles: la fijación de nuevas periodizaciones, la ampliación de los casos institucionales y la incorporación de nuevas figuras. Por un lado, los trabajos recientes insisten en cuestionar la fecha de 1957, cuando se creó el Departamento de Sociología de la UBA, como un punto de inicio, incorporando al análisis otros momentos y períodos, como la primera cátedra (1898) o el Instituto de Sociología (1940). Por otro lado, se logró un mayor conocimiento sobre las diferentes experiencias de formación universitaria de sociólogos y sociólogas en Argentina. Así, se ha ampliado el lente espacial, lo que posibilitó la recuperación de otras trayectorias institucionales en Buenos Aires, más allá de la UBA, como la Universidad Católica Argentina, la Universidad del Salvador y la Universidad de Belgrano; así como también en diferentes ciudades y geografías de Argentina: Córdoba, Mendoza, Santa Fe, La Plata, San Juan, Mar del Plata, Santiago del Estero y Tucumán.

Finalmente, los relatos sobre la historia de la disciplina que se concentran en la figura de Gino Germani, por cierto renovados y cada vez más documentados, comienzan a compartir espacio con la irrupción de trabajos sobre otros autores y liderazgos institucionales que merecen atención. Por ejemplo, se han sucedido trabajos sobre Ricardo Levene \parencite{1530-RAJMANOVICH2016}, Raúl Orgaz \parencite{1528-REQUENA2010}(Grisendi, 2011), Juan Carlos Agulla (Grisendi, 2013), Miguel Figueroa Román (Pereyra, 2012), José Enrique Miguens (Giorgi y Aramburu, 2013); Alejandro Bunge (González Bollo, 2012); Francisco Ayala \parencite{1619-ESCOBAR2011}; Francisco Delich \parencite{1580-CASCO2018}; y Juan Carlos Portantiero \parencite{1581-CASCO2019,1582-CASCO2020}. También debe valorarse el rescate de figuras femeninas, como el caso de Angélica Mendoza \parencite{1623-FICCARDI2022}, Ángela Romera Vera \parencite{1620-ESCOBAR2016} o la investigación sobre sociólogas mujeres que desarrolla el equipo de las \emph{Pioneras. Mujeres de la sociología} \parencite{1564-BLANCO2019}.

En esta línea, el libro \emph{Senderos sociológicos} de Esteban Vila nos brinda un panorama sobre la diversidad de los proyectos y los estilos de trabajo de la sociología en Argentina a lo largo de prácticamente los últimos cien años. El libro reconstruye y analiza las trayectorias de siete destacados sociólogos argentinos del siglo~XX. En otro trabajo (Pereyra, en González Bollo, 2012: 129-143) ya se ha ofrecido una reflexión sobre la historia y los desafíos metodológicos del uso del género biográfico en las ciencias sociales. Meccia (2020) enriqueció y actualizó la discusión, con más datos y lucidez. Recuperando esa tradición, el autor no ha traicionado el mandato de Wright Mills que, en \citetitle{1451-WRIGHTMILLS1956} (1959), reclamaba que los sociólogos y las sociólogas no renuncien a conectar biografía e historia.

De esta forma, Esteban reconstruye una serie de trayectorias individuales entendidas como procesos de transformación biográfica de larga duración dentro de un orden social dinámico y cambiante. Por lo cual, convirtió este ejercicio en una herramienta realmente efectiva para reconstruir la historia de la sociología, especialmente desde una perspectiva institucional. A lo largo del texto, siguió atentamente la perspectiva bourdieusiana al considerar a las trayectorias como una secuencia de posiciones que ocupa un agente o grupo en un espacio que se transforma constantemente. En este sentido, pudo dar cuenta de algunos procesos históricos del devenir de la sociología argentina y encontrar en esas biografías un conjunto de procesos sociales que las exceden. Se reconstruyeron procesos explicativos de los motivos concretos por los que agentes con propiedades sociales y representaciones diversas realizan apuestas en campos institucionales y profesionales.

Se espera que los lectores y las lectoras de estas páginas puedan disfrutar de este libro. Es un texto que cumple con creces el objetivo de toda historia de la sociología. Por un lado, contribuye a reconstruir la memoria del pasado. Pero al mismo tiempo, tiene un sentido contra-celebratorio, ya que buscó recuperar trayectorias silenciadas, olvidadas y marginales. Con todo ello, Esteban puede mostrar la pluralidad de tradiciones y experiencias de la sociología argentina a lo largo de diferentes períodos. La historia de la sociología tiene como aspiración y desafío recuperar y ensamblar todas las voces posibles. Esteban lo ha logrado. Enhorabuena.

\chapter[Referencias]{Referencias}
\Author{Referencias}
\printbibliography[heading=none]

\chaptermark{Índice de autoras y autores del aparato bibliográfico}
\addcontentsline{toc}{chapter}{Índice de autoras y autores del aparato bibliográfico}

\Author{Índice de autoras y autores del aparato bibliográfico}
\raggedright
{\small
	\printindex[names]
}
\justifying


\input{./articulos/colofon.tex}

\newpage
\pagestyle{empty}
{\textcolor{white}{.}}

\end{document}